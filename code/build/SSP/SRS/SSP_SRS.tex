\documentclass[12pt]{article}
\usepackage{fontspec}
\usepackage{fullpage}
\usepackage{hyperref}
\hypersetup{bookmarks=true,colorlinks=true,linkcolor=red,citecolor=blue,filecolor=magenta,urlcolor=cyan}
\usepackage{amsmath}
\usepackage{amssymb}
\usepackage{mathtools}
\usepackage{unicode-math}
\usepackage{tabu}
\usepackage{longtable}
\usepackage{booktabs}
\usepackage{caption}
\usepackage{graphics}
\usepackage{enumitem}
\usepackage{filecontents}
\usepackage[backend=bibtex]{biblatex}
\usepackage{url}
\setmathfont{Latin Modern Math}
\newcommand{\gt}{\ensuremath >}
\newcommand{\lt}{\ensuremath <}
\global\tabulinesep=1mm
\newlist{symbDescription}{description}{1}
\setlist[symbDescription]{noitemsep, topsep=0pt, parsep=0pt, partopsep=0pt}
\bibliography{bibfile}
\title{Software Requirements Specification for Slope Stability analysis Program}
\author{Henry Frankis and Brooks MacLachlan}
\begin{document}
\maketitle
\tableofcontents
\newpage
\section{Reference Material}
\label{Sec:RefMat}
This section records information for easy reference.

\subsection{Table of Units}
\label{Sec:ToU}
The unit system used throughout is SI (Système International d'Unités). In addition to the basic units, several derived units are also used. For each unit, \hyperref[Table:ToU]{Tab: ToU} lists the symbol, a description and the SI name.

\begin{longtable}{l l l}
\toprule
\textbf{Symbol} & \textbf{Description} & \textbf{SI Name}
\\
\midrule
\endhead
${{}^{\circ}}$ & angle & degree
\\
${\text{kg}}$ & mass & kilogram
\\
${\text{m}}$ & length & metre
\\
${\text{N}}$ & force & newton
\\
${\text{Pa}}$ & pressure & pascal
\\
${\text{s}}$ & time & second
\\
\bottomrule
\caption{Table of Units}
\label{Table:ToU}
\end{longtable}
\subsection{Table of Symbols}
\label{Sec:ToS}
The symbols used in this document are summarized in \hyperref[Table:ToS]{Tab: ToS} along with their units. Throughout the document, a subscript $i$ indicates that the value will be taken at, and analyzed at, a slice or slice interface composing the total slip mass. For vector quantities, the units shown are for each component of the vector.

\begin{longtabu}{l X[l] l}
\toprule
\textbf{Symbol} & \textbf{Description} & \textbf{Units}
\\
\midrule
\endhead
$\text{(x,y)}$ & Cartesian Position Coordinates: $y$ is considered parallel to the direction of the force of gravity and $x$ is considered perpendicular to $y$. & ${\text{m}}$
\\
$A$ & Area: A part of an object or surface. & ${\text{m}^{2}}$
\\
$\mathbf{a}$ & Acceleration: The rate of change of a body's velocity. & $\frac{\text{m}}{\text{s}^{2}}$
\\
$\mathbf{b}$ & Base Width of Slices: The width of each slice in the $x$-direction. & ${\text{m}}$
\\
${\mathbf{C}_{\text{den}}}$ & Proportionality Constant Denominator: Values for each slice that sum together to form the denominator of the interslice normal to shear force proportionality constant. & ${\text{N}}$
\\
${\mathbf{C}_{\text{num}}}$ & Proportionality Constant Numerator: Values for each slice that sum together to form the numerator of the interslice normal to shear force proportionality constant. & ${\text{N}}$
\\
$c'$ & Effective Cohesion: The internal pressure that sticks particles of soil together. & ${\text{Pa}}$
\\
$const_f$ & Decision on F: A Boolean decision on which form of f the user desires: constant if true, or half-sine if false. & --
\\
${F_{\text{n}}}$ & Total Normal Force: Component of a force in the normal direction. & ${\text{N}}$
\\
${F_{\text{rot}}}$ & Force Causing Rotation: A force in the direction of rotation. & ${\text{N}}$
\\
${F_{\text{S}}}$ & Factor of Safety: The global stability metric of a slip surface of a slope, defined as the ratio of resistive shear force to mobilized shear force. & --
\\
${{F_{\text{S}}}^{\text{min}}}$ & Minimum Factor of Safety: The minimum factor of safety associated with the critical slip surface. & --
\\
${F_{\text{t}}}$ & Tangential Force: Component of a force in the tangential direction. & ${\text{N}}$
\\
${F_{\text{x}}}$ & $x$-coordinate of the Force: The force acting in the $x$-direction. & ${\text{N}}$
\\
${F_{\text{y}}}$ & $y$-coordinate of the Force: The force acting in the $y$-direction. & ${\text{N}}$
\\
$\mathbf{F}$ & Force: An interaction that tends to produce change in the motion of an object. & ${\text{N}}$
\\
${{\mathbf{F}_{\text{x}}}^{\text{G}}}$ & Sums of the Interslice Normal Forces: The sums of the normal forces acting on each pair of adjacent interslice boundaries. & ${\text{N}}$
\\
${{\mathbf{F}_{\text{x}}}^{\text{H}}}$ & Sums of the Interslice Normal Water Forces: The sums of the normal water forces acting on each pair of adjacent interslice boundaries. & ${\text{N}}$
\\
$\mathbf{f}$ & Interslice Normal to Shear Force Ratio Variation Function: A function of distance in the $x$-direction that describes the variation of the interslice normal to shear ratio. & --
\\
$\mathbf{G}$ & Interslice Normal Forces: The forces per meter in the $z$-direction exerted between each pair of adjacent slices. & $\frac{\text{N}}{\text{m}}$
\\
$\mathbf{g}$ & Gravitational Acceleration: The approximate acceleration due to gravity on Earth at sea level. & $\frac{\text{m}}{\text{s}^{2}}$
\\
$\mathbf{H}$ & Interslice Normal Water Forces: The normal water forces per meter in the $z$-direction exerted in the $x$-direction between each pair of adjacent slices. & $\frac{\text{N}}{\text{m}}$
\\
$h$ & Height: The distance above a reference point for a point of interest. & ${\text{m}}$
\\
$\mathbf{h}$ & $y$-direction Heights of Slices: The heights in the $y$-direction from the base of each slice to the slope surface, at the $x$-direction midpoint of the slice. & ${\text{m}}$
\\
${\mathbf{h}^{\text{L}}}$ & Heights of the Left Side of Slices: The heights of the left side of each slice, assuming slice surfaces have negative slope. & ${\text{m}}$
\\
${\mathbf{h}^{\text{R}}}$ & Heights of the Right Side of Slices: The heights of the right side of each slice, assuming slice surfaces have negative slope. & ${\text{m}}$
\\
${\mathbf{h}_{\text{z}}}$ & Heights of Interslice Normal Forces: The heights in the $y$-direction of the interslice normal forces on each slice. & ${\text{m}}$
\\
${\mathbf{h}_{\text{z,w}}}$ & Heights of the Water Table: The heights in the $y$-direction from the base of each slice to the water table. & ${\text{m}}$
\\
$i$ & Index: A number representing a single slice. & --
\\
$\mathbf{\hat{j}}$ & Unit Vector: A vector that has a magnitude of one. & --
\\
${K_{\text{c}}}$ & Seismic Coefficient: The proportionality factor of force that weight pushes outwards; caused by seismic earth movements. & --
\\
$M$ & Moment: A measure of the tendency of a body to rotate about a specific point or axis. & $\text{N}\text{m}$
\\
$m$ & Mass: The quantity of matter in a body. & ${\text{kg}}$
\\
$\mathbf{N}$ & Normal Forces: The total reactive forces per meter in the $z$-direction for each slice of a soil surface subject to a body resting on it. & $\frac{\text{N}}{\text{m}}$
\\
$\mathbf{N'}$ & Effective Normal Forces: The forces per meter in the $z$-direction for each slice of a soil surface, subtracting pore water reactive force from total reactive force. & $\frac{\text{N}}{\text{m}}$
\\
$n$ & Number of Slices: The number of slices into which the slip surface is divided. & --
\\
$P$ & Resistive Shear Force: The Mohr Coulomb frictional force that describes the limit of mobilized shear force that can be withstood before failure. & ${\text{N}}$
\\
$\mathbf{P}$ & Resistive Shear Forces: The Mohr Coulomb frictional forces per meter in the $z$-direction for each slice that describe the limit of mobilized shear force the slice can withstand before failure. & $\frac{\text{N}}{\text{m}}$
\\
$p$ & Pressure: A force exerted over an area. & ${\text{Pa}}$
\\
$\mathbf{Q}$ & External Forces: The forces per meter in the $z$-direction acting into the surface from the midpoint of each slice. & $\frac{\text{N}}{\text{m}}$
\\
$\mathbf{R}$ & Resistive Shear Forces Without the Influence of Interslice Forces: The resistive shear forces per meter without the influence of interslice forces in the $z$-direction for each slice. & $\frac{\text{N}}{\text{m}}$
\\
$r$ & Length of the Moment Arm: The distance between a force causing rotation and the axis of rotation. & ${\text{m}}$
\\
$\mathbf{r}$ & Position Vector: The a vector from the origin of the Cartesian coordinate system defined to the point where the force is applied. & ${\text{m}}$
\\
$S$ & Mobilized Shear Force: The shear force in the direction of potential motion. & ${\text{N}}$
\\
$\mathbf{S}$ & Mobilized Shear Forces: The mobilized shear forces per meter in the $z$-direction for each slice. & $\frac{\text{N}}{\text{m}}$
\\
$\mathbf{T}$ & Mobilized Shear Forces Without the Influence of Interslice Forces: The mobilized shear forces per meter without the influence of interslice forces in the $z$-direction for each slice. & $\frac{\text{N}}{\text{m}}$
\\
${\mathbf{U}_{\text{b}}}$ & Base Hydrostatic Forces: The forces per meter in the $z$-direction from water pressure within each slice. & $\frac{\text{N}}{\text{m}}$
\\
${\mathbf{U}_{\text{g}}}$ & Surface Hydrostatic Forces: The forces per meter in the $z$-direction from water pressure acting into each slice from standing water on the slope surface. & $\frac{\text{N}}{\text{m}}$
\\
$u$ & Pore Pressure: The pressure that comes from water within the soil. & ${\text{Pa}}$
\\
$\mathbf{u}$ & Displacement: The change in an object's location relative to a reference point. & ${\text{m}}$
\\
$V$ & Volume: The amount of space that a substance or object occupies. & ${\text{m}^{3}}$
\\
${\mathbf{V}_{\text{dry}}}$ & Volumes of Dry Soil: The amount of space occupied by dry soil for each slice. & ${\text{m}^{3}}$
\\
${\mathbf{V}_{\text{sat}}}$ & Volumes of Saturated Soil: The amount of space occupied by saturated soil for each slice. & ${\text{m}^{3}}$
\\
$v$ & Local Index: Used as a bound variable index in calculations. & --
\\
$W$ & Weight: The gravitational force acting on an object. & ${\text{N}}$
\\
$\mathbf{W}$ & Weights: The downward force per meter in the $z$-direction on each slice caused by gravity. & $\frac{\text{N}}{\text{m}}$
\\
$\mathbf{X}$ & Interslice Shear Forces: The shear forces per meter in the $z$-direction exerted between adjacent slices. & $\frac{\text{N}}{\text{m}}$
\\
$x$ & $x$-coordinate: The $x$-coordinate in the Cartesian coordinate system. & ${\text{m}}$
\\
${{x_{\text{slip}}}^{\text{maxEtr}}}$ & Maximum Entry $x$-coordinate: The maximum potential $x$-coordinate for the entry point of a slip surface. & ${\text{m}}$
\\
${{x_{\text{slip}}}^{\text{maxExt}}}$ & Maximum Exit $x$-coordinate: The maximum potential $x$-coordinate for the exit point of a slip surface. & ${\text{m}}$
\\
${{x_{\text{slip}}}^{\text{minEtr}}}$ & Minimum Entry $x$-coordinate: The minimum potential $x$-coordinate for the entry point of a slip surface. & ${\text{m}}$
\\
${{x_{\text{slip}}}^{\text{minExt}}}$ & Minimum Exit $x$-coordinate: The minimum potential $x$-coordinate for the exit point of a slip surface. & ${\text{m}}$
\\
${\mathbf{x}_{\text{cs}}}\text{,}{\mathbf{y}_{\text{cs}}}$ & Critical Slip Surface Coordinates: The set of $x$-coordinates and $y$-coordinates that describe the vertices of the critical slip surface. & ${\text{m}}$
\\
${\mathbf{x}_{\text{slip}}}$ & $x$-coordinates of the Slip Surface: $x$-coordinates of points on the slip surface. & ${\text{m}}$
\\
${\mathbf{x}_{\text{slope}}}$ & $x$-coordinates of the Slope: $x$-coordinates of points on the soil slope. & ${\text{m}}$
\\
${\mathbf{x}_{\text{wt}}}$ & $x$-coordinates of the Water Table: X-positions of the water table. & ${\text{m}}$
\\
$y$ & $y$-coordinate: The $y$-coordinate in the Cartesian coordinate system. & ${\text{m}}$
\\
${{y_{\text{slip}}}^{\text{max}}}$ & Maximum $y$-coordinate: The maximum potential $y$-coordinate of a point on a slip surface. & ${\text{m}}$
\\
${{y_{\text{slip}}}^{\text{min}}}$ & Minimum $y$-coordinate: The minimum potential $y$-coordinate of a point on a slip surface. & ${\text{m}}$
\\
${\mathbf{y}_{\text{slip}}}$ & $y$-coordinates of the Slip Surface: Heights of the slip surface. & ${\text{m}}$
\\
${\mathbf{y}_{\text{slope}}}$ & $y$-coordinates of the Slope: $y$-coordinates of points on the soil slope. & ${\text{m}}$
\\
${\mathbf{y}_{\text{wt}}}$ & $y$-coordinates of the Water Table: Heights of the water table. & ${\text{m}}$
\\
$z$ & $z$-coordinate: The $z$-coordinate in the Cartesian coordinate system. & ${\text{m}}$
\\
$\mathbf{α}$ & Base Angles: The angles between the base of each slice and the horizontal. & ${{}^{\circ}}$
\\
$\mathbf{β}$ & Surface Angles: The angles between the surface of each slice and the horizontal. & ${{}^{\circ}}$
\\
$γ$ & Specific Weight: The weight per unit volume. & $\frac{\text{N}}{\text{m}^{3}}$
\\
${γ_{\text{dry}}}$ & Soil Dry Unit Weight: The weight of a dry soil/ground layer divided by the volume of the layer. & $\frac{\text{N}}{\text{m}^{3}}$
\\
${γ_{\text{sat}}}$ & Soil Saturated Unit Weight: The weight of saturated soil/ground layer divided by the volume of the layer. & $\frac{\text{N}}{\text{m}^{3}}$
\\
${γ_{w}}$ & Unit Weight of Water: The weight of one cubic meter of water. & $\frac{\text{N}}{\text{m}^{3}}$
\\
$λ$ & Proportionality Constant: The ratio of the interslice normal to the interslice shear force. & --
\\
$π$ & Ratio of Circumference to Diameter for Any Circle: The ratio of a circle's circumference to its diameter. & --
\\
$ρ$ & Density: The mass per unit volume. & $\frac{\text{kg}}{\text{m}^{3}}$
\\
$σ$ & Total Normal Stress: The total force per area acting on the soil mass. & ${\text{Pa}}$
\\
$σ'$ & Effective Stress: The stress in a soil mass that is effective in causing volume changes and mobilizes the shear strength arising from friction; represents the average stress carried by the soil skeleton. & ${\text{Pa}}$
\\
${σ_{N}}'$ & Effective Normal Stress: The normal stress in a soil mass that is effective in causing volume changes; represents the average normal stress carried by the soil skeleton. & ${\text{Pa}}$
\\
$τ$ & Tangential Stress: The shear force per unit area. & ${\text{Pa}}$
\\
${τ^{\text{f}}}$ & Shear Strength: The strength of a material against shear failure. & ${\text{Pa}}$
\\
$\mathbf{τ}$ & Torque: A twisting force that tends to cause rotation. & $\text{N}\text{m}$
\\
$Υ$ & Minimization Function: Generic minimization function or algorithm. & --
\\
$\mathbf{Φ}$ & First Function for Incorporating Interslice Forces Into Shear Force: The function for converting resistive shear without the influence of interslice forces, to a calculation considering the interslice forces. & --
\\
$φ'$ & Effective Angle of Friction: The angle of inclination with respect to the horizontal axis of the Mohr-Coulomb shear resistance line. & ${{}^{\circ}}$
\\
$\mathbf{Ψ}$ & Second Function for Incorporating Interslice Forces Into Shear Force: The function for converting mobile shear without the influence of interslice forces, to a calculation considering the interslice forces. & --
\\
$\mathbf{ω}$ & Imposed Load Angles: The angles between the external force acting into the surface of each slice and the vertical. & ${{}^{\circ}}$
\\
${\mathbf{ℓ}_{b}}$ & Total Base Lengths of Slices: The lengths of each slice in the direction parallel to the slope of the base. & ${\text{m}}$
\\
${\mathbf{ℓ}_{s}}$ & Surface Lengths of Slices: The lengths of each slice in the direction parallel to the slope of the surface. & ${\text{m}}$
\\
\bottomrule
\caption{Table of Symbols}
\label{Table:ToS}
\end{longtabu}
\subsection{Abbreviations and Acronyms}
\label{Sec:TAbbAcc}
\begin{longtable}{l l}
\toprule
\textbf{Abbreviation} & \textbf{Full Form}
\\
\midrule
\endhead
2D & Two-Dimensional
\\
3D & Three-Dimensional
\\
A & Assumption
\\
DD & Data Definition
\\
GD & General Definition
\\
GS & Goal Statement
\\
IM & Instance Model
\\
LC & Likely Change
\\
PS & Physical System Description
\\
R & Requirement
\\
SRS & Software Requirements Specification
\\
SSP & Slope Stability analysis Program
\\
TM & Theoretical Model
\\
UC & Unlikely Change
\\
Uncert. & Typical Uncertainty
\\
\bottomrule
\caption{Abbreviations and Acronyms}
\label{Table:TAbbAcc}
\end{longtable}
\section{Introduction}
\label{Sec:Intro}
A slope of geological mass, composed of soil and rock and sometimes water, is subject to the influence of gravity on the mass. This can cause instability in the form of soil or rock movement. The effects of soil or rock movement can range from inconvenient to seriously hazardous, resulting in significant life and economic losses. Slope stability is of interest both when analysing natural slopes, and when designing an excavated slope. Slope stability analysis is the assessment of the safety of a slope, identifying the surface most likely to experience slip and an index of its relative stability known as the factor of safety.

The following section provides an overview of the Software Requirements Specification (SRS) for a slope stability analysis problem. The developed program will be referred to as the Slope Stability analysis Program (SSP). This section explains the purpose of this document, the scope of the requirements, the characteristics of the intended reader, and the organization of the document.

\subsection{Purpose of Document}
\label{Sec:DocPurpose}
The primary purpose of this document is to record the requirements of the Slope Stability analysis Program. Goals, assumptions, theoretical models, definitions, and other model derivation information are specified, allowing the reader to fully understand and verify the purpose and scientific basis of SSP. With the exception of system constraints in \hyperref[Sec:SysConstraints]{Section: System Constraints}, this SRS will remain abstract, describing what problem is being solved, but not how to solve it.

This document will be used as a starting point for subsequent development phases, including writing the design specification and the software verification and validation plan. The design document will show how the requirements are to be realized, including decisions on the numerical algorithms and programming environment. The verification and validation plan will show the steps that will be used to increase confidence in the software documentation and the implementation. Although the SRS fits in a series of documents that follow the so-called waterfall model, the actual development process is not constrained in any way. Even when the waterfall model is not followed, as Parnas and Clements point out \cite{parnasClements1986}, the most logical way to present the documentation is still to ``fake'' a rational design process.

\subsection{Scope of Requirements}
\label{Sec:ReqsScope}
The scope of the requirements includes stability analysis of a two-dimensional (2D) soil mass, composed of a single homogeneous layer with constant material properties. The soil mass is assumed to extend infinitely in the third dimension. The analysis will be at an instant in time; factors that may change the soil properties over time will not be considered.

\subsection{Characteristics of Intended Reader}
\label{Sec:ReaderChars}
Reviewers of this documentation should have an understanding of undergraduate level 4 physics and undergraduate level 2 or higher solid mechanics. It would be an asset to understand soil mechanics. The users of SSP can have a lower level of expertise, as explained in \hyperref[Sec:UserChars]{Section: User Characteristics}.

\subsection{Organization of Document}
\label{Sec:DocOrg}
The organization of this document follows the template for an SRS for scientific computing software proposed by Koothoor \cite{koothoor2013} as well as Smith and Lai \cite{smithLai2005}. The presentation follows the standard pattern of presenting goals, theories, definitions, and assumptions. For readers that would like a more bottom up approach, they can start reading the instance models in \hyperref[Sec:IMs]{Section: Instance Models} and trace back to find any additional information they require.

The goal statements (\hyperref[Sec:GoalStmt]{Section: Goal Statements}) are refined to the theoretical models and the theoretical models (\hyperref[Sec:TMs]{Section: Theoretical Models}) to the instance models (\hyperref[Sec:IMs]{Section: Instance Models}). The instance models provide the set of algebraic equations that must be solved.

\section{General System Description}
\label{Sec:GenSysDesc}
This section provides general information about the system. It identifies the interfaces between the system and its environment, describes the user characteristics, and lists the system constraints.

\subsection{System Context}
\label{Sec:SysContext}
\hyperref[Figure:sysCtxDiag]{Fig:sysCtxDiag} shows the system context. A circle represents an external entity outside the software. A rectangle represents the software system itself (SSP). Arrows are used to show the data flow between the system and its environment.

\begin{figure}
\begin{center}
\includegraphics[width=\textwidth]{../../../datafiles/SSP/SystemContextFigure.png}
\caption{System Context}
\label{Figure:sysCtxDiag}
\end{center}
\end{figure}
The responsibilities of the user and the system are as follows:

\begin{itemize}
\item{User Responsibilities}
\begin{itemize}
\item{Provide the input data related to the soil layer(s) and water table (if applicable), ensuring conformation to input data format required by SSP}
\item{Ensure that consistent units are used for input variables}
\item{Ensure required software assumptions (\hyperref[Sec:Assumps]{Section: Assumptions}) are appropriate for the problem to which the user is applying the software}
\end{itemize}
\item{SSP Responsibilities}
\begin{itemize}
\item{Detect data type mismatch, such as a string of characters input instead of a floating point number}
\item{Verify that the inputs satisfy the required physical and other data constraints (\hyperref[Sec:DataConstraints]{Section: Data Constraints})}
\item{Identify the critical slip surface within the possible input range}
\item{Find the factor of safety for the slope}
\item{Find the interslice normal force and shear force along the critical slip surface}
\end{itemize}
\end{itemize}
\subsection{User Characteristics}
\label{Sec:UserChars}
The end user of SSP should have an understanding of undergraduate Level 1 Calculus and Physics, and be familiar with soil and material properties, specifically effective cohesion, effective angle of friction, and unit weight.

\subsection{System Constraints}
\label{Sec:SysConstraints}
The Morgenstern-Price method \cite{morgenstern1965}, which involves dividing the slope into vertical slices, will be used to derive the equations for analysing the slope.

\section{Specific System Description}
\label{Sec:SpecSystDesc}
This section first presents the problem description, which gives a high-level view of the problem to be solved. This is followed by the solution characteristics specification, which presents the assumptions, theories, and definitions that are used.

\subsection{Problem Description}
\label{Sec:ProbDesc}
A system is needed to evaluate the factor of safety of a slope's slip surface and identify the critical slip surface of the slope, as well as the interslice normal force and shear force along the critical slip surface.

\subsubsection{Terminology and Definitions}
\label{Sec:TermDefs}
This subsection provides a list of terms that are used in the subsequent sections and their meaning, with the purpose of reducing ambiguity and making it easier to correctly understand the requirements.

\begin{itemize}
\item{Factor of safety: The global stability metric of a slip surface of a slope, defined as the ratio of resistive shear force to mobilized shear force.}
\item{Slip surface: A surface within a slope that has the potential to fail or displace due to load or other forces.}
\item{Critical slip surface: Slip surface of the slope that has the lowest factor of safety, and is therefore most likely to experience failure.}
\item{Water table: The upper boundary of a saturated zone in the ground.}
\item{Stress: The ratio of an applied force to a cross-sectional area.}
\item{Strain: A measure of deformation representing the displacement between particles in the body relative to a reference length.}
\item{Normal force: A force applied perpendicular to the plane of the material.}
\item{Shear force: A force applied parallel to the plane of the material.}
\item{Mobilized shear force: The shear force in the direction of potential motion, thus encouraging motion along the plane.}
\item{Resistive shear force: The shear force in the direction opposite to the direction of potential motion, thus hindering motion along the plane.}
\item{Effective forces and stresses: The normal force or normal stress carried by the soil skeleton, composed of the effective force or stress and the force or stress exerted by water.}
\item{Cohesion: An attractive force between adjacent particles that holds the matter together.}
\item{Isotropy: A condition where the value of a property is independent of the direction in which it is measured.}
\item{Plane strain: A condition where the resultant stresses in one of the directions of a  three-dimensional material can be approximated as zero. This condition results when a body is constrained to not deform in one direction, or when the length of one dimension of the body dominates the others, to the point where it can be assumed as infinite. Stresses in the direction of the dominant dimension can be approximated as zero.}
\end{itemize}
\subsubsection{Physical System Description}
\label{Sec:PhysSyst}
The physical system of SSP, as shown in \hyperref[Figure:PhysicalSystem]{Fig:PhysicalSystem}, includes the following elements:

\begin{itemize}
\item[PS1:]{A slope comprised of one soil layer.}
\item[PS2:]{A water table, which may or may not exist.}
\end{itemize}
\begin{figure}
\begin{center}
\includegraphics[width=\textwidth]{../../../datafiles/SSP/PhysSyst.png}
\caption{An example slope for analysis by SSP, where the dashed line represents the water table}
\label{Figure:PhysicalSystem}
\end{center}
\end{figure}
Morgenstern-Price analysis \cite{morgenstern1965} of the slope involves representing the slope as a series of vertical slices. As shown in \hyperref[Figure:IndexConvention]{Fig:IndexConvention}, the index $i$ is used to denote a value for a single slice, and an interslice value at a given index $i$ refers to the value between slice $i$ and adjacent slice $i+1$.

\begin{figure}
\begin{center}
\includegraphics[width=\textwidth]{../../../datafiles/SSP/IndexConvention.png}
\caption{Index convention for slice and interslice values}
\label{Figure:IndexConvention}
\end{center}
\end{figure}
A free body diagram of the forces acting on a slice is displayed in \hyperref[Figure:ForceDiagram]{Fig:ForceDiagram}. The specific forces and symbols will be discussed in detail in \hyperref[Sec:GDs]{Section: General Definitions} and \hyperref[Sec:DDs]{Section: Data Definitions}.

\begin{figure}
\begin{center}
\includegraphics[width=\textwidth]{../../../datafiles/SSP/ForceDiagram.png}
\caption{Free body diagram of forces acting on a slice}
\label{Figure:ForceDiagram}
\end{center}
\end{figure}
\subsubsection{Goal Statements}
\label{Sec:GoalStmt}
Given the shape of the soil mass, the location of the water table, and the material properties of the soil, the goal statements are:

\begin{itemize}
\item[Identify-Crit-and-FS:\phantomsection\label{identifyCritAndFS}]{Identify the critical slip surface and the corresponding factor of safety.}
\item[Determine-Normal-Forces:\phantomsection\label{determineNormalF}]{Determine the interslice normal forces between each pair of vertical slices of the slope.}
\item[Determine-Shear-Forces:\phantomsection\label{determineShearF}]{Determine the interslice shear forces between each pair of vertical slices of the slope.}
\end{itemize}
\subsection{Solution Characteristics Specification}
\label{Sec:SolCharSpec}
The instance models that govern SSP are presented in \hyperref[Sec:IMs]{Section: Instance Models}. The information to understand the meaning of the instance models and their derivation is also presented, so that the instance models can be verified.

\subsubsection{Assumptions}
\label{Sec:Assumps}
This section simplifies the original problem and helps in developing the theoretical models by filling in the missing information for the physical system. The assumptions refine the scope by providing more detail.

\begin{itemize}
\item[Slip-Surface-Concave:\phantomsection\label{assumpSSC}]{The slip surface is concave with respect to the slope surface. The (${\mathbf{x}_{\text{slip}}}$, ${\mathbf{y}_{\text{slip}}}$) coordinates of a slip surface follow a concave up function. (RefBy: \hyperref[IM:crtSlpId]{IM: crtSlpId}.)}
\item[Factor-of-Safety:\phantomsection\label{assumpFOS}]{The factor of safety is assumed to be constant across the entire slip surface. (RefBy: \hyperref[GD:mobShr]{GD: mobShr}.)}
\item[Soil-Layer-Homogeneous:\phantomsection\label{assumpSLH}]{The soil mass is homogeneous, with consistent soil properties throughout. (RefBy: \hyperref[GD:sliceWght]{GD: sliceWght}, \hyperref[GD:resShr]{GD: resShr}, and \hyperref[LC_inhomogeneous]{LC: Calculate-Inhomogeneous-Soil-Layers}.)}
\item[Soil-Properties:\phantomsection\label{assumpSP}]{The soil properties are independent of dry or saturated conditions, with the exception of unit weight. (RefBy: \hyperref[GD:resShr]{GD: resShr}.)}
\item[Soil-Layers-Isotropic:\phantomsection\label{assumpSLI}]{The soil mass is treated as if the effective cohesion and effective angle of friction are isotropic properties. (RefBy: \hyperref[GD:resShr]{GD: resShr}.)}
\item[Interslice-Norm-Shear-Forces-Linear:\phantomsection\label{assumpINSFL}]{Following the assumption of Morgenstern and Price (\cite{morgenstern1965}), interslice normal forces and interslice shear forces have a proportional relationship, depending on a proportionality constant ($λ$) and a function ($\mathbf{f}$) describing variation depending on $x$ position. (RefBy: \hyperref[IM:nrmShrFor]{IM: nrmShrFor}, \hyperref[GD:normShrR]{GD: normShrR}, \hyperref[IM:fctSfty]{IM: fctSfty}, and \hyperref[UC_normshearlinear]{UC: Normal-And-Shear-Linear-Only}.)}
\item[Plane-Strain-Conditions:\phantomsection\label{assumpPSC}]{The slope and slip surface extends far into and out of the geometry ($z$ coordinate). This implies plane strain conditions, making 2D analysis appropriate. (RefBy: \hyperref[GD:srfWtrF]{GD: srfWtrF}, \hyperref[GD:sliceWght]{GD: sliceWght}, \hyperref[GD:resShr]{GD: resShr}, \hyperref[GD:effNormF]{GD: effNormF}, and \hyperref[GD:baseWtrF]{GD: baseWtrF}.)}
\item[Effective-Norm-Stress-Large:\phantomsection\label{assumpENSL}]{The effective normal stress is large enough that the shear strength to effective normal stress relationship can be approximated as a linear relationship. (RefBy: \hyperref[TM:equilibrium]{TM: equilibrium} and \hyperref[UC_2donly]{UC: 2D-Analysis-Only}.)}
\item[Surface-Base-Slice-between-Interslice-Straight-Lines:\phantomsection\label{assumpSBSBISL}]{The surface and base of a slice are approximated as straight lines. (RefBy: \hyperref[GD:srfWtrF]{GD: srfWtrF}, \hyperref[GD:sliceWght]{GD: sliceWght}, \hyperref[TM:mcShrStrgth]{TM: mcShrStrgth}, \hyperref[DD:slcHeight]{DD: slcHeight}, \hyperref[DD:angleB]{DD: angleB}, \hyperref[GD:baseWtrF]{GD: baseWtrF}, and \hyperref[DD:angleA]{DD: angleA}.)}
\item[Edge-Slices:\phantomsection\label{assumpES}]{The interslice forces at the 0th and $n$th interslice interfaces are zero. (RefBy: \hyperref[IM:nrmShrFor]{IM: nrmShrFor}, \hyperref[IM:intsliceFs]{IM: intsliceFs}, and \hyperref[IM:fctSfty]{IM: fctSfty}.)}
\item[Seismic-Force:\phantomsection\label{assumpSF}]{There is no seismic force acting on the slope. (RefBy: \hyperref[IM:nrmShrFor]{IM: nrmShrFor}, \hyperref[IM:fctSfty]{IM: fctSfty}, and \hyperref[LC_seismic]{LC: Calculate-Seismic-Force}.)}
\item[Surface-Load:\phantomsection\label{assumpSL}]{There is no imposed surface load, and therefore no external forces, acting on the slope. (RefBy: \hyperref[IM:nrmShrFor]{IM: nrmShrFor}, \hyperref[IM:fctSfty]{IM: fctSfty}, and \hyperref[LC_external]{LC: Calculate-External-Force}.)}
\item[Water-Intersects-Base-Edge:\phantomsection\label{assumpWIBE}]{The water table only intersects the base of a slice at an edge of the slice. (RefBy: \hyperref[GD:sliceWght]{GD: sliceWght} and \hyperref[GD:baseWtrF]{GD: baseWtrF}.)}
\item[Water-Intersects-Surface-Edge:\phantomsection\label{assumpWISE}]{The water table only intersects the slope surface at the edge of a slice. (RefBy: \hyperref[GD:srfWtrF]{GD: srfWtrF} and \hyperref[GD:sliceWght]{GD: sliceWght}.)}
\item[Negligible-Effect-Surface-Slope-Seismic:\phantomsection\label{assumpNESSS}]{The effect of the slope of the surface of the soil on the seismic force is assumed to be negligible. (RefBy: \hyperref[GD:momentEql]{GD: momentEql}.)}
\item[Hydrostatic-Force-Slice-Midpoint:\phantomsection\label{assumpHFSM}]{The resultant surface hydrostatic forces act into the midpoint of each slice surface and the resultant base hydrostatic forces act into the midpoint of each slice base. (RefBy: \hyperref[GD:srfWtrF]{GD: srfWtrF}, \hyperref[GD:momentEql]{GD: momentEql}, and \hyperref[GD:baseWtrF]{GD: baseWtrF}.)}
\end{itemize}
\subsubsection{Theoretical Models}
\label{Sec:TMs}
This section focuses on the general equations and laws that SSP is based on.

\vspace{\baselineskip}
\noindent
\begin{minipage}{\textwidth}
\begin{tabular}{>{\raggedright}p{0.13\textwidth}>{\raggedright\arraybackslash}p{0.82\textwidth}}
\toprule \textbf{Refname} & \textbf{TM:factOfSafety}
\phantomsection 
\label{TM:factOfSafety}
\\ \midrule \\
Label & Factor of safety
        
\\ \midrule \\
Equation & \begin{displaymath}
           {F_{\text{S}}}=\frac{P}{S}
           \end{displaymath}
\\ \midrule \\
Description & \begin{symbDescription}
              \item{${F_{\text{S}}}$ is the factor of safety (Unitless)}
              \item{$P$ is the resistive shear force (${\text{N}}$)}
              \item{$S$ is the mobilized shear force (${\text{N}}$)}
              \end{symbDescription}
\\ \midrule \\
Source & \cite{fredlund1977}
         
\\ \midrule \\
RefBy & \hyperref[GD:mobShr]{GD: mobShr}
        
\\ \bottomrule
\end{tabular}
\end{minipage}
\vspace{\baselineskip}
\noindent
\begin{minipage}{\textwidth}
\begin{tabular}{>{\raggedright}p{0.13\textwidth}>{\raggedright\arraybackslash}p{0.82\textwidth}}
\toprule \textbf{Refname} & \textbf{TM:equilibrium}
\phantomsection 
\label{TM:equilibrium}
\\ \midrule \\
Label & Equilibrium
        
\\ \midrule \\
Equation & \begin{displaymath}
           \displaystyle\sum{{F_{\text{x}}}}=\displaystyle\sum{{F_{\text{y}}}}=\displaystyle\sum{M}=0
           \end{displaymath}
\\ \midrule \\
Description & \begin{symbDescription}
              \item{${F_{\text{x}}}$ is the $x$-coordinate of the force (${\text{N}}$)}
              \item{${F_{\text{y}}}$ is the $y$-coordinate of the force (${\text{N}}$)}
              \item{$M$ is the moment ($\text{N}\text{m}$)}
              \end{symbDescription}
\\ \midrule \\
Notes & For a body in static equilibrium, the net forces and moments acting on the body will cancel out. Assuming a 2D problem (\hyperref[assumpENSL]{A: Effective-Norm-Stress-Large}), the $x$-coordinate of the force ${F_{\text{x}}}$ and $y$-coordinate of the force ${F_{\text{y}}}$ will be equal to $0$. All forces and their distance from the chosen point of rotation will create a net moment equal to $0$.
        
\\ \midrule \\
Source & \cite{fredlund1977}
         
\\ \midrule \\
RefBy & \hyperref[GD:normForcEq]{GD: normForcEq}, \hyperref[GD:momentEql]{GD: momentEql}, and \hyperref[GD:bsShrFEq]{GD: bsShrFEq}
        
\\ \bottomrule
\end{tabular}
\end{minipage}
\vspace{\baselineskip}
\noindent
\begin{minipage}{\textwidth}
\begin{tabular}{>{\raggedright}p{0.13\textwidth}>{\raggedright\arraybackslash}p{0.82\textwidth}}
\toprule \textbf{Refname} & \textbf{TM:mcShrStrgth}
\phantomsection 
\label{TM:mcShrStrgth}
\\ \midrule \\
Label & Mohr-Coulumb shear strength
        
\\ \midrule \\
Equation & \begin{displaymath}
           {τ^{\text{f}}}={σ_{N}}' \tan\left(φ'\right)+c'
           \end{displaymath}
\\ \midrule \\
Description & \begin{symbDescription}
              \item{${τ^{\text{f}}}$ is the shear strength (${\text{Pa}}$)}
              \item{${σ_{N}}'$ is the effective normal stress (${\text{Pa}}$)}
              \item{$φ'$ is the effective angle of friction (${{}^{\circ}}$)}
              \item{$c'$ is the effective cohesion (${\text{Pa}}$)}
              \end{symbDescription}
\\ \midrule \\
Notes & In this model the shear strength ${τ^{\text{f}}}$ is proportional to the product of the effective normal stress ${σ_{N}}'$ on the plane with its static friction in the angular form $\tan\left(φ'\right)$. The ${τ^{\text{f}}}$ versus ${σ_{N}}'$ relationship is not truly linear, but assuming the effective normal forces is strong enough, it can be approximated with a linear fit (\hyperref[assumpSBSBISL]{A: Surface-Base-Slice-between-Interslice-Straight-Lines}) where the effective cohesion $c'$ represents the ${τ^{\text{f}}}$ intercept of the fitted line.
        
\\ \midrule \\
Source & \cite{fredlund1977}
         
\\ \midrule \\
RefBy & \hyperref[GD:resShr]{GD: resShr}
        
\\ \bottomrule
\end{tabular}
\end{minipage}
\vspace{\baselineskip}
\noindent
\begin{minipage}{\textwidth}
\begin{tabular}{>{\raggedright}p{0.13\textwidth}>{\raggedright\arraybackslash}p{0.82\textwidth}}
\toprule \textbf{Refname} & \textbf{TM:effStress}
\phantomsection 
\label{TM:effStress}
\\ \midrule \\
Label & Effective stress
        
\\ \midrule \\
Equation & \begin{displaymath}
           σ'=σ-u
           \end{displaymath}
\\ \midrule \\
Description & \begin{symbDescription}
              \item{$σ'$ is the effective stress (${\text{Pa}}$)}
              \item{$σ$ is the total normal stress (${\text{Pa}}$)}
              \item{$u$ is the pore pressure (${\text{Pa}}$)}
              \end{symbDescription}
\\ \midrule \\
Notes & $σ$ is defined in \hyperref[DD:normStress]{DD: normStress}.
        
\\ \midrule \\
Source & \cite{fredlund1977}
         
\\ \midrule \\
RefBy & \hyperref[GD:effNormF]{GD: effNormF}
        
\\ \bottomrule
\end{tabular}
\end{minipage}
\vspace{\baselineskip}
\noindent
\begin{minipage}{\textwidth}
\begin{tabular}{>{\raggedright}p{0.13\textwidth}>{\raggedright\arraybackslash}p{0.82\textwidth}}
\toprule \textbf{Refname} & \textbf{TM:NewtonSecLawMot}
\phantomsection 
\label{TM:NewtonSecLawMot}
\\ \midrule \\
Label & Newton's second law of motion
        
\\ \midrule \\
Equation & \begin{displaymath}
           \mathbf{F}=m \mathbf{a}
           \end{displaymath}
\\ \midrule \\
Description & \begin{symbDescription}
              \item{$\mathbf{F}$ is the force (${\text{N}}$)}
              \item{$m$ is the mass (${\text{kg}}$)}
              \item{$\mathbf{a}$ is the acceleration ($\frac{\text{m}}{\text{s}^{2}}$)}
              \end{symbDescription}
\\ \midrule \\
Notes & The net force $\mathbf{F}$ on a body is proportional to the acceleration $\mathbf{a}$ of the body, where $m$ denotes the mass of the body as the constant of proportionality.
        
\\ \midrule \\
Source & --
         
\\ \midrule \\
RefBy & \hyperref[GD:weight]{GD: weight}
        
\\ \bottomrule
\end{tabular}
\end{minipage}
\subsubsection{General Definitions}
\label{Sec:GDs}
This section collects the laws and equations that will be used to build the instance models.

\vspace{\baselineskip}
\noindent
\begin{minipage}{\textwidth}
\begin{tabular}{>{\raggedright}p{0.13\textwidth}>{\raggedright\arraybackslash}p{0.82\textwidth}}
\toprule \textbf{Refname} & \textbf{GD:normForcEq}
\phantomsection 
\label{GD:normForcEq}
\\ \midrule \\
Label & Normal force equilibrium
        
\\ \midrule \\
Units & $\frac{\text{N}}{\text{m}}$
        
\\ \midrule \\
Equation & \begin{displaymath}
           {\mathbf{N}}_{i}=\left({\mathbf{W}}_{i}-{\mathbf{X}}_{i-1}+{\mathbf{X}}_{i}+{\mathbf{U}_{\text{g},i}} \cos\left({\mathbf{β}}_{i}\right)+{\mathbf{Q}}_{i} \cos\left({\mathbf{ω}}_{i}\right)\right) \cos\left({\mathbf{α}}_{i}\right)+\left(-{K_{\text{c}}} {\mathbf{W}}_{i}-{\mathbf{G}}_{i}+{\mathbf{G}}_{i-1}-{\mathbf{H}}_{i}+{\mathbf{H}}_{i-1}+{\mathbf{U}_{\text{g},i}} \sin\left({\mathbf{β}}_{i}\right)+{\mathbf{Q}}_{i} \sin\left({\mathbf{ω}}_{i}\right)\right) \sin\left({\mathbf{α}}_{i}\right)
           \end{displaymath}
\\ \midrule \\
Description & \begin{symbDescription}
              \item{$\mathbf{N}$ is the normal forces ($\frac{\text{N}}{\text{m}}$)}
              \item{$i$ is the index (Unitless)}
              \item{$\mathbf{W}$ is the weights ($\frac{\text{N}}{\text{m}}$)}
              \item{$\mathbf{X}$ is the interslice shear forces ($\frac{\text{N}}{\text{m}}$)}
              \item{${\mathbf{U}_{\text{g}}}$ is the surface hydrostatic forces ($\frac{\text{N}}{\text{m}}$)}
              \item{$\mathbf{β}$ is the surface angles (${{}^{\circ}}$)}
              \item{$\mathbf{Q}$ is the external forces ($\frac{\text{N}}{\text{m}}$)}
              \item{$\mathbf{ω}$ is the imposed load angles (${{}^{\circ}}$)}
              \item{$\mathbf{α}$ is the base angles (${{}^{\circ}}$)}
              \item{${K_{\text{c}}}$ is the seismic coefficient (Unitless)}
              \item{$\mathbf{G}$ is the interslice normal forces ($\frac{\text{N}}{\text{m}}$)}
              \item{$\mathbf{H}$ is the interslice normal water forces ($\frac{\text{N}}{\text{m}}$)}
              \end{symbDescription}
\\ \midrule \\
Notes & This equation satisfies \hyperref[TM:equilibrium]{TM: equilibrium} in the normal direction. $\mathbf{W}$ is defined in \hyperref[GD:sliceWght]{GD: sliceWght}, ${\mathbf{U}_{\text{g}}}$ is defined in \hyperref[GD:srfWtrF]{GD: srfWtrF}, $\mathbf{β}$ is defined in \hyperref[DD:angleB]{DD: angleB}, and $\mathbf{α}$ is defined in \hyperref[DD:angleA]{DD: angleA}.
        
\\ \midrule \\
Source & \cite{chen2005}
         
\\ \midrule \\
RefBy & \hyperref[IM:fctSfty]{IM: fctSfty}
        
\\ \bottomrule
\end{tabular}
\end{minipage}
\paragraph{}
\label{GD:normForcEqDeriv}
Normal force equilibrium is derived from the free body diagram of \hyperref[Figure:ForceDiagram]{Fig:ForceDiagram} in \hyperref[Sec:PhysSyst]{Section: Physical System Description}.

\vspace{\baselineskip}
\noindent
\begin{minipage}{\textwidth}
\begin{tabular}{>{\raggedright}p{0.13\textwidth}>{\raggedright\arraybackslash}p{0.82\textwidth}}
\toprule \textbf{Refname} & \textbf{GD:bsShrFEq}
\phantomsection 
\label{GD:bsShrFEq}
\\ \midrule \\
Label & Base shear force equilibrium
        
\\ \midrule \\
Units & $\frac{\text{N}}{\text{m}}$
        
\\ \midrule \\
Equation & \begin{displaymath}
           {\mathbf{S}}_{i}=\left({\mathbf{W}}_{i}-{\mathbf{X}}_{i-1}+{\mathbf{X}}_{i}+{\mathbf{U}_{\text{g},i}} \cos\left({\mathbf{β}}_{i}\right)+{\mathbf{Q}}_{i} \cos\left({\mathbf{ω}}_{i}\right)\right) \sin\left({\mathbf{α}}_{i}\right)-\left(-{K_{\text{c}}} {\mathbf{W}}_{i}-{\mathbf{G}}_{i}+{\mathbf{G}}_{i-1}-{\mathbf{H}}_{i}+{\mathbf{H}}_{i-1}+{\mathbf{U}_{\text{g},i}} \sin\left({\mathbf{β}}_{i}\right)+{\mathbf{Q}}_{i} \sin\left({\mathbf{ω}}_{i}\right)\right) \cos\left({\mathbf{α}}_{i}\right)
           \end{displaymath}
\\ \midrule \\
Description & \begin{symbDescription}
              \item{$\mathbf{S}$ is the mobilized shear force ($\frac{\text{N}}{\text{m}}$)}
              \item{$i$ is the index (Unitless)}
              \item{$\mathbf{W}$ is the weights ($\frac{\text{N}}{\text{m}}$)}
              \item{$\mathbf{X}$ is the interslice shear forces ($\frac{\text{N}}{\text{m}}$)}
              \item{${\mathbf{U}_{\text{g}}}$ is the surface hydrostatic forces ($\frac{\text{N}}{\text{m}}$)}
              \item{$\mathbf{β}$ is the surface angles (${{}^{\circ}}$)}
              \item{$\mathbf{Q}$ is the external forces ($\frac{\text{N}}{\text{m}}$)}
              \item{$\mathbf{ω}$ is the imposed load angles (${{}^{\circ}}$)}
              \item{$\mathbf{α}$ is the base angles (${{}^{\circ}}$)}
              \item{${K_{\text{c}}}$ is the seismic coefficient (Unitless)}
              \item{$\mathbf{G}$ is the interslice normal forces ($\frac{\text{N}}{\text{m}}$)}
              \item{$\mathbf{H}$ is the interslice normal water forces ($\frac{\text{N}}{\text{m}}$)}
              \end{symbDescription}
\\ \midrule \\
Notes & This equation satisfies \hyperref[TM:equilibrium]{TM: equilibrium} in the shear direction. $\mathbf{W}$ is defined in \hyperref[GD:sliceWght]{GD: sliceWght}, ${\mathbf{U}_{\text{g}}}$ is defined in \hyperref[GD:srfWtrF]{GD: srfWtrF}, $\mathbf{β}$ is defined in \hyperref[DD:angleB]{DD: angleB}, and $\mathbf{α}$ is defined in \hyperref[DD:angleA]{DD: angleA}.
        
\\ \midrule \\
Source & \cite{chen2005}
         
\\ \midrule \\
RefBy & \hyperref[IM:fctSfty]{IM: fctSfty}
        
\\ \bottomrule
\end{tabular}
\end{minipage}
\paragraph{}
\label{GD:bsShrFEqDeriv}
Base shear force equilibrium is derived from the free body diagram of \hyperref[Figure:ForceDiagram]{Fig:ForceDiagram} in \hyperref[Sec:PhysSyst]{Section: Physical System Description}.

\vspace{\baselineskip}
\noindent
\begin{minipage}{\textwidth}
\begin{tabular}{>{\raggedright}p{0.13\textwidth}>{\raggedright\arraybackslash}p{0.82\textwidth}}
\toprule \textbf{Refname} & \textbf{GD:resShr}
\phantomsection 
\label{GD:resShr}
\\ \midrule \\
Label & Resistive shear force
        
\\ \midrule \\
Units & $\frac{\text{N}}{\text{m}}$
        
\\ \midrule \\
Equation & \begin{displaymath}
           {\mathbf{P}}_{i}={\mathbf{N'}}_{i} \tan\left({φ'}_{i}\right)+{c'}_{i} {\mathbf{ℓ}_{b,i}}
           \end{displaymath}
\\ \midrule \\
Description & \begin{symbDescription}
              \item{$\mathbf{P}$ is the resistive shear forces ($\frac{\text{N}}{\text{m}}$)}
              \item{$i$ is the index (Unitless)}
              \item{$\mathbf{N'}$ is the effective normal forces ($\frac{\text{N}}{\text{m}}$)}
              \item{$φ'$ is the effective angle of friction (${{}^{\circ}}$)}
              \item{$c'$ is the effective cohesion (${\text{Pa}}$)}
              \item{${\mathbf{ℓ}_{b}}$ is the total base lengths of slices (${\text{m}}$)}
              \end{symbDescription}
\\ \midrule \\
Notes & ${\mathbf{ℓ}_{b}}$ is defined in \hyperref[DD:lengthLb]{DD: lengthLb}.
        
\\ \midrule \\
Source & \cite{chen2005}
         
\\ \midrule \\
RefBy & \hyperref[GD:mobShr]{GD: mobShr}
        
\\ \bottomrule
\end{tabular}
\end{minipage}
\paragraph{}
\label{GD:resShrDeriv}
Derived by substituting \hyperref[DD:normStress]{DD: normStress} and \hyperref[DD:tangStress]{DD: tangStress} into the Mohr-Coulomb shear strength, \hyperref[TM:mcShrStrgth]{TM: mcShrStrgth}, and multiplying both sides of the equation by the area of the slice in the shear-$z$ plane. Since the slope is assumed to extend infinitely in the $z$-direction (\hyperref[assumpPSC]{A: Plane-Strain-Conditions}), the resulting forces are expressed per metre in the $z$-direction. The effective angle of friction $φ'$ and the effective cohesion $c'$ are not indexed by $i$ because they are assumed to be isotropic (\hyperref[assumpSLI]{A: Soil-Layers-Isotropic}) and the soil is assumed to be homogeneous, with constant soil properties throughout (\hyperref[assumpSLH]{A: Soil-Layer-Homogeneous}, \hyperref[assumpSP]{A: Soil-Properties}).

\vspace{\baselineskip}
\noindent
\begin{minipage}{\textwidth}
\begin{tabular}{>{\raggedright}p{0.13\textwidth}>{\raggedright\arraybackslash}p{0.82\textwidth}}
\toprule \textbf{Refname} & \textbf{GD:mobShr}
\phantomsection 
\label{GD:mobShr}
\\ \midrule \\
Label & Mobilized shear force
        
\\ \midrule \\
Units & $\frac{\text{N}}{\text{m}}$
        
\\ \midrule \\
Equation & \begin{displaymath}
           {\mathbf{S}}_{i}=\frac{{\mathbf{P}}_{i}}{{F_{\text{S}}}}=\frac{{\mathbf{N'}}_{i} \tan\left({φ'}_{i}\right)+{c'}_{i} {\mathbf{ℓ}_{b,i}}}{{F_{\text{S}}}}
           \end{displaymath}
\\ \midrule \\
Description & \begin{symbDescription}
              \item{$\mathbf{S}$ is the mobilized shear force ($\frac{\text{N}}{\text{m}}$)}
              \item{$i$ is the index (Unitless)}
              \item{$\mathbf{P}$ is the resistive shear forces ($\frac{\text{N}}{\text{m}}$)}
              \item{${F_{\text{S}}}$ is the factor of safety (Unitless)}
              \item{$\mathbf{N'}$ is the effective normal forces ($\frac{\text{N}}{\text{m}}$)}
              \item{$φ'$ is the effective angle of friction (${{}^{\circ}}$)}
              \item{$c'$ is the effective cohesion (${\text{Pa}}$)}
              \item{${\mathbf{ℓ}_{b}}$ is the total base lengths of slices (${\text{m}}$)}
              \end{symbDescription}
\\ \midrule \\
Notes & ${\mathbf{ℓ}_{b}}$ is defined in \hyperref[DD:lengthLb]{DD: lengthLb}.
        
\\ \midrule \\
Source & \cite{chen2005}
         
\\ \midrule \\
RefBy & \hyperref[IM:fctSfty]{IM: fctSfty}
        
\\ \bottomrule
\end{tabular}
\end{minipage}
\paragraph{}
\label{GD:mobShrDeriv}
Mobilized shear forces is derived by dividing the definition of the $\mathbf{P}$ from \hyperref[GD:resShr]{GD: resShr}. by the definition of the factor of safety from \hyperref[TM:factOfSafety]{TM: factOfSafety}. The factor of safety ${F_{\text{S}}}$ is not indexed by $i$ because it is assumed to be constant for the entire slip surface (\hyperref[assumpFOS]{A: Factor-of-Safety}).

\vspace{\baselineskip}
\noindent
\begin{minipage}{\textwidth}
\begin{tabular}{>{\raggedright}p{0.13\textwidth}>{\raggedright\arraybackslash}p{0.82\textwidth}}
\toprule \textbf{Refname} & \textbf{GD:effNormF}
\phantomsection 
\label{GD:effNormF}
\\ \midrule \\
Label & Effective normal force
        
\\ \midrule \\
Units & $\frac{\text{N}}{\text{m}}$
        
\\ \midrule \\
Equation & \begin{displaymath}
           {\mathbf{N'}}_{i}={\mathbf{N}}_{i}-{\mathbf{U}_{\text{b},i}}
           \end{displaymath}
\\ \midrule \\
Description & \begin{symbDescription}
              \item{$\mathbf{N'}$ is the effective normal forces ($\frac{\text{N}}{\text{m}}$)}
              \item{$i$ is the index (Unitless)}
              \item{$\mathbf{N}$ is the normal forces ($\frac{\text{N}}{\text{m}}$)}
              \item{${\mathbf{U}_{\text{b}}}$ is the base hydrostatic forces ($\frac{\text{N}}{\text{m}}$)}
              \end{symbDescription}
\\ \midrule \\
Notes & ${\mathbf{U}_{\text{b}}}$ is defined in \hyperref[GD:baseWtrF]{GD: baseWtrF}.
        
\\ \midrule \\
Source & \cite{chen2005}
         
\\ \midrule \\
RefBy & 
\\ \bottomrule
\end{tabular}
\end{minipage}
\paragraph{}
\label{GD:effNormFDeriv}
Derived by substituting \hyperref[DD:normStress]{DD: normStress} into \hyperref[TM:effStress]{TM: effStress} and multiplying both sides of the equation by the the area of the slice in the shear-$z$ plane. Since the slope is assumed to extend infinitely in the $z$-direction (\hyperref[assumpPSC]{A: Plane-Strain-Conditions}), the resulting forces are expressed per metre in the $z$-direction.

\vspace{\baselineskip}
\noindent
\begin{minipage}{\textwidth}
\begin{tabular}{>{\raggedright}p{0.13\textwidth}>{\raggedright\arraybackslash}p{0.82\textwidth}}
\toprule \textbf{Refname} & \textbf{GD:resShearWO}
\phantomsection 
\label{GD:resShearWO}
\\ \midrule \\
Label & Resistive shear force, without interslice normal and shear forces
        
\\ \midrule \\
Units & $\frac{\text{N}}{\text{m}}$
        
\\ \midrule \\
Equation & \begin{displaymath}
           {\mathbf{R}}_{i}=\left(\left({\mathbf{W}}_{i}+{\mathbf{U}_{\text{g},i}} \cos\left({\mathbf{β}}_{i}\right)\right) \cos\left({\mathbf{α}}_{i}\right)+\left(-{\mathbf{H}}_{i}+{\mathbf{H}}_{i-1}+{\mathbf{U}_{\text{g},i}} \sin\left({\mathbf{β}}_{i}\right)\right) \sin\left({\mathbf{α}}_{i}\right)-{\mathbf{U}_{\text{b},i}}\right) \tan\left({φ'}_{i}\right)+{c'}_{i} {\mathbf{ℓ}_{b,i}}
           \end{displaymath}
\\ \midrule \\
Description & \begin{symbDescription}
              \item{$\mathbf{R}$ is the resistive shear forces without the influence of interslice forces ($\frac{\text{N}}{\text{m}}$)}
              \item{$i$ is the index (Unitless)}
              \item{$\mathbf{W}$ is the weights ($\frac{\text{N}}{\text{m}}$)}
              \item{${\mathbf{U}_{\text{g}}}$ is the surface hydrostatic forces ($\frac{\text{N}}{\text{m}}$)}
              \item{$\mathbf{β}$ is the surface angles (${{}^{\circ}}$)}
              \item{$\mathbf{α}$ is the base angles (${{}^{\circ}}$)}
              \item{$\mathbf{H}$ is the interslice normal water forces ($\frac{\text{N}}{\text{m}}$)}
              \item{${\mathbf{U}_{\text{b}}}$ is the base hydrostatic forces ($\frac{\text{N}}{\text{m}}$)}
              \item{$φ'$ is the effective angle of friction (${{}^{\circ}}$)}
              \item{$c'$ is the effective cohesion (${\text{Pa}}$)}
              \item{${\mathbf{ℓ}_{b}}$ is the total base lengths of slices (${\text{m}}$)}
              \end{symbDescription}
\\ \midrule \\
Notes & $\mathbf{W}$ is defined in \hyperref[GD:sliceWght]{GD: sliceWght}, ${\mathbf{U}_{\text{g}}}$ is defined in \hyperref[GD:srfWtrF]{GD: srfWtrF}, $\mathbf{β}$ is defined in \hyperref[DD:angleB]{DD: angleB}, $\mathbf{α}$ is defined in \hyperref[DD:angleA]{DD: angleA}, $\mathbf{H}$ is defined in \hyperref[DD:intersliceWtrF]{DD: intersliceWtrF}, ${\mathbf{U}_{\text{b}}}$ is defined in \hyperref[GD:baseWtrF]{GD: baseWtrF}, and ${\mathbf{ℓ}_{b}}$ is defined in \hyperref[DD:lengthLb]{DD: lengthLb}.
        
\\ \midrule \\
Source & \cite{chen2005} and \cite{karchewski2012}
         
\\ \midrule \\
RefBy & \hyperref[IM:intsliceFs]{IM: intsliceFs} and \hyperref[IM:fctSfty]{IM: fctSfty}
        
\\ \bottomrule
\end{tabular}
\end{minipage}

\vspace{\baselineskip}
\noindent
\begin{minipage}{\textwidth}
\begin{tabular}{>{\raggedright}p{0.13\textwidth}>{\raggedright\arraybackslash}p{0.82\textwidth}}
\toprule \textbf{Refname} & \textbf{GD:mobShearWO}
\phantomsection 
\label{GD:mobShearWO}
\\ \midrule \\
Label & Mobilized shear force, without interslice normal and shear forces
        
\\ \midrule \\
Units & $\frac{\text{N}}{\text{m}}$
        
\\ \midrule \\
Equation & \begin{displaymath}
           {\mathbf{T}}_{i}=\left({\mathbf{W}}_{i}+{\mathbf{U}_{\text{g},i}} \cos\left({\mathbf{β}}_{i}\right)\right) \sin\left({\mathbf{α}}_{i}\right)-\left(-{\mathbf{H}}_{i}+{\mathbf{H}}_{i-1}+{\mathbf{U}_{\text{g},i}} \sin\left({\mathbf{β}}_{i}\right)\right) \cos\left({\mathbf{α}}_{i}\right)
           \end{displaymath}
\\ \midrule \\
Description & \begin{symbDescription}
              \item{$\mathbf{T}$ is the mobilized shear forces without the influence of interslice forces ($\frac{\text{N}}{\text{m}}$)}
              \item{$i$ is the index (Unitless)}
              \item{$\mathbf{W}$ is the weights ($\frac{\text{N}}{\text{m}}$)}
              \item{${\mathbf{U}_{\text{g}}}$ is the surface hydrostatic forces ($\frac{\text{N}}{\text{m}}$)}
              \item{$\mathbf{β}$ is the surface angles (${{}^{\circ}}$)}
              \item{$\mathbf{α}$ is the base angles (${{}^{\circ}}$)}
              \item{$\mathbf{H}$ is the interslice normal water forces ($\frac{\text{N}}{\text{m}}$)}
              \end{symbDescription}
\\ \midrule \\
Notes & $\mathbf{W}$ is defined in \hyperref[GD:sliceWght]{GD: sliceWght}, ${\mathbf{U}_{\text{g}}}$ is defined in \hyperref[GD:srfWtrF]{GD: srfWtrF}, $\mathbf{β}$ is defined in \hyperref[DD:angleB]{DD: angleB}, $\mathbf{α}$ is defined in \hyperref[DD:angleA]{DD: angleA}, and $\mathbf{H}$ is defined in \hyperref[DD:intersliceWtrF]{DD: intersliceWtrF}.
        
\\ \midrule \\
Source & \cite{chen2005} and \cite{karchewski2012}
         
\\ \midrule \\
RefBy & \hyperref[IM:intsliceFs]{IM: intsliceFs} and \hyperref[IM:fctSfty]{IM: fctSfty}
        
\\ \bottomrule
\end{tabular}
\end{minipage}

\vspace{\baselineskip}
\noindent
\begin{minipage}{\textwidth}
\begin{tabular}{>{\raggedright}p{0.13\textwidth}>{\raggedright\arraybackslash}p{0.82\textwidth}}
\toprule \textbf{Refname} & \textbf{GD:normShrR}
\phantomsection 
\label{GD:normShrR}
\\ \midrule \\
Label & Interslice normal and shear force proportionality
        
\\ \midrule \\
Units & $\frac{\text{N}}{\text{m}}$
        
\\ \midrule \\
Equation & \begin{displaymath}
           \mathbf{X}=λ \mathbf{f} \mathbf{G}
           \end{displaymath}
\\ \midrule \\
Description & \begin{symbDescription}
              \item{$\mathbf{X}$ is the interslice shear forces ($\frac{\text{N}}{\text{m}}$)}
              \item{$λ$ is the proportionality constant (Unitless)}
              \item{$\mathbf{f}$ is the interslice normal to shear force ratio variation function (Unitless)}
              \item{$\mathbf{G}$ is the interslice normal forces ($\frac{\text{N}}{\text{m}}$)}
              \end{symbDescription}
\\ \midrule \\
Notes & Mathematical representation of the primary assumption for the Morgenstern-Price method (\hyperref[assumpINSFL]{A: Interslice-Norm-Shear-Forces-Linear}). $\mathbf{f}$ is defined in \hyperref[DD:ratioVariation]{DD: ratioVariation}.
        
\\ \midrule \\
Source & \cite{chen2005}
         
\\ \midrule \\
RefBy & \hyperref[IM:nrmShrFor]{IM: nrmShrFor} and \hyperref[IM:fctSfty]{IM: fctSfty}
        
\\ \bottomrule
\end{tabular}
\end{minipage}

\vspace{\baselineskip}
\noindent
\begin{minipage}{\textwidth}
\begin{tabular}{>{\raggedright}p{0.13\textwidth}>{\raggedright\arraybackslash}p{0.82\textwidth}}
\toprule \textbf{Refname} & \textbf{GD:momentEql}
\phantomsection 
\label{GD:momentEql}
\\ \midrule \\
Label & Moment equilibrium
        
\\ \midrule \\
Units & ${\text{N}}$
        
\\ \midrule \\
Equation & \begin{displaymath}
           0=-{\mathbf{G}}_{i} \left({\mathbf{h}_{\text{z},i}}+\frac{{\mathbf{b}}_{i}}{2} \tan\left({\mathbf{α}}_{i}\right)\right)+{\mathbf{G}}_{i-1} \left({\mathbf{h}_{\text{z},i-1}}-\frac{{\mathbf{b}}_{i}}{2} \tan\left({\mathbf{α}}_{i}\right)\right)-{\mathbf{H}}_{i} \left(\frac{1}{3} {\mathbf{h}_{\text{z,w},i}}+\frac{{\mathbf{b}}_{i}}{2} \tan\left({\mathbf{α}}_{i}\right)\right)+{\mathbf{H}}_{i-1} \left(\frac{1}{3} {\mathbf{h}_{\text{z,w},i-1}}-\frac{{\mathbf{b}}_{i}}{2} \tan\left({\mathbf{α}}_{i}\right)\right)+\frac{{\mathbf{b}}_{i}}{2} \left({\mathbf{X}}_{i}+{\mathbf{X}}_{i-1}\right)+\frac{-{K_{\text{c}}} {\mathbf{W}}_{i} {\mathbf{h}}_{i}}{2}+{\mathbf{U}_{\text{g},i}} \sin\left({\mathbf{β}}_{i}\right) {\mathbf{h}}_{i}+{\mathbf{Q}}_{i} \sin\left({\mathbf{ω}}_{i}\right) {\mathbf{h}}_{i}
           \end{displaymath}
\\ \midrule \\
Description & \begin{symbDescription}
              \item{$\mathbf{G}$ is the interslice normal forces ($\frac{\text{N}}{\text{m}}$)}
              \item{$i$ is the index (Unitless)}
              \item{${\mathbf{h}_{\text{z}}}$ is the heights of interslice normal forces (${\text{m}}$)}
              \item{$\mathbf{b}$ is the base width of slices (${\text{m}}$)}
              \item{$\mathbf{α}$ is the base angles (${{}^{\circ}}$)}
              \item{$\mathbf{H}$ is the interslice normal water forces ($\frac{\text{N}}{\text{m}}$)}
              \item{${\mathbf{h}_{\text{z,w}}}$ is the heights of the water table (${\text{m}}$)}
              \item{$\mathbf{X}$ is the interslice shear forces ($\frac{\text{N}}{\text{m}}$)}
              \item{${K_{\text{c}}}$ is the seismic coefficient (Unitless)}
              \item{$\mathbf{W}$ is the weights ($\frac{\text{N}}{\text{m}}$)}
              \item{$\mathbf{h}$ is the $y$-direction heights of slices (${\text{m}}$)}
              \item{${\mathbf{U}_{\text{g}}}$ is the surface hydrostatic forces ($\frac{\text{N}}{\text{m}}$)}
              \item{$\mathbf{β}$ is the surface angles (${{}^{\circ}}$)}
              \item{$\mathbf{Q}$ is the external forces ($\frac{\text{N}}{\text{m}}$)}
              \item{$\mathbf{ω}$ is the imposed load angles (${{}^{\circ}}$)}
              \end{symbDescription}
\\ \midrule \\
Notes & This equation satisfies \hyperref[TM:equilibrium]{TM: equilibrium} for the net moment. $\mathbf{b}$ is defined in \hyperref[DD:lengthB]{DD: lengthB}, $\mathbf{α}$ is defined in \hyperref[DD:angleA]{DD: angleA}, $\mathbf{W}$ is defined in \hyperref[GD:sliceWght]{GD: sliceWght}, $\mathbf{h}$ is defined in \hyperref[DD:slcHeight]{DD: slcHeight}, ${\mathbf{U}_{\text{g}}}$ is defined in \hyperref[GD:srfWtrF]{GD: srfWtrF}, and $\mathbf{β}$ is defined in \hyperref[DD:angleB]{DD: angleB}.
        
\\ \midrule \\
Source & \cite{chen2005}
         
\\ \midrule \\
RefBy & \hyperref[IM:nrmShrFor]{IM: nrmShrFor}
        
\\ \bottomrule
\end{tabular}
\end{minipage}
\paragraph{}
\label{GD:momentEqlDeriv}
Moment is equal to torque, so the equation from \hyperref[DD:torque]{DD: torque} will be used to calculate moments:

\begin{displaymath}
\mathbf{τ}=\mathbf{u}\times\mathbf{F}
\end{displaymath}
Considering one dimension, with moments in the clockwise direction as positive and moments in the counterclockwise direction as negative, and replacing the torque symbol with the moment symbol, the equation simplifies to:

\begin{displaymath}
M={F_{\text{rot}}} r
\end{displaymath}
where ${F_{\text{rot}}}$ is the force causing rotation and $r$ is the length of the moment arm, or the distance between the force and the axis about which the rotation acts. To represent the moment equilibrium, the moments from each force acting on a slice must be considered and added together. The forces acting on a slice are all shown in \hyperref[Figure:ForceDiagram]{Fig:ForceDiagram}. The midpoint of the base of a slice is considered as the axis of rotation, from which the length of the moment arm is measured. Considering first the interslice normal force acting on slice interface $i$, the moment is negative because the force tends to rotate the slice in a counterclockwise direction, and the length of the moment arm is the height of the force plus the difference in height between the base at slice interface $i$ and the base at the midpoint of slice $i$. Thus, the moment is expressed as:

\begin{displaymath}
-{\mathbf{G}}_{i} \left({\mathbf{h}_{\text{z},i}}+\frac{{\mathbf{b}}_{i}}{2} \tan\left({\mathbf{α}}_{i}\right)\right)
\end{displaymath}
For the $i-1$th slice interface, the moment is similar but in the opposite direction:

\begin{displaymath}
{\mathbf{G}}_{i-1} \left({\mathbf{h}_{\text{z},i-1}}-\frac{{\mathbf{b}}_{i}}{2} \tan\left({\mathbf{α}}_{i}\right)\right)
\end{displaymath}
Next, the interslice normal water force is considered. This force is zero at the height of the water table, then increases linearly towards the base of the slice due to the increasing water pressure. For such a triangular distribution, the resultant force acts at one-third of the height. Thus, for the interslice normal water force acting on slice interface $i$, the moment is:

\begin{displaymath}
-{\mathbf{H}}_{i} \left(\frac{1}{3} {\mathbf{h}_{\text{z,w},i}}+\frac{{\mathbf{b}}_{i}}{2} \tan\left({\mathbf{α}}_{i}\right)\right)
\end{displaymath}
The moment for the interslice normal water force acting on slice interface $i-1$ is:

\begin{displaymath}
{\mathbf{H}}_{i-1} \left(\frac{1}{3} {\mathbf{h}_{\text{z,w},i-1}}+\frac{{\mathbf{b}}_{i}}{2} \tan\left({\mathbf{α}}_{i}\right)\right)
\end{displaymath}
The interslice shear force at slice interface $i$ tends to rotate in the clockwise direction, and the length of the moment arm is the length from the slice edge to the slice midpoint, equivalent to half of the width of the slice, so the moment is:

\begin{displaymath}
{\mathbf{X}}_{i} \frac{{\mathbf{b}}_{i}}{2}
\end{displaymath}
The interslice shear force at slice interface $i-1$ also tends to rotate in the clockwise direction, and has the same length of the moment arm, so the moment is:

\begin{displaymath}
{\mathbf{X}}_{i-1} \frac{{\mathbf{b}}_{i}}{2}
\end{displaymath}
Seismic forces act over the entire height of the slice. For each horizontal segment of the slice, the seismic force is ${K_{\text{c}}} {\mathbf{W}}_{i}$ where ${\mathbf{W}}_{i}$ can be expressed as $γ {\mathbf{b}}_{i} y$ using \hyperref[GD:weight]{GD: weight} where $y$ is the height of the segment under consideration. The corresponding length of the moment arm is $y$, the height from the base of the slice to the segment under consideration. In reality, the forces near the surface of the soil mass are slightly different due to the slope of the surface, but this difference is assumed to be negligible (\hyperref[assumpNESSS]{A: Negligible-Effect-Surface-Slope-Seismic}). The resultant moment from the forces on all of the segments with an equivalent resultant length of the moment arm is determined by taking the integral over the slice height. The forces tend to rotate in the counterclockwise direction, so the moment is negative:

\begin{displaymath}
-\int_{0}^{{\mathbf{h}}_{i}}{{K_{\text{c}}} γ {\mathbf{b}}_{i} y}\,dy
\end{displaymath}
Solving the definite integral yields:

\begin{displaymath}
-{K_{\text{c}}} γ {\mathbf{b}}_{i} \frac{{\mathbf{h}}_{i}^{2}}{2}
\end{displaymath}
Using \hyperref[GD:weight]{GD: weight} again to express $γ {\mathbf{b}}_{i} {\mathbf{h}}_{i}$ as ${\mathbf{W}}_{i}$, the moment is:

\begin{displaymath}
-{K_{\text{c}}} {\mathbf{W}}_{i} \frac{{\mathbf{h}}_{i}}{2}
\end{displaymath}
The surface hydrostatic force acts into the midpoint of the surface of the slice (\hyperref[assumpHFSM]{A: Hydrostatic-Force-Slice-Midpoint}). Thus, the vertical component of the force acts directly towards the point of rotation, and has a moment of zero. The horizontal component of the force tends to rotate in a clockwise direction and the length of the moment arm is the entire height of the slice. Thus, the moment is:

\begin{displaymath}
{\mathbf{U}_{\text{g},i}} \sin\left({\mathbf{β}}_{i}\right) {\mathbf{h}}_{i}
\end{displaymath}
The external force again acts into the midpoint of the slice surface, so the vertical component does not contribute to the moment, and the length of the moment arm is again the entire height of the slice. The moment is:

\begin{displaymath}
{\mathbf{Q}}_{i} \sin\left({\mathbf{ω}}_{i}\right) {\mathbf{h}}_{i}
\end{displaymath}
The base hydrostatic force and slice weight both act in the direction of the point of rotation (\hyperref[assumpHFSM]{A: Hydrostatic-Force-Slice-Midpoint}), therefore both have moments of zero. Thus, all of the moments have been determined. The moment equilibrium is then represented by the sum of all moments:

\begin{displaymath}
0=-{\mathbf{G}}_{i} \left({\mathbf{h}_{\text{z},i}}+\frac{{\mathbf{b}}_{i}}{2} \tan\left({\mathbf{α}}_{i}\right)\right)+{\mathbf{G}}_{i-1} \left({\mathbf{h}_{\text{z},i-1}}-\frac{{\mathbf{b}}_{i}}{2} \tan\left({\mathbf{α}}_{i}\right)\right)-{\mathbf{H}}_{i} \left(\frac{1}{3} {\mathbf{h}_{\text{z,w},i}}+\frac{{\mathbf{b}}_{i}}{2} \tan\left({\mathbf{α}}_{i}\right)\right)+{\mathbf{H}}_{i-1} \left(\frac{1}{3} {\mathbf{h}_{\text{z,w},i-1}}-\frac{{\mathbf{b}}_{i}}{2} \tan\left({\mathbf{α}}_{i}\right)\right)+\frac{{\mathbf{b}}_{i}}{2} \left({\mathbf{X}}_{i}+{\mathbf{X}}_{i-1}\right)+\frac{-{K_{\text{c}}} {\mathbf{W}}_{i} {\mathbf{h}}_{i}}{2}+{\mathbf{U}_{\text{g},i}} \sin\left({\mathbf{β}}_{i}\right) {\mathbf{h}}_{i}+{\mathbf{Q}}_{i} \sin\left({\mathbf{ω}}_{i}\right) {\mathbf{h}}_{i}
\end{displaymath}
\vspace{\baselineskip}
\noindent
\begin{minipage}{\textwidth}
\begin{tabular}{>{\raggedright}p{0.13\textwidth}>{\raggedright\arraybackslash}p{0.82\textwidth}}
\toprule \textbf{Refname} & \textbf{GD:weight}
\phantomsection 
\label{GD:weight}
\\ \midrule \\
Label & Weight
        
\\ \midrule \\
Units & ${\text{N}}$
        
\\ \midrule \\
Equation & \begin{displaymath}
           W=V γ
           \end{displaymath}
\\ \midrule \\
Description & \begin{symbDescription}
              \item{$W$ is the weight (${\text{N}}$)}
              \item{$V$ is the volume (${\text{m}^{3}}$)}
              \item{$γ$ is the specific weight ($\frac{\text{N}}{\text{m}^{3}}$)}
              \end{symbDescription}
\\ \midrule \\
Source & \hyperref{https://en.wikipedia.org/wiki/Weight}{}{}{Definition of Weight}
         
\\ \midrule \\
RefBy & \hyperref[GD:sliceWght]{GD: sliceWght} and \hyperref[GD:momentEql]{GD: momentEql}
        
\\ \bottomrule
\end{tabular}
\end{minipage}
\paragraph{Detailed derivation of weight:}
\label{GD:weightDeriv}
Under the influence of gravity, and assuming a 2D Cartesian coordinate system with down as positive, an object has an acceleration vector of:

\begin{displaymath}
\mathbf{a}=\begin{bmatrix}
           0\\
           \mathbf{g} \mathbf{\hat{j}}
           \end{bmatrix}
\end{displaymath}
Since there is only one non-zero vector component, the scalar value $W$ will be used for the weight. In this scenario, Newton's second law of motion from \hyperref[TM:NewtonSecLawMot]{TM: NewtonSecLawMot} can be expressed as:

\begin{displaymath}
W=m \mathbf{g}
\end{displaymath}
Mass can be expressed as density multiplied by volume, resulting in:

\begin{displaymath}
W=ρ V \mathbf{g}
\end{displaymath}
Substituting specific weight as the product of density and gravitational acceleration yields:

\begin{displaymath}
W=V γ
\end{displaymath}
\vspace{\baselineskip}
\noindent
\begin{minipage}{\textwidth}
\begin{tabular}{>{\raggedright}p{0.13\textwidth}>{\raggedright\arraybackslash}p{0.82\textwidth}}
\toprule \textbf{Refname} & \textbf{GD:sliceWght}
\phantomsection 
\label{GD:sliceWght}
\\ \midrule \\
Label & Slice weight
        
\\ \midrule \\
Units & $\frac{\text{N}}{\text{m}}$
        
\\ \midrule \\
Equation & \begin{displaymath}
           {\mathbf{W}}_{i}={\mathbf{b}}_{i} \frac{1}{2} \begin{cases}
                                                         \left({\mathbf{y}_{\text{slope},i}}-{\mathbf{y}_{\text{slip},i}}+{\mathbf{y}_{\text{slope},i-1}}-{\mathbf{y}_{\text{slip},i-1}}\right) {γ_{\text{sat}}}, & {\mathbf{y}_{\text{wt},i}}\gt{}{\mathbf{y}_{\text{slope},i}}\lor{}{\mathbf{y}_{\text{wt},i-1}}\gt{}{\mathbf{y}_{\text{slope},i-1}}\\
                                                         \left({\mathbf{y}_{\text{slope},i}}-{\mathbf{y}_{\text{wt},i}}+{\mathbf{y}_{\text{slope},i-1}}-{\mathbf{y}_{\text{wt},i-1}}\right) {γ_{\text{dry}}}+\left({\mathbf{y}_{\text{wt},i}}-{\mathbf{y}_{\text{slip},i}}+{\mathbf{y}_{\text{wt},i-1}}-{\mathbf{y}_{\text{slip},i-1}}\right) {γ_{\text{sat}}}, & {\mathbf{y}_{\text{slope},i}}\geq{}{\mathbf{y}_{\text{wt},i}}\geq{}{\mathbf{y}_{\text{slip},i}}\land{}{\mathbf{y}_{\text{slope},i-1}}\geq{}{\mathbf{y}_{\text{wt},i-1}}\geq{}{\mathbf{y}_{\text{slip},i-1}}\\
                                                         \left({\mathbf{y}_{\text{slope},i}}-{\mathbf{y}_{\text{slip},i}}+{\mathbf{y}_{\text{slope},i-1}}-{\mathbf{y}_{\text{slip},i-1}}\right) {γ_{\text{dry}}}, & {\mathbf{y}_{\text{wt},i}}\lt{}{\mathbf{y}_{\text{slip},i}}\lor{}{\mathbf{y}_{\text{wt},i-1}}\lt{}{\mathbf{y}_{\text{slip},i-1}}
                                                         \end{cases}
           \end{displaymath}
\\ \midrule \\
Description & \begin{symbDescription}
              \item{$\mathbf{W}$ is the weights ($\frac{\text{N}}{\text{m}}$)}
              \item{$i$ is the index (Unitless)}
              \item{$\mathbf{b}$ is the base width of slices (${\text{m}}$)}
              \item{${\mathbf{y}_{\text{slope}}}$ is the $y$-coordinates of the slope (${\text{m}}$)}
              \item{${\mathbf{y}_{\text{slip}}}$ is the $y$-coordinates of the slip surface (${\text{m}}$)}
              \item{${γ_{\text{sat}}}$ is the soil saturated unit weight ($\frac{\text{N}}{\text{m}^{3}}$)}
              \item{${\mathbf{y}_{\text{wt}}}$ is the $y$-coordinates of the water table (${\text{m}}$)}
              \item{${γ_{\text{dry}}}$ is the soil dry unit weight ($\frac{\text{N}}{\text{m}^{3}}$)}
              \end{symbDescription}
\\ \midrule \\
Notes & This equation is based on the assumption that the surface and the base of a slice are straight lines (\hyperref[assumpSBSBISL]{A: Surface-Base-Slice-between-Interslice-Straight-Lines}). The soil dry unit weight ${γ_{\text{dry}}}$ and the soil saturated unit weight ${γ_{\text{sat}}}$ are not indexed by $i$ because the soil is assumed to be homogeneous, with constant soil properties throughout (\hyperref[assumpSLH]{A: Soil-Layer-Homogeneous}). $\mathbf{b}$ is defined in \hyperref[DD:lengthB]{DD: lengthB}.
        
\\ \midrule \\
Source & \cite{fredlund1977}
         
\\ \midrule \\
RefBy & \hyperref[GD:resShearWO]{GD: resShearWO}, \hyperref[GD:normForcEq]{GD: normForcEq}, \hyperref[GD:momentEql]{GD: momentEql}, \hyperref[GD:mobShearWO]{GD: mobShearWO}, and \hyperref[GD:bsShrFEq]{GD: bsShrFEq}
        
\\ \bottomrule
\end{tabular}
\end{minipage}
\paragraph{}
\label{GD:sliceWghtDeriv}
For the case where the water table is above the slope surface, the weights come from the weight of the saturated soil. Substituting values for saturated soil into the equation for weight from \hyperref[GD:weight]{GD: weight} yields:

\begin{displaymath}
{\mathbf{W}}_{i}={\mathbf{V}_{\text{sat},i}} {γ_{\text{sat}}}
\end{displaymath}
Due to \hyperref[assumpPSC]{A: Plane-Strain-Conditions}, only two dimensions are considered, so the areas of saturated soil are considered instead of the volumes of saturated soil. Any given slice has a trapezoidal shape. The area of a trapezoid is the average of the lengths of the parallel sides multiplied by the length between the parallel sides. The parallel sides in this case are the interslice edges and the length between them is the width of the slice. Thus, the weights are defined as:

\begin{displaymath}
{\mathbf{W}}_{i}={\mathbf{b}}_{i} \frac{1}{2} \left({\mathbf{y}_{\text{slope},i}}-{\mathbf{y}_{\text{slip},i}}+{\mathbf{y}_{\text{slope},i-1}}-{\mathbf{y}_{\text{slip},i-1}}\right) {γ_{\text{sat}}}
\end{displaymath}
For the case where the water table is below the slip surface, the weights come from the weight of the dry soil. Substituting values for dry soil into the equation for weight from \hyperref[GD:weight]{GD: weight} yields:

\begin{displaymath}
{\mathbf{W}}_{i}={\mathbf{V}_{\text{dry},i}} {γ_{\text{dry}}}
\end{displaymath}
\hyperref[assumpPSC]{A: Plane-Strain-Conditions} again allows for two-dimensional analysis so the areas of dry soil are considered instead of the volumes of dry soil. The trapezoidal slice shape is the same as in the previous case, so the weights are defined as:

\begin{displaymath}
{\mathbf{W}}_{i}={\mathbf{b}}_{i} \frac{1}{2} \left({\mathbf{y}_{\text{slope},i}}-{\mathbf{y}_{\text{slip},i}}+{\mathbf{y}_{\text{slope},i-1}}-{\mathbf{y}_{\text{slip},i-1}}\right) {γ_{\text{dry}}}
\end{displaymath}
For the case where the water table is between the slope surface and slip surface, the weights are the sums of the the weights of the dry portions and weights of the saturated portions of the soil. Substituting values for dry and saturated soil into the equation for weight from \hyperref[GD:weight]{GD: weight} and adding them together yields:

\begin{displaymath}
{\mathbf{W}}_{i}={\mathbf{V}_{\text{dry},i}} {γ_{\text{dry}}}+{\mathbf{V}_{\text{sat},i}} {γ_{\text{sat}}}
\end{displaymath}
\hyperref[assumpPSC]{A: Plane-Strain-Conditions} again allows for two-dimensional analysis so the areas of dry soil and areas of saturated soil are considered instead of the volumes of dry soil and volumes of saturated soil. The water table is assumed to only intersect a slice surface or base at a slice edge (\hyperref[assumpWISE]{A: Water-Intersects-Surface-Edge}, \hyperref[assumpWIBE]{A: Water-Intersects-Base-Edge}), so the dry and saturated portions each have trapezoidal shape. For the dry portion, the parallel sides of the trapezoid are the lengths between the slope surface and water table at the slice edges. For the saturated portion, the parallel sides of the trapezoid are the lengths between the water table and slip surface at the slice edges. Thus, the weights are defined as:

\begin{displaymath}
{\mathbf{W}}_{i}={\mathbf{b}}_{i} \frac{1}{2} \left(\left({\mathbf{y}_{\text{slope},i}}-{\mathbf{y}_{\text{wt},i}}+{\mathbf{y}_{\text{slope},i-1}}-{\mathbf{y}_{\text{wt},i-1}}\right) {γ_{\text{dry}}}+\left({\mathbf{y}_{\text{wt},i}}-{\mathbf{y}_{\text{slip},i}}+{\mathbf{y}_{\text{wt},i-1}}-{\mathbf{y}_{\text{slip},i-1}}\right) {γ_{\text{sat}}}\right)
\end{displaymath}
\vspace{\baselineskip}
\noindent
\begin{minipage}{\textwidth}
\begin{tabular}{>{\raggedright}p{0.13\textwidth}>{\raggedright\arraybackslash}p{0.82\textwidth}}
\toprule \textbf{Refname} & \textbf{GD:hsPressure}
\phantomsection 
\label{GD:hsPressure}
\\ \midrule \\
Label & Hydrostatic pressure
        
\\ \midrule \\
Units & ${\text{Pa}}$
        
\\ \midrule \\
Equation & \begin{displaymath}
           p=γ h
           \end{displaymath}
\\ \midrule \\
Description & \begin{symbDescription}
              \item{$p$ is the pressure (${\text{Pa}}$)}
              \item{$γ$ is the specific weight ($\frac{\text{N}}{\text{m}^{3}}$)}
              \item{$h$ is the height (${\text{m}}$)}
              \end{symbDescription}
\\ \midrule \\
Notes & This equation is derived from Bernoulli's equation for a slow moving fluid through a porous material.
        
\\ \midrule \\
Source & \hyperref{https://en.wikipedia.org/wiki/Pressure}{}{}{Definition of Pressure}
         
\\ \midrule \\
RefBy & \hyperref[GD:srfWtrF]{GD: srfWtrF} and \hyperref[GD:baseWtrF]{GD: baseWtrF}
        
\\ \bottomrule
\end{tabular}
\end{minipage}

\vspace{\baselineskip}
\noindent
\begin{minipage}{\textwidth}
\begin{tabular}{>{\raggedright}p{0.13\textwidth}>{\raggedright\arraybackslash}p{0.82\textwidth}}
\toprule \textbf{Refname} & \textbf{GD:baseWtrF}
\phantomsection 
\label{GD:baseWtrF}
\\ \midrule \\
Label & Base hydrostatic force
        
\\ \midrule \\
Units & $\frac{\text{N}}{\text{m}}$
        
\\ \midrule \\
Equation & \begin{displaymath}
           {\mathbf{U}_{\text{b},i}}={\mathbf{ℓ}_{b,i}} {γ_{w}} \frac{1}{2} \begin{cases}
                                                                            {\mathbf{y}_{\text{wt},i}}-{\mathbf{y}_{\text{slip},i}}+{\mathbf{y}_{\text{wt},i-1}}-{\mathbf{y}_{\text{slip},i-1}}, & {\mathbf{y}_{\text{wt},i}}\gt{}{\mathbf{y}_{\text{slip},i}}\lor{}{\mathbf{y}_{\text{wt},i-1}}\gt{}{\mathbf{y}_{\text{slip},i-1}}\\
                                                                            0, & {\mathbf{y}_{\text{wt},i}}\leq{}{\mathbf{y}_{\text{slip},i}}\land{}{\mathbf{y}_{\text{wt},i-1}}\leq{}{\mathbf{y}_{\text{slip},i-1}}
                                                                            \end{cases}
           \end{displaymath}
\\ \midrule \\
Description & \begin{symbDescription}
              \item{${\mathbf{U}_{\text{b}}}$ is the base hydrostatic forces ($\frac{\text{N}}{\text{m}}$)}
              \item{$i$ is the index (Unitless)}
              \item{${\mathbf{ℓ}_{b}}$ is the total base lengths of slices (${\text{m}}$)}
              \item{${γ_{w}}$ is the unit weight of water ($\frac{\text{N}}{\text{m}^{3}}$)}
              \item{${\mathbf{y}_{\text{wt}}}$ is the $y$-coordinates of the water table (${\text{m}}$)}
              \item{${\mathbf{y}_{\text{slip}}}$ is the $y$-coordinates of the slip surface (${\text{m}}$)}
              \end{symbDescription}
\\ \midrule \\
Notes & This equation is based on the assumption that the base of a slice is a straight line (\hyperref[assumpSBSBISL]{A: Surface-Base-Slice-between-Interslice-Straight-Lines}). ${\mathbf{ℓ}_{b}}$ is defined in \hyperref[DD:lengthLb]{DD: lengthLb}.
        
\\ \midrule \\
Source & \cite{fredlund1977}
         
\\ \midrule \\
RefBy & \hyperref[GD:resShearWO]{GD: resShearWO}, \hyperref[GD:effNormF]{GD: effNormF}, and \hyperref[GD:baseWtrF]{GD: baseWtrF}
        
\\ \bottomrule
\end{tabular}
\end{minipage}
\paragraph{}
\label{GD:baseWtrFDeriv}
The base hydrostatic forces come from the hydrostatic pressure exerted by the water above the base of each slice. The equation for hydrostatic pressure from \hyperref[GD:hsPressure]{GD: hsPressure} is:

\begin{displaymath}
p=γ h
\end{displaymath}
The specific weight in this case is the unit weight of water ${γ_{w}}$. The height in this case is the height from the slice base to the water table. This height is measured from the midpoint of the slice because the resultant hydrostatic force is assumed to act at the slice midpoint (\hyperref[assumpHFSM]{A: Hydrostatic-Force-Slice-Midpoint}). The height at the midpoint is the average of the height at slice interface $i$ and the height at slice interface $i-1$:

\begin{displaymath}
\frac{1}{2} \left({\mathbf{y}_{\text{wt},i}}-{\mathbf{y}_{\text{slip},i}}+{\mathbf{y}_{\text{wt},i-1}}-{\mathbf{y}_{\text{slip},i-1}}\right)
\end{displaymath}
Due to \hyperref[assumpPSC]{A: Plane-Strain-Conditions}, only two dimensions are considered, so the base hydrostatic forces are expressed as forces per meter. The pressures acting on the slices can thus be converted to base hydrostatic forces by multiplying by the corresponding length of the slice base ${\mathbf{ℓ}_{b,i}}$, assuming the water table does not intersect a slice base except at a slice edge (\hyperref[assumpWIBE]{A: Water-Intersects-Base-Edge}). Thus, in the case where the height of the water table is above the height of the slip surface, the base hydrostatic forces are defined as:

\begin{displaymath}
{\mathbf{U}_{\text{b},i}}={\mathbf{ℓ}_{b,i}} {γ_{w}} \frac{1}{2} \left({\mathbf{y}_{\text{wt},i}}-{\mathbf{y}_{\text{slip},i}}+{\mathbf{y}_{\text{wt},i-1}}-{\mathbf{y}_{\text{slip},i-1}}\right)
\end{displaymath}
This equation is the non-zero case of \hyperref[GD:baseWtrF]{GD: baseWtrF}. The zero case is when the height of the water table is below the height of the slip surface, so there is no hydrostatic force.

\vspace{\baselineskip}
\noindent
\begin{minipage}{\textwidth}
\begin{tabular}{>{\raggedright}p{0.13\textwidth}>{\raggedright\arraybackslash}p{0.82\textwidth}}
\toprule \textbf{Refname} & \textbf{GD:srfWtrF}
\phantomsection 
\label{GD:srfWtrF}
\\ \midrule \\
Label & Surface hydrostatic force
        
\\ \midrule \\
Units & $\frac{\text{N}}{\text{m}}$
        
\\ \midrule \\
Equation & \begin{displaymath}
           {\mathbf{U}_{\text{g},i}}={\mathbf{ℓ}_{s,i}} {γ_{w}} \frac{1}{2} \begin{cases}
                                                                            {\mathbf{y}_{\text{wt},i}}-{\mathbf{y}_{\text{slope},i}}+{\mathbf{y}_{\text{wt},i-1}}-{\mathbf{y}_{\text{slope},i-1}}, & {\mathbf{y}_{\text{wt},i}}\gt{}{\mathbf{y}_{\text{slope},i}}\lor{}{\mathbf{y}_{\text{wt},i-1}}\gt{}{\mathbf{y}_{\text{slope},i-1}}\\
                                                                            0, & {\mathbf{y}_{\text{wt},i}}\leq{}{\mathbf{y}_{\text{slope},i}}\land{}{\mathbf{y}_{\text{wt},i-1}}\leq{}{\mathbf{y}_{\text{slope},i-1}}
                                                                            \end{cases}
           \end{displaymath}
\\ \midrule \\
Description & \begin{symbDescription}
              \item{${\mathbf{U}_{\text{g}}}$ is the surface hydrostatic forces ($\frac{\text{N}}{\text{m}}$)}
              \item{$i$ is the index (Unitless)}
              \item{${\mathbf{ℓ}_{s}}$ is the surface lengths of slices (${\text{m}}$)}
              \item{${γ_{w}}$ is the unit weight of water ($\frac{\text{N}}{\text{m}^{3}}$)}
              \item{${\mathbf{y}_{\text{wt}}}$ is the $y$-coordinates of the water table (${\text{m}}$)}
              \item{${\mathbf{y}_{\text{slope}}}$ is the $y$-coordinates of the slope (${\text{m}}$)}
              \end{symbDescription}
\\ \midrule \\
Notes & This equation is based on the assumption that the surface of a slice is a straight line (\hyperref[assumpSBSBISL]{A: Surface-Base-Slice-between-Interslice-Straight-Lines}). ${\mathbf{ℓ}_{s}}$ is defined in \hyperref[DD:lengthLs]{DD: lengthLs}.
        
\\ \midrule \\
Source & \cite{fredlund1977}
         
\\ \midrule \\
RefBy & \hyperref[GD:srfWtrF]{GD: srfWtrF}, \hyperref[GD:resShearWO]{GD: resShearWO}, \hyperref[IM:nrmShrForNum]{IM: nrmShrForNum}, \hyperref[GD:normForcEq]{GD: normForcEq}, \hyperref[GD:momentEql]{GD: momentEql}, \hyperref[GD:mobShearWO]{GD: mobShearWO}, and \hyperref[GD:bsShrFEq]{GD: bsShrFEq}
        
\\ \bottomrule
\end{tabular}
\end{minipage}
\paragraph{}
\label{GD:srfWtrFDeriv}
The surface hydrostatic forces come from the hydrostatic pressure exerted by the water above the surface of each slice. The equation for hydrostatic pressure from \hyperref[GD:hsPressure]{GD: hsPressure} is:

\begin{displaymath}
p=γ h
\end{displaymath}
The specific weight in this case is the unit weight of water ${γ_{w}}$. The height in this case is the height from the slice surface to the water table. This height is measured from the midpoint of the slice because the resultant hydrostatic force is assumed to act at the slice midpoint (\hyperref[assumpHFSM]{A: Hydrostatic-Force-Slice-Midpoint}). The height at the midpoint is the average of the height at slice interface $i$ and the height at slice interface $i-1$:

\begin{displaymath}
\frac{1}{2} \left({\mathbf{y}_{\text{wt},i}}-{\mathbf{y}_{\text{slope},i}}+{\mathbf{y}_{\text{wt},i-1}}-{\mathbf{y}_{\text{slope},i-1}}\right)
\end{displaymath}
Due to \hyperref[assumpPSC]{A: Plane-Strain-Conditions}, only two dimensions are considered, so the surface hydrostatic forces are expressed as forces per meter. The pressures acting on the slices can thus be converted to surface hydrostatic forces by multiplying by the corresponding length of the slice surface ${\mathbf{ℓ}_{s,i}}$, assuming the water table does not intersect a slice surface except at a slice edge (\hyperref[assumpWISE]{A: Water-Intersects-Surface-Edge}). Thus, in the case where the height of the water table is above the height of the slope surface, the surface hydrostatic forces are defined as:

\begin{displaymath}
{\mathbf{U}_{\text{g},i}}={\mathbf{ℓ}_{s,i}} {γ_{w}} \frac{1}{2} \left({\mathbf{y}_{\text{wt},i}}-{\mathbf{y}_{\text{slope},i}}+{\mathbf{y}_{\text{wt},i-1}}-{\mathbf{y}_{\text{slope},i-1}}\right)
\end{displaymath}
This equation is the non-zero case of \hyperref[GD:srfWtrF]{GD: srfWtrF}. The zero case is when the height of the water table is below the height of the slope surface, so there is no hydrostatic force.

\subsubsection{Data Definitions}
\label{Sec:DDs}
This section collects and defines all the data needed to build the instance models.

\vspace{\baselineskip}
\noindent
\begin{minipage}{\textwidth}
\begin{tabular}{>{\raggedright}p{0.13\textwidth}>{\raggedright\arraybackslash}p{0.82\textwidth}}
\toprule \textbf{Refname} & \textbf{DD:intersliceWtrF}
\phantomsection 
\label{DD:intersliceWtrF}
\\ \midrule \\
Label & Interslice normal water forces
        
\\ \midrule \\
Symbol & $\mathbf{H}$
         
\\ \midrule \\
Units & $\frac{\text{N}}{\text{m}}$
        
\\ \midrule \\
Equation & \begin{displaymath}
           \mathbf{H}=\begin{cases}
                      \frac{\left({\mathbf{y}_{\text{slope},i}}-{\mathbf{y}_{\text{slip},i}}\right)^{2}}{2} {γ_{w}}+\left({\mathbf{y}_{\text{wt},i}}-{\mathbf{y}_{\text{slope},i}}\right)^{2} {γ_{w}}, & {\mathbf{y}_{\text{wt},i}}\geq{}{\mathbf{y}_{\text{slope},i}}\\
                      \frac{\left({\mathbf{y}_{\text{wt},i}}-{\mathbf{y}_{\text{slip},i}}\right)^{2}}{2} {γ_{w}}, & {\mathbf{y}_{\text{slope},i}}\gt{}{\mathbf{y}_{\text{wt},i}}\gt{}{\mathbf{y}_{\text{slip},i}}\\
                      0, & {\mathbf{y}_{\text{wt},i}}\leq{}{\mathbf{y}_{\text{slip},i}}
                      \end{cases}
           \end{displaymath}
\\ \midrule \\
Description & \begin{symbDescription}
              \item{$\mathbf{H}$ is the interslice normal water forces ($\frac{\text{N}}{\text{m}}$)}
              \item{${\mathbf{y}_{\text{slope}}}$ is the $y$-coordinates of the slope (${\text{m}}$)}
              \item{$i$ is the index (Unitless)}
              \item{${\mathbf{y}_{\text{slip}}}$ is the $y$-coordinates of the slip surface (${\text{m}}$)}
              \item{${γ_{w}}$ is the unit weight of water ($\frac{\text{N}}{\text{m}^{3}}$)}
              \item{${\mathbf{y}_{\text{wt}}}$ is the $y$-coordinates of the water table (${\text{m}}$)}
              \end{symbDescription}
\\ \midrule \\
Source & \cite{fredlund1977}
         
\\ \midrule \\
RefBy & \hyperref[GD:resShearWO]{GD: resShearWO}, \hyperref[IM:nrmShrForNum]{IM: nrmShrForNum}, and \hyperref[GD:mobShearWO]{GD: mobShearWO}
        
\\ \bottomrule
\end{tabular}
\end{minipage}

\vspace{\baselineskip}
\noindent
\begin{minipage}{\textwidth}
\begin{tabular}{>{\raggedright}p{0.13\textwidth}>{\raggedright\arraybackslash}p{0.82\textwidth}}
\toprule \textbf{Refname} & \textbf{DD:angleA}
\phantomsection 
\label{DD:angleA}
\\ \midrule \\
Label & Base angles
        
\\ \midrule \\
Symbol & $\mathbf{α}$
         
\\ \midrule \\
Units & ${{}^{\circ}}$
        
\\ \midrule \\
Equation & \begin{displaymath}
           \mathbf{α}=\arctan\left(\frac{{\mathbf{y}_{\text{slip},i}}-{\mathbf{y}_{\text{slip},i-1}}}{{\mathbf{x}_{\text{slip},i}}-{\mathbf{x}_{\text{slip},i-1}}}\right)
           \end{displaymath}
\\ \midrule \\
Description & \begin{symbDescription}
              \item{$\mathbf{α}$ is the base angles (${{}^{\circ}}$)}
              \item{${\mathbf{y}_{\text{slip}}}$ is the $y$-coordinates of the slip surface (${\text{m}}$)}
              \item{$i$ is the index (Unitless)}
              \item{${\mathbf{x}_{\text{slip}}}$ is the $x$-coordinates of the slip surface (${\text{m}}$)}
              \end{symbDescription}
\\ \midrule \\
Notes & This equation is based on the assumption that the base of a slice is a straight line (\hyperref[assumpSBSBISL]{A: Surface-Base-Slice-between-Interslice-Straight-Lines}).
        
\\ \midrule \\
Source & \cite{fredlund1977}
         
\\ \midrule \\
RefBy & \hyperref[GD:resShearWO]{GD: resShearWO}, \hyperref[IM:nrmShrForNum]{IM: nrmShrForNum}, \hyperref[GD:normForcEq]{GD: normForcEq}, \hyperref[GD:momentEql]{GD: momentEql}, \hyperref[GD:mobShearWO]{GD: mobShearWO}, \hyperref[DD:lengthLb]{DD: lengthLb}, \hyperref[GD:bsShrFEq]{GD: bsShrFEq}, \hyperref[DD:convertFunc2]{DD: convertFunc2}, and \hyperref[DD:convertFunc1]{DD: convertFunc1}
        
\\ \bottomrule
\end{tabular}
\end{minipage}

\vspace{\baselineskip}
\noindent
\begin{minipage}{\textwidth}
\begin{tabular}{>{\raggedright}p{0.13\textwidth}>{\raggedright\arraybackslash}p{0.82\textwidth}}
\toprule \textbf{Refname} & \textbf{DD:angleB}
\phantomsection 
\label{DD:angleB}
\\ \midrule \\
Label & Surface angles
        
\\ \midrule \\
Symbol & $\mathbf{β}$
         
\\ \midrule \\
Units & ${{}^{\circ}}$
        
\\ \midrule \\
Equation & \begin{displaymath}
           \mathbf{β}=\arctan\left(\frac{{\mathbf{y}_{\text{slope},i}}-{\mathbf{y}_{\text{slope},i-1}}}{{\mathbf{x}_{\text{slope},i}}-{\mathbf{x}_{\text{slope},i-1}}}\right)
           \end{displaymath}
\\ \midrule \\
Description & \begin{symbDescription}
              \item{$\mathbf{β}$ is the surface angles (${{}^{\circ}}$)}
              \item{${\mathbf{y}_{\text{slope}}}$ is the $y$-coordinates of the slope (${\text{m}}$)}
              \item{$i$ is the index (Unitless)}
              \item{${\mathbf{x}_{\text{slope}}}$ is the $x$-coordinates of the slope (${\text{m}}$)}
              \end{symbDescription}
\\ \midrule \\
Notes & This equation is based on the assumption that the surface of a slice is a straight line (\hyperref[assumpSBSBISL]{A: Surface-Base-Slice-between-Interslice-Straight-Lines}).
        
\\ \midrule \\
Source & \cite{fredlund1977}
         
\\ \midrule \\
RefBy & \hyperref[GD:resShearWO]{GD: resShearWO}, \hyperref[IM:nrmShrForNum]{IM: nrmShrForNum}, \hyperref[GD:normForcEq]{GD: normForcEq}, \hyperref[GD:momentEql]{GD: momentEql}, \hyperref[GD:mobShearWO]{GD: mobShearWO}, \hyperref[DD:lengthLs]{DD: lengthLs}, and \hyperref[GD:bsShrFEq]{GD: bsShrFEq}
        
\\ \bottomrule
\end{tabular}
\end{minipage}

\vspace{\baselineskip}
\noindent
\begin{minipage}{\textwidth}
\begin{tabular}{>{\raggedright}p{0.13\textwidth}>{\raggedright\arraybackslash}p{0.82\textwidth}}
\toprule \textbf{Refname} & \textbf{DD:lengthB}
\phantomsection 
\label{DD:lengthB}
\\ \midrule \\
Label & Base width of slices
        
\\ \midrule \\
Symbol & $\mathbf{b}$
         
\\ \midrule \\
Units & ${\text{m}}$
        
\\ \midrule \\
Equation & \begin{displaymath}
           \mathbf{b}={\mathbf{x}_{\text{slip},i}}-{\mathbf{x}_{\text{slip},i-1}}
           \end{displaymath}
\\ \midrule \\
Description & \begin{symbDescription}
              \item{$\mathbf{b}$ is the base width of slices (${\text{m}}$)}
              \item{${\mathbf{x}_{\text{slip}}}$ is the $x$-coordinates of the slip surface (${\text{m}}$)}
              \item{$i$ is the index (Unitless)}
              \end{symbDescription}
\\ \midrule \\
Source & \cite{fredlund1977}
         
\\ \midrule \\
RefBy & \hyperref[GD:sliceWght]{GD: sliceWght}, \hyperref[IM:nrmShrForNum]{IM: nrmShrForNum}, \hyperref[IM:nrmShrForDen]{IM: nrmShrForDen}, \hyperref[GD:momentEql]{GD: momentEql}, \hyperref[DD:lengthLs]{DD: lengthLs}, and \hyperref[DD:lengthLb]{DD: lengthLb}
        
\\ \bottomrule
\end{tabular}
\end{minipage}

\vspace{\baselineskip}
\noindent
\begin{minipage}{\textwidth}
\begin{tabular}{>{\raggedright}p{0.13\textwidth}>{\raggedright\arraybackslash}p{0.82\textwidth}}
\toprule \textbf{Refname} & \textbf{DD:lengthLb}
\phantomsection 
\label{DD:lengthLb}
\\ \midrule \\
Label & Total base lengths of slices
        
\\ \midrule \\
Symbol & ${\mathbf{ℓ}_{b}}$
         
\\ \midrule \\
Units & ${\text{m}}$
        
\\ \midrule \\
Equation & \begin{displaymath}
           {\mathbf{ℓ}_{b}}={\mathbf{b}}_{i} \sec\left({\mathbf{α}}_{i}\right)
           \end{displaymath}
\\ \midrule \\
Description & \begin{symbDescription}
              \item{${\mathbf{ℓ}_{b}}$ is the total base lengths of slices (${\text{m}}$)}
              \item{$\mathbf{b}$ is the base width of slices (${\text{m}}$)}
              \item{$i$ is the index (Unitless)}
              \item{$\mathbf{α}$ is the base angles (${{}^{\circ}}$)}
              \end{symbDescription}
\\ \midrule \\
Notes & $\mathbf{b}$ is defined in \hyperref[DD:lengthB]{DD: lengthB} and $\mathbf{α}$ is defined in \hyperref[DD:angleA]{DD: angleA}.
        
\\ \midrule \\
Source & \cite{fredlund1977}
         
\\ \midrule \\
RefBy & \hyperref[GD:resShr]{GD: resShr}, \hyperref[GD:resShearWO]{GD: resShearWO}, \hyperref[GD:mobShr]{GD: mobShr}, and \hyperref[GD:baseWtrF]{GD: baseWtrF}
        
\\ \bottomrule
\end{tabular}
\end{minipage}

\vspace{\baselineskip}
\noindent
\begin{minipage}{\textwidth}
\begin{tabular}{>{\raggedright}p{0.13\textwidth}>{\raggedright\arraybackslash}p{0.82\textwidth}}
\toprule \textbf{Refname} & \textbf{DD:lengthLs}
\phantomsection 
\label{DD:lengthLs}
\\ \midrule \\
Label & Surface lengths of slices
        
\\ \midrule \\
Symbol & ${\mathbf{ℓ}_{s}}$
         
\\ \midrule \\
Units & ${\text{m}}$
        
\\ \midrule \\
Equation & \begin{displaymath}
           {\mathbf{ℓ}_{s}}={\mathbf{b}}_{i} \sec\left({\mathbf{β}}_{i}\right)
           \end{displaymath}
\\ \midrule \\
Description & \begin{symbDescription}
              \item{${\mathbf{ℓ}_{s}}$ is the surface lengths of slices (${\text{m}}$)}
              \item{$\mathbf{b}$ is the base width of slices (${\text{m}}$)}
              \item{$i$ is the index (Unitless)}
              \item{$\mathbf{β}$ is the surface angles (${{}^{\circ}}$)}
              \end{symbDescription}
\\ \midrule \\
Notes & $\mathbf{b}$ is defined in \hyperref[DD:lengthB]{DD: lengthB} and $\mathbf{β}$ is defined in \hyperref[DD:angleB]{DD: angleB}.
        
\\ \midrule \\
Source & \cite{fredlund1977}
         
\\ \midrule \\
RefBy & \hyperref[GD:srfWtrF]{GD: srfWtrF}
        
\\ \bottomrule
\end{tabular}
\end{minipage}

\vspace{\baselineskip}
\noindent
\begin{minipage}{\textwidth}
\begin{tabular}{>{\raggedright}p{0.13\textwidth}>{\raggedright\arraybackslash}p{0.82\textwidth}}
\toprule \textbf{Refname} & \textbf{DD:slcHeight}
\phantomsection 
\label{DD:slcHeight}
\\ \midrule \\
Label & $y$-direction heights of slices
        
\\ \midrule \\
Symbol & $\mathbf{h}$
         
\\ \midrule \\
Units & ${\text{m}}$
        
\\ \midrule \\
Equation & \begin{displaymath}
           \mathbf{h}=\frac{1}{2} \left({\mathbf{h}^{\text{R}}}+{\mathbf{h}^{\text{L}}}\right)
           \end{displaymath}
\\ \midrule \\
Description & \begin{symbDescription}
              \item{$\mathbf{h}$ is the $y$-direction heights of slices (${\text{m}}$)}
              \item{${\mathbf{h}^{\text{R}}}$ is the heights of the right side of slices (${\text{m}}$)}
              \item{${\mathbf{h}^{\text{L}}}$ is the heights of the left side of slices (${\text{m}}$)}
              \end{symbDescription}
\\ \midrule \\
Notes & This equation is based on the assumption that the surface and base of a slice are straight lines (\hyperref[assumpSBSBISL]{A: Surface-Base-Slice-between-Interslice-Straight-Lines}).
        
        ${\mathbf{h}^{\text{R}}}$ and ${\mathbf{h}^{\text{L}}}$ are defined in \hyperref[DD:sliceHghtRightDD]{DD: sliceHghtRightDD} and \hyperref[DD:sliceHghtLeftDD]{DD: sliceHghtLeftDD}, respectively.
        
\\ \midrule \\
Source & \cite{fredlund1977}
         
\\ \midrule \\
RefBy & \hyperref[IM:nrmShrForNum]{IM: nrmShrForNum} and \hyperref[GD:momentEql]{GD: momentEql}
        
\\ \bottomrule
\end{tabular}
\end{minipage}

\vspace{\baselineskip}
\noindent
\begin{minipage}{\textwidth}
\begin{tabular}{>{\raggedright}p{0.13\textwidth}>{\raggedright\arraybackslash}p{0.82\textwidth}}
\toprule \textbf{Refname} & \textbf{DD:normStress}
\phantomsection 
\label{DD:normStress}
\\ \midrule \\
Label & Total normal stress
        
\\ \midrule \\
Symbol & $σ$
         
\\ \midrule \\
Units & ${\text{Pa}}$
        
\\ \midrule \\
Equation & \begin{displaymath}
           σ=\frac{{F_{\text{n}}}}{A}
           \end{displaymath}
\\ \midrule \\
Description & \begin{symbDescription}
              \item{$σ$ is the total normal stress (${\text{Pa}}$)}
              \item{${F_{\text{n}}}$ is the total normal force (${\text{N}}$)}
              \item{$A$ is the area (${\text{m}^{2}}$)}
              \end{symbDescription}
\\ \midrule \\
Source & \cite{huston2008}
         
\\ \midrule \\
RefBy & \hyperref[GD:resShr]{GD: resShr}, \hyperref[TM:effStress]{TM: effStress}, and \hyperref[GD:effNormF]{GD: effNormF}
        
\\ \bottomrule
\end{tabular}
\end{minipage}

\vspace{\baselineskip}
\noindent
\begin{minipage}{\textwidth}
\begin{tabular}{>{\raggedright}p{0.13\textwidth}>{\raggedright\arraybackslash}p{0.82\textwidth}}
\toprule \textbf{Refname} & \textbf{DD:tangStress}
\phantomsection 
\label{DD:tangStress}
\\ \midrule \\
Label & Tangential stress
        
\\ \midrule \\
Symbol & $τ$
         
\\ \midrule \\
Units & ${\text{Pa}}$
        
\\ \midrule \\
Equation & \begin{displaymath}
           τ=\frac{{F_{\text{t}}}}{A}
           \end{displaymath}
\\ \midrule \\
Description & \begin{symbDescription}
              \item{$τ$ is the tangential stress (${\text{Pa}}$)}
              \item{${F_{\text{t}}}$ is the tangential force (${\text{N}}$)}
              \item{$A$ is the area (${\text{m}^{2}}$)}
              \end{symbDescription}
\\ \midrule \\
Source & \cite{huston2008}
         
\\ \midrule \\
RefBy & \hyperref[GD:resShr]{GD: resShr}
        
\\ \bottomrule
\end{tabular}
\end{minipage}

\vspace{\baselineskip}
\noindent
\begin{minipage}{\textwidth}
\begin{tabular}{>{\raggedright}p{0.13\textwidth}>{\raggedright\arraybackslash}p{0.82\textwidth}}
\toprule \textbf{Refname} & \textbf{DD:torque}
\phantomsection 
\label{DD:torque}
\\ \midrule \\
Label & Torque
        
\\ \midrule \\
Symbol & $\mathbf{τ}$
         
\\ \midrule \\
Units & $\text{N}\text{m}$
        
\\ \midrule \\
Equation & \begin{displaymath}
           \mathbf{τ}=\mathbf{r}\times\mathbf{F}
           \end{displaymath}
\\ \midrule \\
Description & \begin{symbDescription}
              \item{$\mathbf{τ}$ is the torque ($\text{N}\text{m}$)}
              \item{$\mathbf{r}$ is the position vector (${\text{m}}$)}
              \item{$\mathbf{F}$ is the force (${\text{N}}$)}
              \end{symbDescription}
\\ \midrule \\
Notes & The torque on a body measures the the tendency of a force to rotate the body around an axis or pivot.
        
\\ \midrule \\
Source & --
         
\\ \midrule \\
RefBy & \hyperref[GD:momentEql]{GD: momentEql}
        
\\ \bottomrule
\end{tabular}
\end{minipage}

\vspace{\baselineskip}
\noindent
\begin{minipage}{\textwidth}
\begin{tabular}{>{\raggedright}p{0.13\textwidth}>{\raggedright\arraybackslash}p{0.82\textwidth}}
\toprule \textbf{Refname} & \textbf{DD:ratioVariation}
\phantomsection 
\label{DD:ratioVariation}
\\ \midrule \\
Label & Interslice normal to shear force ratio variation function
        
\\ \midrule \\
Symbol & $\mathbf{f}$
         
\\ \midrule \\
Units & Unitless
        
\\ \midrule \\
Equation & \begin{displaymath}
           \mathbf{f}=\begin{cases}
                      1, & const_f\\
                      \sin\left(π \frac{{\mathbf{x}_{\text{slip},i}}-{\mathbf{x}_{\text{slip},0}}}{{\mathbf{x}_{\text{slip},n}}-{\mathbf{x}_{\text{slip},0}}}\right), & \neg{}const_f
                      \end{cases}
           \end{displaymath}
\\ \midrule \\
Description & \begin{symbDescription}
              \item{$\mathbf{f}$ is the interslice normal to shear force ratio variation function (Unitless)}
              \item{$π$ is the ratio of circumference to diameter for any circle (Unitless)}
              \item{${\mathbf{x}_{\text{slip}}}$ is the $x$-coordinates of the slip surface (${\text{m}}$)}
              \item{$i$ is the index (Unitless)}
              \item{$n$ is the number of slices (Unitless)}
              \item{$const_f$ is the decision on f (Unitless)}
              \end{symbDescription}
\\ \midrule \\
Source & \cite{fredlund1977}
         
\\ \midrule \\
RefBy & \hyperref[IM:nrmShrForDen]{IM: nrmShrForDen}, \hyperref[GD:normShrR]{GD: normShrR}, \hyperref[DD:convertFunc2]{DD: convertFunc2}, and \hyperref[DD:convertFunc1]{DD: convertFunc1}
        
\\ \bottomrule
\end{tabular}
\end{minipage}

\vspace{\baselineskip}
\noindent
\begin{minipage}{\textwidth}
\begin{tabular}{>{\raggedright}p{0.13\textwidth}>{\raggedright\arraybackslash}p{0.82\textwidth}}
\toprule \textbf{Refname} & \textbf{DD:convertFunc1}
\phantomsection 
\label{DD:convertFunc1}
\\ \midrule \\
Label & First function for incorporating interslice forces into shear force
        
\\ \midrule \\
Symbol & $\mathbf{Φ}$
         
\\ \midrule \\
Units & Unitless
        
\\ \midrule \\
Equation & \begin{displaymath}
           \mathbf{Φ}=\left(λ {\mathbf{f}}_{i} \cos\left({\mathbf{α}}_{i}\right)-\sin\left({\mathbf{α}}_{i}\right)\right) \tan\left(φ'\right)-\left(λ {\mathbf{f}}_{i} \sin\left({\mathbf{α}}_{i}\right)+\cos\left({\mathbf{α}}_{i}\right)\right) {F_{\text{S}}}
           \end{displaymath}
\\ \midrule \\
Description & \begin{symbDescription}
              \item{$\mathbf{Φ}$ is the first function for incorporating interslice forces into shear force (Unitless)}
              \item{$λ$ is the proportionality constant (Unitless)}
              \item{$\mathbf{f}$ is the interslice normal to shear force ratio variation function (Unitless)}
              \item{$i$ is the index (Unitless)}
              \item{$\mathbf{α}$ is the base angles (${{}^{\circ}}$)}
              \item{$φ'$ is the effective angle of friction (${{}^{\circ}}$)}
              \item{${F_{\text{S}}}$ is the factor of safety (Unitless)}
              \end{symbDescription}
\\ \midrule \\
Notes & $\mathbf{f}$ is defined in \hyperref[DD:ratioVariation]{DD: ratioVariation} and $\mathbf{α}$ is defined in \hyperref[DD:angleA]{DD: angleA}.
        
\\ \midrule \\
Source & \cite{chen2005} and \cite{karchewski2012}
         
\\ \midrule \\
RefBy & \hyperref[IM:intsliceFs]{IM: intsliceFs}, \hyperref[IM:fctSfty]{IM: fctSfty}, and \hyperref[DD:convertFunc2]{DD: convertFunc2}
        
\\ \bottomrule
\end{tabular}
\end{minipage}

\vspace{\baselineskip}
\noindent
\begin{minipage}{\textwidth}
\begin{tabular}{>{\raggedright}p{0.13\textwidth}>{\raggedright\arraybackslash}p{0.82\textwidth}}
\toprule \textbf{Refname} & \textbf{DD:convertFunc2}
\phantomsection 
\label{DD:convertFunc2}
\\ \midrule \\
Label & Second function for incorporating interslice forces into shear force
        
\\ \midrule \\
Symbol & $\mathbf{Ψ}$
         
\\ \midrule \\
Units & Unitless
        
\\ \midrule \\
Equation & \begin{displaymath}
           \mathbf{Ψ}=\frac{\left(λ {\mathbf{f}}_{i} \cos\left({\mathbf{α}}_{i}\right)-\sin\left({\mathbf{α}}_{i}\right)\right) \tan\left(φ'\right)-\left(λ {\mathbf{f}}_{i} \sin\left({\mathbf{α}}_{i}\right)+\cos\left({\mathbf{α}}_{i}\right)\right) {F_{\text{S}}}}{{\mathbf{Φ}}_{i-1}}
           \end{displaymath}
\\ \midrule \\
Description & \begin{symbDescription}
              \item{$\mathbf{Ψ}$ is the second function for incorporating interslice forces into shear force (Unitless)}
              \item{$λ$ is the proportionality constant (Unitless)}
              \item{$\mathbf{f}$ is the interslice normal to shear force ratio variation function (Unitless)}
              \item{$i$ is the index (Unitless)}
              \item{$\mathbf{α}$ is the base angles (${{}^{\circ}}$)}
              \item{$φ'$ is the effective angle of friction (${{}^{\circ}}$)}
              \item{${F_{\text{S}}}$ is the factor of safety (Unitless)}
              \item{$\mathbf{Φ}$ is the first function for incorporating interslice forces into shear force (Unitless)}
              \end{symbDescription}
\\ \midrule \\
Notes & $\mathbf{f}$ is defined in \hyperref[DD:ratioVariation]{DD: ratioVariation}, $\mathbf{α}$ is defined in \hyperref[DD:angleA]{DD: angleA}, and $\mathbf{Φ}$ is defined in \hyperref[DD:convertFunc1]{DD: convertFunc1}.
        
\\ \midrule \\
Source & \cite{chen2005} and \cite{karchewski2012}
         
\\ \midrule \\
RefBy & \hyperref[IM:intsliceFs]{IM: intsliceFs} and \hyperref[IM:fctSfty]{IM: fctSfty}
        
\\ \bottomrule
\end{tabular}
\end{minipage}

\vspace{\baselineskip}
\noindent
\begin{minipage}{\textwidth}
\begin{tabular}{>{\raggedright}p{0.13\textwidth}>{\raggedright\arraybackslash}p{0.82\textwidth}}
\toprule \textbf{Refname} & \textbf{DD:nrmForceSumDD}
\phantomsection 
\label{DD:nrmForceSumDD}
\\ \midrule \\
Label & Sums of the interslice normal forces
        
\\ \midrule \\
Symbol & ${{\mathbf{F}_{\text{x}}}^{\text{G}}}$
         
\\ \midrule \\
Units & ${\text{N}}$
        
\\ \midrule \\
Equation & \begin{displaymath}
           {{\mathbf{F}_{\text{x}}}^{\text{G}}}={\mathbf{G}}_{i}+{\mathbf{G}}_{i-1}
           \end{displaymath}
\\ \midrule \\
Description & \begin{symbDescription}
              \item{${{\mathbf{F}_{\text{x}}}^{\text{G}}}$ is the sums of the interslice normal forces (${\text{N}}$)}
              \item{$\mathbf{G}$ is the interslice normal forces ($\frac{\text{N}}{\text{m}}$)}
              \item{$i$ is the index (Unitless)}
              \end{symbDescription}
\\ \midrule \\
Source & \cite{fredlund1977}
         
\\ \midrule \\
RefBy & 
\\ \bottomrule
\end{tabular}
\end{minipage}

\vspace{\baselineskip}
\noindent
\begin{minipage}{\textwidth}
\begin{tabular}{>{\raggedright}p{0.13\textwidth}>{\raggedright\arraybackslash}p{0.82\textwidth}}
\toprule \textbf{Refname} & \textbf{DD:watForceSumDD}
\phantomsection 
\label{DD:watForceSumDD}
\\ \midrule \\
Label & Sums of the interslice normal water forces
        
\\ \midrule \\
Symbol & ${{\mathbf{F}_{\text{x}}}^{\text{H}}}$
         
\\ \midrule \\
Units & ${\text{N}}$
        
\\ \midrule \\
Equation & \begin{displaymath}
           {{\mathbf{F}_{\text{x}}}^{\text{H}}}={\mathbf{H}}_{i}+{\mathbf{H}}_{i-1}
           \end{displaymath}
\\ \midrule \\
Description & \begin{symbDescription}
              \item{${{\mathbf{F}_{\text{x}}}^{\text{H}}}$ is the sums of the interslice normal water forces (${\text{N}}$)}
              \item{$\mathbf{H}$ is the interslice normal water forces ($\frac{\text{N}}{\text{m}}$)}
              \item{$i$ is the index (Unitless)}
              \end{symbDescription}
\\ \midrule \\
Source & \cite{fredlund1977}
         
\\ \midrule \\
RefBy & 
\\ \bottomrule
\end{tabular}
\end{minipage}

\vspace{\baselineskip}
\noindent
\begin{minipage}{\textwidth}
\begin{tabular}{>{\raggedright}p{0.13\textwidth}>{\raggedright\arraybackslash}p{0.82\textwidth}}
\toprule \textbf{Refname} & \textbf{DD:sliceHghtRightDD}
\phantomsection 
\label{DD:sliceHghtRightDD}
\\ \midrule \\
Label & Heights of the right side of slices
        
\\ \midrule \\
Symbol & ${\mathbf{h}^{\text{R}}}$
         
\\ \midrule \\
Units & ${\text{m}}$
        
\\ \midrule \\
Equation & \begin{displaymath}
           {\mathbf{h}^{\text{R}}}={\mathbf{y}_{\text{slope},i}}-{\mathbf{y}_{\text{slip},i}}
           \end{displaymath}
\\ \midrule \\
Description & \begin{symbDescription}
              \item{${\mathbf{h}^{\text{R}}}$ is the heights of the right side of slices (${\text{m}}$)}
              \item{${\mathbf{y}_{\text{slope}}}$ is the $y$-coordinates of the slope (${\text{m}}$)}
              \item{$i$ is the index (Unitless)}
              \item{${\mathbf{y}_{\text{slip}}}$ is the $y$-coordinates of the slip surface (${\text{m}}$)}
              \end{symbDescription}
\\ \midrule \\
Source & \cite{fredlund1977}
         
\\ \midrule \\
RefBy & \hyperref[DD:slcHeight]{DD: slcHeight}
        
\\ \bottomrule
\end{tabular}
\end{minipage}

\vspace{\baselineskip}
\noindent
\begin{minipage}{\textwidth}
\begin{tabular}{>{\raggedright}p{0.13\textwidth}>{\raggedright\arraybackslash}p{0.82\textwidth}}
\toprule \textbf{Refname} & \textbf{DD:sliceHghtLeftDD}
\phantomsection 
\label{DD:sliceHghtLeftDD}
\\ \midrule \\
Label & Heights of the left side of slices
        
\\ \midrule \\
Symbol & ${\mathbf{h}^{\text{L}}}$
         
\\ \midrule \\
Units & ${\text{m}}$
        
\\ \midrule \\
Equation & \begin{displaymath}
           {\mathbf{h}^{\text{L}}}={\mathbf{y}_{\text{slope},i-1}}-{\mathbf{y}_{\text{slip},i-1}}
           \end{displaymath}
\\ \midrule \\
Description & \begin{symbDescription}
              \item{${\mathbf{h}^{\text{L}}}$ is the heights of the left side of slices (${\text{m}}$)}
              \item{${\mathbf{y}_{\text{slope}}}$ is the $y$-coordinates of the slope (${\text{m}}$)}
              \item{$i$ is the index (Unitless)}
              \item{${\mathbf{y}_{\text{slip}}}$ is the $y$-coordinates of the slip surface (${\text{m}}$)}
              \end{symbDescription}
\\ \midrule \\
Source & \cite{fredlund1977}
         
\\ \midrule \\
RefBy & \hyperref[DD:slcHeight]{DD: slcHeight}
        
\\ \bottomrule
\end{tabular}
\end{minipage}

\subsubsection{Instance Models}
\label{Sec:IMs}
This section transforms the problem defined in \hyperref[Sec:ProbDesc]{Section: Problem Description} into one which is expressed in mathematical terms. It uses concrete symbols defined in \hyperref[Sec:DDs]{Section: Data Definitions} to replace the abstract symbols in the models identified in \hyperref[Sec:TMs]{Section: Theoretical Models} and \hyperref[Sec:GDs]{Section: General Definitions}.

The goals \hyperref[identifyCritAndFS]{GS: Identify-Crit-and-FS}, \hyperref[determineNormalF]{GS: Determine-Normal-Forces}, and \hyperref[determineShearF]{GS: Determine-Shear-Forces} are met by the simultaneous solution of \hyperref[IM:fctSfty]{IM: fctSfty}, \hyperref[IM:nrmShrFor]{IM: nrmShrFor}, and \hyperref[IM:intsliceFs]{IM: intsliceFs}. The goal \hyperref[identifyCritAndFS]{GS: Identify-Crit-and-FS} is also contributed to by \hyperref[IM:crtSlpId]{IM: crtSlpId}.

The Morgenstern-Price method is a vertical slice, limit equilibrium slope stability analysis method. Analysis is performed by breaking the assumed slip surface into a series of vertical slices of mass. Static equilibrium analysis is performed, using two force equations and one moment equation as in \hyperref[TM:equilibrium]{TM: equilibrium}. The problem is statically indeterminate with only these 3 equations and one constitutive equation (the Mohr Coulomb shear strength of \hyperref[TM:mcShrStrgth]{TM: mcShrStrgth}) so the assumption \hyperref[GD:normShrR]{GD: normShrR} and corresponding equation \hyperref[GD:normShrR]{GD: normShrR} are used. The force equilibrium equations can be modified to be expressed only in terms of known physical values, as done in \hyperref[GD:resShearWO]{GD: resShearWO} and \hyperref[GD:mobShearWO]{GD: mobShearWO}.

\vspace{\baselineskip}
\noindent
\begin{minipage}{\textwidth}
\begin{tabular}{>{\raggedright}p{0.13\textwidth}>{\raggedright\arraybackslash}p{0.82\textwidth}}
\toprule \textbf{Refname} & \textbf{IM:fctSfty}
\phantomsection 
\label{IM:fctSfty}
\\ \midrule \\
Label & Factor of safety
        
\\ \midrule \\
Input & ${\mathbf{x}_{\text{slope}}}$, ${\mathbf{y}_{\text{slope}}}$, ${\mathbf{y}_{\text{wt}}}$, $c'$, $φ'$, ${γ_{\text{dry}}}$, ${γ_{\text{sat}}}$, ${γ_{w}}$, ${\mathbf{x}_{\text{slip}}}$, ${\mathbf{y}_{\text{slip}}}$, $const_f$
        
\\ \midrule \\
Output & ${F_{\text{S}}}$
         
\\ \midrule \\
Input Constraints & 
\\ \midrule \\
Output Constraints & 
\\ \midrule \\
Equation & \begin{displaymath}
           {F_{\text{S}}}=\frac{\displaystyle\sum_{i=1}^{n-1}{{\mathbf{R}}_{i} \displaystyle\prod_{v=i}^{n-1}{{\mathbf{Ψ}}_{v}}}+{\mathbf{R}}_{n}}{\displaystyle\sum_{i=1}^{n-1}{{\mathbf{T}}_{i} \displaystyle\prod_{v=i}^{n-1}{{\mathbf{Ψ}}_{v}}}+{\mathbf{T}}_{n}}
           \end{displaymath}
\\ \midrule \\
Description & \begin{symbDescription}
              \item{${F_{\text{S}}}$ is the factor of safety (Unitless)}
              \item{$\mathbf{R}$ is the resistive shear forces without the influence of interslice forces ($\frac{\text{N}}{\text{m}}$)}
              \item{$i$ is the index (Unitless)}
              \item{$\mathbf{Ψ}$ is the second function for incorporating interslice forces into shear force (Unitless)}
              \item{$v$ is the local index (Unitless)}
              \item{$n$ is the number of slices (Unitless)}
              \item{$\mathbf{T}$ is the mobilized shear forces without the influence of interslice forces ($\frac{\text{N}}{\text{m}}$)}
              \end{symbDescription}
\\ \midrule \\
Notes & $\mathbf{R}$ is defined in \hyperref[GD:resShearWO]{GD: resShearWO}, $\mathbf{Ψ}$ is defined in \hyperref[DD:convertFunc2]{DD: convertFunc2}, and $\mathbf{T}$ is defined in \hyperref[GD:mobShearWO]{GD: mobShearWO}
        
\\ \midrule \\
Source & \cite{chen2005} and \cite{karchewski2012}
         
\\ \midrule \\
RefBy & \hyperref[IM:nrmShrFor]{IM: nrmShrFor}, \hyperref[IM:intsliceFs]{IM: intsliceFs}, \hyperref[IM:fctSfty]{IM: fctSfty}, \hyperref[displayShear]{FR: Display-Interslice-Shear-Forces}, \hyperref[displayNormal]{FR: Display-Interslice-Normal-Forces}, \hyperref[displayFS]{FR: Display-Factor-of-Safety}, and \hyperref[determineCritSlip]{FR: Determine-Critical-Slip-Surface}
        
\\ \bottomrule
\end{tabular}
\end{minipage}
\paragraph{}
\label{IM:fctSftyDeriv}
The mobilized shear force defined in \hyperref[GD:bsShrFEq]{GD: bsShrFEq} can be substituted into the definition of mobilized shear force based on the factor of safety, from \hyperref[GD:mobShr]{GD: mobShr} yielding Equation (1) below:

\begin{displaymath}
\left({\mathbf{W}}_{i}-{\mathbf{X}}_{i-1}+{\mathbf{X}}_{i}+{\mathbf{U}_{\text{g},i}} \cos\left({\mathbf{β}}_{i}\right)+{\mathbf{Q}}_{i} \cos\left({\mathbf{ω}}_{i}\right)\right) \sin\left({\mathbf{α}}_{i}\right)-\left(-{K_{\text{c}}} {\mathbf{W}}_{i}-{\mathbf{G}}_{i}+{\mathbf{G}}_{i-1}-{\mathbf{H}}_{i}+{\mathbf{H}}_{i-1}+{\mathbf{U}_{\text{g},i}} \sin\left({\mathbf{β}}_{i}\right)+{\mathbf{Q}}_{i} \sin\left({\mathbf{ω}}_{i}\right)\right) \cos\left({\mathbf{α}}_{i}\right)=\frac{{\mathbf{N'}}_{i} \tan\left(φ'\right)+c' {\mathbf{ℓ}_{b,i}}}{{F_{\text{S}}}}
\end{displaymath}
An expression for the effective normal forces, $\mathbf{N'}$, can be derived by substituting the normal forces equilibrium from \hyperref[GD:normForcEq]{GD: normForcEq} into the definition for effective normal forces from \hyperref[GD:resShearWO]{GD: resShearWO}. This results in Equation (2):

\begin{displaymath}
{\mathbf{N'}}_{i}=\left({\mathbf{W}}_{i}-{\mathbf{X}}_{i-1}+{\mathbf{X}}_{i}+{\mathbf{U}_{\text{g},i}} \cos\left({\mathbf{β}}_{i}\right)+{\mathbf{Q}}_{i} \cos\left({\mathbf{ω}}_{i}\right)\right) \cos\left({\mathbf{α}}_{i}\right)+\left(-{K_{\text{c}}} {\mathbf{W}}_{i}-{\mathbf{G}}_{i}+{\mathbf{G}}_{i-1}-{\mathbf{H}}_{i}+{\mathbf{H}}_{i-1}+{\mathbf{U}_{\text{g},i}} \sin\left({\mathbf{β}}_{i}\right)+{\mathbf{Q}}_{i} \sin\left({\mathbf{ω}}_{i}\right)\right) \sin\left({\mathbf{α}}_{i}\right)-{\mathbf{U}_{\text{b},i}}
\end{displaymath}
Substituting Equation (2) into Equation (1) gives:

\begin{displaymath}
\left({\mathbf{W}}_{i}-{\mathbf{X}}_{i-1}+{\mathbf{X}}_{i}+{\mathbf{U}_{\text{g},i}} \cos\left({\mathbf{β}}_{i}\right)+{\mathbf{Q}}_{i} \cos\left({\mathbf{ω}}_{i}\right)\right) \sin\left({\mathbf{α}}_{i}\right)-\left(-{K_{\text{c}}} {\mathbf{W}}_{i}-{\mathbf{G}}_{i}+{\mathbf{G}}_{i-1}-{\mathbf{H}}_{i}+{\mathbf{H}}_{i-1}+{\mathbf{U}_{\text{g},i}} \sin\left({\mathbf{β}}_{i}\right)+{\mathbf{Q}}_{i} \sin\left({\mathbf{ω}}_{i}\right)\right) \cos\left({\mathbf{α}}_{i}\right)=\frac{\left(\left({\mathbf{W}}_{i}-{\mathbf{X}}_{i-1}+{\mathbf{X}}_{i}+{\mathbf{U}_{\text{g},i}} \cos\left({\mathbf{β}}_{i}\right)+{\mathbf{Q}}_{i} \cos\left({\mathbf{ω}}_{i}\right)\right) \cos\left({\mathbf{α}}_{i}\right)+\left(-{K_{\text{c}}} {\mathbf{W}}_{i}-{\mathbf{G}}_{i}+{\mathbf{G}}_{i-1}-{\mathbf{H}}_{i}+{\mathbf{H}}_{i-1}+{\mathbf{U}_{\text{g},i}} \sin\left({\mathbf{β}}_{i}\right)+{\mathbf{Q}}_{i} \sin\left({\mathbf{ω}}_{i}\right)\right) \sin\left({\mathbf{α}}_{i}\right)-{\mathbf{U}_{\text{b},i}}\right) \tan\left(φ'\right)+c' {\mathbf{ℓ}_{b,i}}}{{F_{\text{S}}}}
\end{displaymath}
Since the interslice shear forces $\mathbf{X}$ and interslice normal forces $\mathbf{G}$ are unknown, they are separated from the other terms as follows:

\begin{displaymath}
\left({\mathbf{W}}_{i}+{\mathbf{U}_{\text{g},i}} \cos\left({\mathbf{β}}_{i}\right)+{\mathbf{Q}}_{i} \cos\left({\mathbf{ω}}_{i}\right)\right) \sin\left({\mathbf{α}}_{i}\right)-\left(-{K_{\text{c}}} {\mathbf{W}}_{i}-{\mathbf{H}}_{i}+{\mathbf{H}}_{i-1}+{\mathbf{U}_{\text{g},i}} \sin\left({\mathbf{β}}_{i}\right)+{\mathbf{Q}}_{i} \sin\left({\mathbf{ω}}_{i}\right)\right) \cos\left({\mathbf{α}}_{i}\right)-\left(-{\mathbf{G}}_{i}+{\mathbf{G}}_{i-1}\right) \cos\left({\mathbf{α}}_{i}\right)+\left(-{\mathbf{X}}_{i-1}+{\mathbf{X}}_{i}\right) \sin\left({\mathbf{α}}_{i}\right)=\frac{\left(\left({\mathbf{W}}_{i}+{\mathbf{U}_{\text{g},i}} \cos\left({\mathbf{β}}_{i}\right)+{\mathbf{Q}}_{i} \cos\left({\mathbf{ω}}_{i}\right)\right) \cos\left({\mathbf{α}}_{i}\right)+\left(-{K_{\text{c}}} {\mathbf{W}}_{i}-{\mathbf{H}}_{i}+{\mathbf{H}}_{i-1}+{\mathbf{U}_{\text{g},i}} \sin\left({\mathbf{β}}_{i}\right)+{\mathbf{Q}}_{i} \sin\left({\mathbf{ω}}_{i}\right)\right) \sin\left({\mathbf{α}}_{i}\right)+\left(-{\mathbf{G}}_{i}+{\mathbf{G}}_{i-1}\right) \sin\left({\mathbf{α}}_{i}\right)+\left(-{\mathbf{X}}_{i-1}+{\mathbf{X}}_{i}\right) \cos\left({\mathbf{α}}_{i}\right)-{\mathbf{U}_{\text{b},i}}\right) \tan\left(φ'\right)+c' {\mathbf{ℓ}_{b,i}}}{{F_{\text{S}}}}
\end{displaymath}
Applying assumptions \hyperref[assumpSF]{A: Seismic-Force} and \hyperref[assumpSL]{A: Surface-Load}, which state that the seismic coefficient and the external forces, respectively, are zero, allows for further simplification as shown below:

\begin{displaymath}
\left({\mathbf{W}}_{i}+{\mathbf{U}_{\text{g},i}} \cos\left({\mathbf{β}}_{i}\right)\right) \sin\left({\mathbf{α}}_{i}\right)-\left(-{\mathbf{H}}_{i}+{\mathbf{H}}_{i-1}+{\mathbf{U}_{\text{g},i}} \sin\left({\mathbf{β}}_{i}\right)\right) \cos\left({\mathbf{α}}_{i}\right)-\left(-{\mathbf{G}}_{i}+{\mathbf{G}}_{i-1}\right) \cos\left({\mathbf{α}}_{i}\right)+\left(-{\mathbf{X}}_{i-1}+{\mathbf{X}}_{i}\right) \sin\left({\mathbf{α}}_{i}\right)=\frac{\left(\left({\mathbf{W}}_{i}+{\mathbf{U}_{\text{g},i}} \cos\left({\mathbf{β}}_{i}\right)\right) \cos\left({\mathbf{α}}_{i}\right)+\left(-{\mathbf{H}}_{i}+{\mathbf{H}}_{i-1}+{\mathbf{U}_{\text{g},i}} \sin\left({\mathbf{β}}_{i}\right)\right) \sin\left({\mathbf{α}}_{i}\right)+\left(-{\mathbf{G}}_{i}+{\mathbf{G}}_{i-1}\right) \sin\left({\mathbf{α}}_{i}\right)+\left(-{\mathbf{X}}_{i-1}+{\mathbf{X}}_{i}\right) \cos\left({\mathbf{α}}_{i}\right)-{\mathbf{U}_{\text{b},i}}\right) \tan\left(φ'\right)+c' {\mathbf{ℓ}_{b,i}}}{{F_{\text{S}}}}
\end{displaymath}
The definitions of \hyperref[GD:resShearWO]{GD: resShearWO} and \hyperref[GD:mobShearWO]{GD: mobShearWO} are present in this equation, and thus can be replaced by ${\mathbf{R}}_{i}$ and ${\mathbf{T}}_{i}$, respectively:

\begin{displaymath}
{\mathbf{T}}_{i}+\left(-{\mathbf{X}}_{i-1}+{\mathbf{X}}_{i}\right) \sin\left({\mathbf{α}}_{i}\right)-\left(-{\mathbf{G}}_{i}+{\mathbf{G}}_{i-1}\right) \cos\left({\mathbf{α}}_{i}\right)=\frac{{\mathbf{R}}_{i}+\left(\left(-{\mathbf{X}}_{i-1}+{\mathbf{X}}_{i}\right) \cos\left({\mathbf{α}}_{i}\right)+\left(-{\mathbf{G}}_{i}+{\mathbf{G}}_{i-1}\right) \sin\left({\mathbf{α}}_{i}\right)\right) \tan\left(φ'\right)}{{F_{\text{S}}}}
\end{displaymath}
The interslice shear forces $\mathbf{X}$ can be expressed in terms of the interslice normal forces $\mathbf{G}$ using \hyperref[assumpINSFL]{A: Interslice-Norm-Shear-Forces-Linear} and \hyperref[GD:normShrR]{GD: normShrR}, resulting in:

\begin{displaymath}
{\mathbf{T}}_{i}+\left(-λ {\mathbf{f}}_{i-1} {\mathbf{G}}_{i-1}+λ {\mathbf{f}}_{i} {\mathbf{G}}_{i}\right) \sin\left({\mathbf{α}}_{i}\right)-\left(-{\mathbf{G}}_{i}+{\mathbf{G}}_{i-1}\right) \cos\left({\mathbf{α}}_{i}\right)=\frac{{\mathbf{R}}_{i}+\left(\left(-λ {\mathbf{f}}_{i-1} {\mathbf{G}}_{i-1}+λ {\mathbf{f}}_{i} {\mathbf{G}}_{i}\right) \cos\left({\mathbf{α}}_{i}\right)+\left(-{\mathbf{G}}_{i}+{\mathbf{G}}_{i-1}\right) \sin\left({\mathbf{α}}_{i}\right)\right) \tan\left(φ'\right)}{{F_{\text{S}}}}
\end{displaymath}
Rearranging yields the following:

\begin{displaymath}
{\mathbf{G}}_{i} \left(\left(λ {\mathbf{f}}_{i} \cos\left({\mathbf{α}}_{i}\right)-\sin\left({\mathbf{α}}_{i}\right)\right) \tan\left(φ'\right)-\left(λ {\mathbf{f}}_{i} \sin\left({\mathbf{α}}_{i}\right)+\cos\left({\mathbf{α}}_{i}\right)\right) {F_{\text{S}}}\right)={\mathbf{G}}_{i-1} \left(\left(λ {\mathbf{f}}_{i-1} \cos\left({\mathbf{α}}_{i}\right)-\sin\left({\mathbf{α}}_{i}\right)\right) \tan\left(φ'\right)-\left(λ {\mathbf{f}}_{i-1} \sin\left({\mathbf{α}}_{i}\right)+\cos\left({\mathbf{α}}_{i}\right)\right) {F_{\text{S}}}\right)+{F_{\text{S}}} {\mathbf{T}}_{i}-{\mathbf{R}}_{i}
\end{displaymath}
The definitions for $\mathbf{Φ}$ and $\mathbf{Ψ}$ from \hyperref[DD:convertFunc1]{DD: convertFunc1} and \hyperref[DD:convertFunc2]{DD: convertFunc2} simplify the above to Equation (3):

\begin{displaymath}
{\mathbf{G}}_{i} {\mathbf{Φ}}_{i}={\mathbf{Ψ}}_{i-1} {\mathbf{G}}_{i-1} {\mathbf{Φ}}_{i-1}+{F_{\text{S}}} {\mathbf{T}}_{i}-{\mathbf{R}}_{i}
\end{displaymath}
Versions of Equation (3) instantiated for slices 1 to $n$ are shown below:

\begin{displaymath}
{\mathbf{G}}_{1} {\mathbf{Φ}}_{1}={\mathbf{Ψ}}_{0} {\mathbf{G}}_{0} {\mathbf{Φ}}_{0}+{F_{\text{S}}} {\mathbf{T}}_{1}-{\mathbf{R}}_{1}
\end{displaymath}
\begin{displaymath}
{\mathbf{G}}_{2} {\mathbf{Φ}}_{2}={\mathbf{Ψ}}_{1} {\mathbf{G}}_{1} {\mathbf{Φ}}_{1}+{F_{\text{S}}} {\mathbf{T}}_{2}-{\mathbf{R}}_{2}
\end{displaymath}
\begin{displaymath}
{\mathbf{G}}_{3} {\mathbf{Φ}}_{3}={\mathbf{Ψ}}_{2} {\mathbf{G}}_{2} {\mathbf{Φ}}_{2}+{F_{\text{S}}} {\mathbf{T}}_{3}-{\mathbf{R}}_{3}
\end{displaymath}
...

\begin{displaymath}
{\mathbf{G}}_{n-2} {\mathbf{Φ}}_{n-2}={\mathbf{Ψ}}_{n-3} {\mathbf{G}}_{n-3} {\mathbf{Φ}}_{n-3}+{F_{\text{S}}} {\mathbf{T}}_{n-2}-{\mathbf{R}}_{n-2}
\end{displaymath}
\begin{displaymath}
{\mathbf{G}}_{n-1} {\mathbf{Φ}}_{n-1}={\mathbf{Ψ}}_{n-2} {\mathbf{G}}_{n-2} {\mathbf{Φ}}_{n-2}+{F_{\text{S}}} {\mathbf{T}}_{n-1}-{\mathbf{R}}_{n-1}
\end{displaymath}
\begin{displaymath}
{\mathbf{G}}_{n} {\mathbf{Φ}}_{n}={\mathbf{Ψ}}_{n-1} {\mathbf{G}}_{n-1} {\mathbf{Φ}}_{n-1}+{F_{\text{S}}} {\mathbf{T}}_{n}-{\mathbf{R}}_{n}
\end{displaymath}
Applying \hyperref[assumpES]{A: Edge-Slices}, which says that ${\mathbf{G}}_{0}$ and ${\mathbf{G}}_{n}$ are zero, results in the following special cases: Equation (8) for the first slice:

\begin{displaymath}
{\mathbf{G}}_{1} {\mathbf{Φ}}_{1}={F_{\text{S}}} {\mathbf{T}}_{1}-{\mathbf{R}}_{1}
\end{displaymath}
and Equation (9) for the $n$th slice:

\begin{displaymath}
-\left(\frac{{F_{\text{S}}} {\mathbf{T}}_{n}-{\mathbf{R}}_{n}}{{\mathbf{Ψ}}_{n-1}}\right)={\mathbf{G}}_{n-1} {\mathbf{Φ}}_{n-1}
\end{displaymath}
Substituting Equation (8) into Equation (4) yields Equation (10):

\begin{displaymath}
{\mathbf{G}}_{2} {\mathbf{Φ}}_{2}={\mathbf{Ψ}}_{1} \left({F_{\text{S}}} {\mathbf{T}}_{1}-{\mathbf{R}}_{1}\right)+{F_{\text{S}}} {\mathbf{T}}_{2}-{\mathbf{R}}_{2}
\end{displaymath}
which can be substituted into Equation (5) to get Equation (11):

\begin{displaymath}
{\mathbf{G}}_{3} {\mathbf{Φ}}_{3}={\mathbf{Ψ}}_{2} \left({\mathbf{Ψ}}_{1} \left({F_{\text{S}}} {\mathbf{T}}_{1}-{\mathbf{R}}_{1}\right)+{F_{\text{S}}} {\mathbf{T}}_{2}-{\mathbf{R}}_{2}\right)+{F_{\text{S}}} {\mathbf{T}}_{3}-{\mathbf{R}}_{3}
\end{displaymath}
and so on until Equation (12) is obtained from Equation (7):

\begin{displaymath}
{\mathbf{G}}_{n-1} {\mathbf{Φ}}_{n-1}={\mathbf{Ψ}}_{n-2} \left({\mathbf{Ψ}}_{n-3} \left({\mathbf{Ψ}}_{1} \left({F_{\text{S}}} {\mathbf{T}}_{1}-{\mathbf{R}}_{1}\right)+{F_{\text{S}}} {\mathbf{T}}_{2}-{\mathbf{R}}_{2}\right)+{F_{\text{S}}} {\mathbf{T}}_{n-2}-{\mathbf{R}}_{n-2}\right)+{F_{\text{S}}} {\mathbf{T}}_{n-1}-{\mathbf{R}}_{n-1}
\end{displaymath}
Equation (9) can then be substituted into the left hand side of Equation (12), resulting in:

\begin{displaymath}
-\left(\frac{{F_{\text{S}}} {\mathbf{T}}_{n}-{\mathbf{R}}_{n}}{{\mathbf{Ψ}}_{n-1}}\right)={\mathbf{Ψ}}_{n-2} \left({\mathbf{Ψ}}_{n-3} \left({\mathbf{Ψ}}_{1} \left({F_{\text{S}}} {\mathbf{T}}_{1}-{\mathbf{R}}_{1}\right)+{F_{\text{S}}} {\mathbf{T}}_{2}-{\mathbf{R}}_{2}\right)+{F_{\text{S}}} {\mathbf{T}}_{n-2}-{\mathbf{R}}_{n-2}\right)+{F_{\text{S}}} {\mathbf{T}}_{n-1}-{\mathbf{R}}_{n-1}
\end{displaymath}
This can be rearranged by multiplying both sides by ${\mathbf{Ψ}}_{n-1}$ and then distributing the multiplication of each $\mathbf{Ψ}$ over addition to obtain:

\begin{displaymath}
-\left({F_{\text{S}}} {\mathbf{T}}_{n}-{\mathbf{R}}_{n}\right)={\mathbf{Ψ}}_{n-1} {\mathbf{Ψ}}_{n-2} {\mathbf{Ψ}}_{1} \left({F_{\text{S}}} {\mathbf{T}}_{1}-{\mathbf{R}}_{1}\right)+{\mathbf{Ψ}}_{n-1} {\mathbf{Ψ}}_{n-2} {\mathbf{Ψ}}_{2} \left({F_{\text{S}}} {\mathbf{T}}_{2}-{\mathbf{R}}_{2}\right)+{\mathbf{Ψ}}_{n-1} \left({F_{\text{S}}} {\mathbf{T}}_{n-1}-{\mathbf{R}}_{n-1}\right)
\end{displaymath}
The multiplication of the $\mathbf{Ψ}$ terms can be further distributed over the subtractions, resulting in the equation having terms that each either contain an $\mathbf{R}$ or a $\mathbf{T}$. The equation can then be rearranged so terms containing an $\mathbf{R}$ are on one side of the equality, and terms containing a $\mathbf{T}$ are on the other. The multiplication by the factor of safety is common to all of the $\mathbf{T}$ terms, and thus can be factored out, resulting in:

\begin{displaymath}
{F_{\text{S}}} \left({\mathbf{Ψ}}_{n-1} {\mathbf{Ψ}}_{n-2} {\mathbf{Ψ}}_{1} {\mathbf{T}}_{1}+{\mathbf{Ψ}}_{n-1} {\mathbf{Ψ}}_{n-2} {\mathbf{Ψ}}_{2} {\mathbf{T}}_{2}+{\mathbf{Ψ}}_{n-1} {\mathbf{T}}_{n-1}+{\mathbf{T}}_{n}\right)={\mathbf{Ψ}}_{n-1} {\mathbf{Ψ}}_{n-2} {\mathbf{Ψ}}_{1} {\mathbf{R}}_{1}+{\mathbf{Ψ}}_{n-1} {\mathbf{Ψ}}_{n-2} {\mathbf{Ψ}}_{2} {\mathbf{R}}_{2}+{\mathbf{Ψ}}_{n-1} {\mathbf{R}}_{n-1}+{\mathbf{R}}_{n}
\end{displaymath}
Isolating the factor of safety on the left-hand side and using compact notation for the products and sums yields Equation (13), which can also be seen in \hyperref[IM:fctSfty]{IM: fctSfty}:

\begin{displaymath}
{F_{\text{S}}}=\frac{\displaystyle\sum_{i=1}^{n-1}{{\mathbf{R}}_{i} \displaystyle\prod_{v=i}^{n-1}{{\mathbf{Ψ}}_{v}}}+{\mathbf{R}}_{n}}{\displaystyle\sum_{i=1}^{n-1}{{\mathbf{T}}_{i} \displaystyle\prod_{v=i}^{n-1}{{\mathbf{Ψ}}_{v}}}+{\mathbf{T}}_{n}}
\end{displaymath}
${F_{\text{S}}}$ depends on the unknowns $λ$ (\hyperref[IM:nrmShrFor]{IM: nrmShrFor}) and $\mathbf{G}$ (\hyperref[IM:intsliceFs]{IM: intsliceFs}).

\vspace{\baselineskip}
\noindent
\begin{minipage}{\textwidth}
\begin{tabular}{>{\raggedright}p{0.13\textwidth}>{\raggedright\arraybackslash}p{0.82\textwidth}}
\toprule \textbf{Refname} & \textbf{IM:nrmShrFor}
\phantomsection 
\label{IM:nrmShrFor}
\\ \midrule \\
Label & Normal and shear force proportionality constant
        
\\ \midrule \\
Input & ${\mathbf{x}_{\text{slope}}}$, ${\mathbf{y}_{\text{slope}}}$, ${\mathbf{y}_{\text{wt}}}$, ${γ_{w}}$, ${\mathbf{x}_{\text{slip}}}$, ${\mathbf{y}_{\text{slip}}}$, $const_f$
        
\\ \midrule \\
Output & $λ$
         
\\ \midrule \\
Input Constraints & 
\\ \midrule \\
Output Constraints & 
\\ \midrule \\
Equation & \begin{displaymath}
           λ=\frac{\displaystyle\sum_{i=1}^{n}{{\mathbf{C}_{\text{num},i}}}}{\displaystyle\sum_{i=1}^{n}{{\mathbf{C}_{\text{den},i}}}}
           \end{displaymath}
\\ \midrule \\
Description & \begin{symbDescription}
              \item{$λ$ is the proportionality constant (Unitless)}
              \item{${\mathbf{C}_{\text{num}}}$ is the proportionality constant numerator (${\text{N}}$)}
              \item{$i$ is the index (Unitless)}
              \item{${\mathbf{C}_{\text{den}}}$ is the proportionality constant denominator (${\text{N}}$)}
              \end{symbDescription}
\\ \midrule \\
Notes & ${\mathbf{C}_{\text{num}}}$ is defined in \hyperref[IM:nrmShrForNum]{IM: nrmShrForNum} and ${\mathbf{C}_{\text{den}}}$ is defined in \hyperref[IM:nrmShrForDen]{IM: nrmShrForDen}.
        
\\ \midrule \\
Source & \cite{chen2005} and \cite{karchewski2012}
         
\\ \midrule \\
RefBy & \hyperref[IM:nrmShrFor]{IM: nrmShrFor}, \hyperref[IM:nrmShrForNum]{IM: nrmShrForNum}, \hyperref[IM:nrmShrForDen]{IM: nrmShrForDen}, \hyperref[IM:intsliceFs]{IM: intsliceFs}, \hyperref[IM:fctSfty]{IM: fctSfty}, \hyperref[displayShear]{FR: Display-Interslice-Shear-Forces}, \hyperref[displayNormal]{FR: Display-Interslice-Normal-Forces}, \hyperref[displayFS]{FR: Display-Factor-of-Safety}, and \hyperref[determineCritSlip]{FR: Determine-Critical-Slip-Surface}
        
\\ \bottomrule
\end{tabular}
\end{minipage}
\paragraph{}
\label{IM:nrmShrForDeriv}
From the moment equilibrium of \hyperref[GD:momentEql]{GD: momentEql} with the primary assumption for the Morgenstern-Price method of \hyperref[assumpINSFL]{A: Interslice-Norm-Shear-Forces-Linear} and associated definition \hyperref[GD:normShrR]{GD: normShrR}, Equation (14) can be derived:

\begin{displaymath}
0=-{\mathbf{G}}_{i} \left({\mathbf{h}_{\text{z},i}}+\frac{{\mathbf{b}}_{i}}{2} \tan\left({\mathbf{α}}_{i}\right)\right)+{\mathbf{G}}_{i-1} \left({\mathbf{h}_{\text{z},i-1}}-\frac{{\mathbf{b}}_{i}}{2} \tan\left({\mathbf{α}}_{i}\right)\right)-{\mathbf{H}}_{i} \left(\frac{1}{3} {\mathbf{h}_{\text{z,w},i}}+\frac{{\mathbf{b}}_{i}}{2} \tan\left({\mathbf{α}}_{i}\right)\right)+{\mathbf{H}}_{i-1} \left(\frac{1}{3} {\mathbf{h}_{\text{z,w},i-1}}-\frac{{\mathbf{b}}_{i}}{2} \tan\left({\mathbf{α}}_{i}\right)\right)+λ \frac{{\mathbf{b}}_{i}}{2} \left({\mathbf{G}}_{i} {\mathbf{f}}_{i}+{\mathbf{G}}_{i-1} {\mathbf{f}}_{i-1}\right)+\frac{-{K_{\text{c}}} {\mathbf{W}}_{i} {\mathbf{h}}_{i}}{2}+{\mathbf{U}_{\text{g},i}} \sin\left({\mathbf{β}}_{i}\right) {\mathbf{h}}_{i}+{\mathbf{Q}}_{i} \sin\left({\mathbf{ω}}_{i}\right) {\mathbf{h}}_{i}
\end{displaymath}
Rearranging the equation in terms of $λ$ leads to Equation (15):

\begin{displaymath}
λ=\frac{-{\mathbf{G}}_{i} \left({\mathbf{h}_{\text{z},i}}+\frac{{\mathbf{b}}_{i}}{2} \tan\left({\mathbf{α}}_{i}\right)\right)+{\mathbf{G}}_{i-1} \left({\mathbf{h}_{\text{z},i-1}}-\frac{{\mathbf{b}}_{i}}{2} \tan\left({\mathbf{α}}_{i}\right)\right)-{\mathbf{H}}_{i} \left(\frac{1}{3} {\mathbf{h}_{\text{z,w},i}}+\frac{{\mathbf{b}}_{i}}{2} \tan\left({\mathbf{α}}_{i}\right)\right)+{\mathbf{H}}_{i-1} \left(\frac{1}{3} {\mathbf{h}_{\text{z,w},i-1}}-\frac{{\mathbf{b}}_{i}}{2} \tan\left({\mathbf{α}}_{i}\right)\right)+\frac{-{K_{\text{c}}} {\mathbf{W}}_{i} {\mathbf{h}}_{i}}{2}+{\mathbf{U}_{\text{g},i}} \sin\left({\mathbf{β}}_{i}\right) {\mathbf{h}}_{i}+{\mathbf{Q}}_{i} \sin\left({\mathbf{ω}}_{i}\right) {\mathbf{h}}_{i}}{-\frac{{\mathbf{b}}_{i}}{2} \left({\mathbf{G}}_{i} {\mathbf{f}}_{i}+{\mathbf{G}}_{i-1} {\mathbf{f}}_{i-1}\right)}
\end{displaymath}
This equation can be simplified by applying assumptions \hyperref[assumpSF]{A: Seismic-Force} and \hyperref[assumpSL]{A: Surface-Load}, which state that the seismic and external forces, respectively, are zero:

\begin{displaymath}
λ=\frac{-{\mathbf{G}}_{i} \left({\mathbf{h}_{\text{z},i}}+\frac{{\mathbf{b}}_{i}}{2} \tan\left({\mathbf{α}}_{i}\right)\right)+{\mathbf{G}}_{i-1} \left({\mathbf{h}_{\text{z},i-1}}-\frac{{\mathbf{b}}_{i}}{2} \tan\left({\mathbf{α}}_{i}\right)\right)-{\mathbf{H}}_{i} \left(\frac{1}{3} {\mathbf{h}_{\text{z,w},i}}+\frac{{\mathbf{b}}_{i}}{2} \tan\left({\mathbf{α}}_{i}\right)\right)+{\mathbf{H}}_{i-1} \left(\frac{1}{3} {\mathbf{h}_{\text{z,w},i-1}}-\frac{{\mathbf{b}}_{i}}{2} \tan\left({\mathbf{α}}_{i}\right)\right)+{\mathbf{U}_{\text{g},i}} \sin\left({\mathbf{β}}_{i}\right) {\mathbf{h}}_{i}}{-\frac{{\mathbf{b}}_{i}}{2} \left({\mathbf{G}}_{i} {\mathbf{f}}_{i}+{\mathbf{G}}_{i-1} {\mathbf{f}}_{i-1}\right)}
\end{displaymath}
Taking the summation of all slices, and applying \hyperref[assumpES]{A: Edge-Slices} to set ${\mathbf{G}}_{0}$, ${\mathbf{G}}_{n}$, ${\mathbf{H}}_{0}$, and ${\mathbf{H}}_{n}$ equal to zero, a general equation for the proportionality constant $λ$ is developed in Equation (16), which combines \hyperref[IM:nrmShrFor]{IM: nrmShrFor}, \hyperref[IM:nrmShrForNum]{IM: nrmShrForNum}, and \hyperref[IM:nrmShrForDen]{IM: nrmShrForDen}:

\begin{displaymath}
λ=\frac{\displaystyle\sum_{i=1}^{n}{{\mathbf{b}}_{i} \left({{\mathbf{F}_{\text{x}}}^{\text{G}}}+{{\mathbf{F}_{\text{x}}}^{\text{H}}}\right) \tan\left({\mathbf{α}}_{i}\right)+{\mathbf{h}}_{i} -2 {\mathbf{U}_{\text{g},i}} \sin\left({\mathbf{β}}_{i}\right)}}{\displaystyle\sum_{i=1}^{n}{{\mathbf{b}}_{i} \left({\mathbf{G}}_{i} {\mathbf{f}}_{i}+{\mathbf{G}}_{i-1} {\mathbf{f}}_{i-1}\right)}}
\end{displaymath}
Equation (16) for $λ$ is a function of the unknown interslice normal forces $\mathbf{G}$ (\hyperref[IM:intsliceFs]{IM: intsliceFs}) which itself depends on the unknown factor of safety ${F_{\text{S}}}$ (\hyperref[IM:fctSfty]{IM: fctSfty}).

\vspace{\baselineskip}
\noindent
\begin{minipage}{\textwidth}
\begin{tabular}{>{\raggedright}p{0.13\textwidth}>{\raggedright\arraybackslash}p{0.82\textwidth}}
\toprule \textbf{Refname} & \textbf{IM:nrmShrForNum}
\phantomsection 
\label{IM:nrmShrForNum}
\\ \midrule \\
Label & Normal and shear force proportionality constant numerator
        
\\ \midrule \\
Input & ${\mathbf{x}_{\text{slope}}}$, ${\mathbf{y}_{\text{slope}}}$, ${\mathbf{y}_{\text{wt}}}$, ${γ_{w}}$, ${\mathbf{x}_{\text{slip}}}$, ${\mathbf{y}_{\text{slip}}}$
        
\\ \midrule \\
Output & ${\mathbf{C}_{\text{num}}}$
         
\\ \midrule \\
Input Constraints & 
\\ \midrule \\
Output Constraints & 
\\ \midrule \\
Equation & \begin{displaymath}
           {\mathbf{C}_{\text{num},i}}=\begin{cases}
                                       {\mathbf{b}}_{1} \left({\mathbf{G}}_{1}+{\mathbf{H}}_{1}\right) \tan\left({\mathbf{α}}_{1}\right), & i=1\\
                                       {\mathbf{b}}_{i} \left({{\mathbf{F}_{\text{x}}}^{\text{G}}}+{{\mathbf{F}_{\text{x}}}^{\text{H}}}\right) \tan\left({\mathbf{α}}_{i}\right)+\mathbf{h} -2 {\mathbf{U}_{\text{g},i}} \sin\left({\mathbf{β}}_{i}\right), & 2\leq{}i\leq{}n-1\\
                                       {\mathbf{b}}_{n} \left({\mathbf{G}}_{n-1}+{\mathbf{H}}_{n-1}\right) \tan\left({\mathbf{α}}_{n-1}\right), & i=n
                                       \end{cases}
           \end{displaymath}
\\ \midrule \\
Description & \begin{symbDescription}
              \item{${\mathbf{C}_{\text{num}}}$ is the proportionality constant numerator (${\text{N}}$)}
              \item{$i$ is the index (Unitless)}
              \item{$\mathbf{b}$ is the base width of slices (${\text{m}}$)}
              \item{$\mathbf{G}$ is the interslice normal forces ($\frac{\text{N}}{\text{m}}$)}
              \item{$\mathbf{H}$ is the interslice normal water forces ($\frac{\text{N}}{\text{m}}$)}
              \item{$\mathbf{α}$ is the base angles (${{}^{\circ}}$)}
              \item{${{\mathbf{F}_{\text{x}}}^{\text{G}}}$ is the sums of the interslice normal forces (${\text{N}}$)}
              \item{${{\mathbf{F}_{\text{x}}}^{\text{H}}}$ is the sums of the interslice normal water forces (${\text{N}}$)}
              \item{$\mathbf{h}$ is the $y$-direction heights of slices (${\text{m}}$)}
              \item{${\mathbf{U}_{\text{g}}}$ is the surface hydrostatic forces ($\frac{\text{N}}{\text{m}}$)}
              \item{$\mathbf{β}$ is the surface angles (${{}^{\circ}}$)}
              \item{$n$ is the number of slices (Unitless)}
              \end{symbDescription}
\\ \midrule \\
Notes & $\mathbf{b}$ is defined in \hyperref[DD:lengthB]{DD: lengthB}, $\mathbf{H}$ is defined in \hyperref[DD:intersliceWtrF]{DD: intersliceWtrF}, $\mathbf{α}$ is defined in \hyperref[DD:angleA]{DD: angleA}, $\mathbf{h}$ is defined in \hyperref[DD:slcHeight]{DD: slcHeight}, ${\mathbf{U}_{\text{g}}}$ is defined in \hyperref[GD:srfWtrF]{GD: srfWtrF}, and $\mathbf{β}$ is defined in \hyperref[DD:angleB]{DD: angleB}.
        
\\ \midrule \\
Source & \cite{chen2005} and \cite{karchewski2012}
         
\\ \midrule \\
RefBy & \hyperref[IM:nrmShrFor]{IM: nrmShrFor}
        
\\ \bottomrule
\end{tabular}
\end{minipage}
\paragraph{}
\label{IM:nrmShrForNumDeriv}
See \hyperref[IM:nrmShrFor]{IM: nrmShrFor} for the derivation of ${\mathbf{C}_{\text{num}}}$.

\vspace{\baselineskip}
\noindent
\begin{minipage}{\textwidth}
\begin{tabular}{>{\raggedright}p{0.13\textwidth}>{\raggedright\arraybackslash}p{0.82\textwidth}}
\toprule \textbf{Refname} & \textbf{IM:nrmShrForDen}
\phantomsection 
\label{IM:nrmShrForDen}
\\ \midrule \\
Label & Normal and shear force proportionality constant denominator
        
\\ \midrule \\
Input & ${\mathbf{x}_{\text{slip}}}$, $const_f$
        
\\ \midrule \\
Output & ${\mathbf{C}_{\text{den}}}$
         
\\ \midrule \\
Input Constraints & 
\\ \midrule \\
Output Constraints & 
\\ \midrule \\
Equation & \begin{displaymath}
           {\mathbf{C}_{\text{den},i}}=\begin{cases}
                                       {\mathbf{b}}_{1} {\mathbf{f}}_{1} {\mathbf{G}}_{1}, & i=1\\
                                       {\mathbf{b}}_{i} \left({\mathbf{f}}_{i} {\mathbf{G}}_{i}+{\mathbf{f}}_{i-1} {\mathbf{G}}_{i-1}\right), & 2\leq{}i\leq{}n-1\\
                                       {\mathbf{b}}_{n} {\mathbf{G}}_{n-1} {\mathbf{f}}_{n-1}, & i=n
                                       \end{cases}
           \end{displaymath}
\\ \midrule \\
Description & \begin{symbDescription}
              \item{${\mathbf{C}_{\text{den}}}$ is the proportionality constant denominator (${\text{N}}$)}
              \item{$i$ is the index (Unitless)}
              \item{$\mathbf{b}$ is the base width of slices (${\text{m}}$)}
              \item{$\mathbf{f}$ is the interslice normal to shear force ratio variation function (Unitless)}
              \item{$\mathbf{G}$ is the interslice normal forces ($\frac{\text{N}}{\text{m}}$)}
              \item{$n$ is the number of slices (Unitless)}
              \end{symbDescription}
\\ \midrule \\
Notes & $\mathbf{b}$ is defined in \hyperref[DD:lengthB]{DD: lengthB} and $\mathbf{f}$ is defined in \hyperref[DD:ratioVariation]{DD: ratioVariation}.
        
\\ \midrule \\
Source & \cite{chen2005} and \cite{karchewski2012}
         
\\ \midrule \\
RefBy & \hyperref[IM:nrmShrFor]{IM: nrmShrFor}
        
\\ \bottomrule
\end{tabular}
\end{minipage}
\paragraph{}
\label{IM:nrmShrForDenDeriv}
See \hyperref[IM:nrmShrFor]{IM: nrmShrFor} for the derivation of ${\mathbf{C}_{\text{den}}}$.

\vspace{\baselineskip}
\noindent
\begin{minipage}{\textwidth}
\begin{tabular}{>{\raggedright}p{0.13\textwidth}>{\raggedright\arraybackslash}p{0.82\textwidth}}
\toprule \textbf{Refname} & \textbf{IM:intsliceFs}
\phantomsection 
\label{IM:intsliceFs}
\\ \midrule \\
Label & Interslice normal forces
        
\\ \midrule \\
Input & ${\mathbf{x}_{\text{slope}}}$, ${\mathbf{y}_{\text{slope}}}$, ${\mathbf{y}_{\text{wt}}}$, $c'$, $φ'$, ${γ_{\text{dry}}}$, ${γ_{\text{sat}}}$, ${γ_{w}}$, ${\mathbf{x}_{\text{slip}}}$, ${\mathbf{y}_{\text{slip}}}$, $const_f$
        
\\ \midrule \\
Output & $\mathbf{G}$
         
\\ \midrule \\
Input Constraints & 
\\ \midrule \\
Output Constraints & 
\\ \midrule \\
Equation & \begin{displaymath}
           {\mathbf{G}}_{i}=\begin{cases}
                            \frac{{F_{\text{S}}} {\mathbf{T}}_{1}-{\mathbf{R}}_{1}}{{\mathbf{Φ}}_{1}}, & i=1\\
                            \frac{{\mathbf{Ψ}}_{i-1} {\mathbf{G}}_{i-1}+{F_{\text{S}}} {\mathbf{T}}_{i}-{\mathbf{R}}_{i}}{{\mathbf{Φ}}_{i}}, & 2\leq{}i\leq{}n-1\\
                            0, & i=0\lor{}i=n
                            \end{cases}
           \end{displaymath}
\\ \midrule \\
Description & \begin{symbDescription}
              \item{$\mathbf{G}$ is the interslice normal forces ($\frac{\text{N}}{\text{m}}$)}
              \item{$i$ is the index (Unitless)}
              \item{${F_{\text{S}}}$ is the factor of safety (Unitless)}
              \item{$\mathbf{T}$ is the mobilized shear forces without the influence of interslice forces ($\frac{\text{N}}{\text{m}}$)}
              \item{$\mathbf{R}$ is the resistive shear forces without the influence of interslice forces ($\frac{\text{N}}{\text{m}}$)}
              \item{$\mathbf{Φ}$ is the first function for incorporating interslice forces into shear force (Unitless)}
              \item{$\mathbf{Ψ}$ is the second function for incorporating interslice forces into shear force (Unitless)}
              \item{$n$ is the number of slices (Unitless)}
              \end{symbDescription}
\\ \midrule \\
Notes & $\mathbf{T}$ is defined in \hyperref[GD:mobShearWO]{GD: mobShearWO}, $\mathbf{R}$ is defined in \hyperref[GD:resShearWO]{GD: resShearWO}, $\mathbf{Φ}$ is defined in \hyperref[DD:convertFunc1]{DD: convertFunc1}, and $\mathbf{Ψ}$ is defined in \hyperref[DD:convertFunc2]{DD: convertFunc2}.
        
\\ \midrule \\
Source & \cite{chen2005}
         
\\ \midrule \\
RefBy & \hyperref[IM:nrmShrFor]{IM: nrmShrFor}, \hyperref[IM:intsliceFs]{IM: intsliceFs}, \hyperref[IM:fctSfty]{IM: fctSfty}, \hyperref[displayShear]{FR: Display-Interslice-Shear-Forces}, \hyperref[displayNormal]{FR: Display-Interslice-Normal-Forces}, \hyperref[displayFS]{FR: Display-Factor-of-Safety}, and \hyperref[determineCritSlip]{FR: Determine-Critical-Slip-Surface}
        
\\ \bottomrule
\end{tabular}
\end{minipage}
\paragraph{}
\label{IM:intsliceFsDeriv}
This derivation is identical to the derivation for \hyperref[IM:fctSfty]{IM: fctSfty} up until Equation (3) shown again below:

\begin{displaymath}
{\mathbf{G}}_{i} {\mathbf{Φ}}_{i}={\mathbf{Ψ}}_{i-1} {\mathbf{G}}_{i-1} {\mathbf{Φ}}_{i-1}+{F_{\text{S}}} {\mathbf{T}}_{i}-{\mathbf{R}}_{i}
\end{displaymath}
A simple rearrangement of Equation (3) leads to Equation (17), also seen in \hyperref[IM:intsliceFs]{IM: intsliceFs}:

\begin{displaymath}
{\mathbf{G}}_{i}=\frac{{\mathbf{Ψ}}_{i-1} {\mathbf{G}}_{i-1}+{F_{\text{S}}} {\mathbf{T}}_{i}-{\mathbf{R}}_{i}}{{\mathbf{Φ}}_{i}}
\end{displaymath}
The cases shown in \hyperref[IM:intsliceFs]{IM: intsliceFs} for when $i=0$, $i=1$, or $i=n$ are derived by applying \hyperref[assumpES]{A: Edge-Slices}, which says that ${\mathbf{G}}_{0}$ and ${\mathbf{G}}_{n}$ are zero, to Equation (17). $\mathbf{G}$ depends on the unknowns ${F_{\text{S}}}$ (\hyperref[IM:fctSfty]{IM: fctSfty}) and $λ$ (\hyperref[IM:nrmShrFor]{IM: nrmShrFor}).

\vspace{\baselineskip}
\noindent
\begin{minipage}{\textwidth}
\begin{tabular}{>{\raggedright}p{0.13\textwidth}>{\raggedright\arraybackslash}p{0.82\textwidth}}
\toprule \textbf{Refname} & \textbf{IM:crtSlpId}
\phantomsection 
\label{IM:crtSlpId}
\\ \midrule \\
Label & Critical slip surface identification
        
\\ \midrule \\
Input & ${\mathbf{x}_{\text{slope}}}$, ${\mathbf{y}_{\text{slope}}}$, ${\mathbf{x}_{\text{wt}}}$, ${\mathbf{y}_{\text{wt}}}$, $c'$, $φ'$, ${γ_{\text{dry}}}$, ${γ_{\text{sat}}}$, ${γ_{w}}$, $const_f$
        
\\ \midrule \\
Output & ${{F_{\text{S}}}^{\text{min}}}$
         
\\ \midrule \\
Input Constraints & 
\\ \midrule \\
Output Constraints & 
\\ \midrule \\
Equation & \begin{displaymath}
           {{F_{\text{S}}}^{\text{min}}}=Υ\left({\mathbf{x}_{\text{slope}}},{\mathbf{y}_{\text{slope}}},{\mathbf{x}_{\text{wt}}},{\mathbf{y}_{\text{wt}}},c',φ',{γ_{\text{dry}}},{γ_{\text{sat}}},{γ_{w}},const_f\right)
           \end{displaymath}
\\ \midrule \\
Description & \begin{symbDescription}
              \item{${{F_{\text{S}}}^{\text{min}}}$ is the minimum factor of safety (Unitless)}
              \item{$Υ$ is the minimization function (Unitless)}
              \item{${\mathbf{x}_{\text{slope}}}$ is the $x$-coordinates of the slope (${\text{m}}$)}
              \item{${\mathbf{y}_{\text{slope}}}$ is the $y$-coordinates of the slope (${\text{m}}$)}
              \item{${\mathbf{x}_{\text{wt}}}$ is the $x$-coordinates of the water table (${\text{m}}$)}
              \item{${\mathbf{y}_{\text{wt}}}$ is the $y$-coordinates of the water table (${\text{m}}$)}
              \item{$c'$ is the effective cohesion (${\text{Pa}}$)}
              \item{$φ'$ is the effective angle of friction (${{}^{\circ}}$)}
              \item{${γ_{\text{dry}}}$ is the soil dry unit weight ($\frac{\text{N}}{\text{m}^{3}}$)}
              \item{${γ_{\text{sat}}}$ is the soil saturated unit weight ($\frac{\text{N}}{\text{m}^{3}}$)}
              \item{${γ_{w}}$ is the unit weight of water ($\frac{\text{N}}{\text{m}^{3}}$)}
              \item{$const_f$ is the decision on f (Unitless)}
              \end{symbDescription}
\\ \midrule \\
Notes & The minimization function must enforce the constraints on the critical slip surface expressed in \hyperref[assumpSSC]{A: Slip-Surface-Concave} and \hyperref[Sec:CorSolProps]{Section: Properties of a Correct Solution}. The sizes of ${\mathbf{x}_{\text{wt}}}$ and ${\mathbf{y}_{\text{wt}}}$ must be equal and not 1. The sizes of ${\mathbf{x}_{\text{slope}}}$ and ${\mathbf{y}_{\text{slope}}}$ must be equal and at least 2. The first and last ${\mathbf{x}_{\text{wt}}}$ values must be equal to the first and last ${\mathbf{x}_{\text{slope}}}$ values. ${\mathbf{x}_{\text{wt}}}$ and ${\mathbf{x}_{\text{slope}}}$ values must be monotonically increasing. ${{x_{\text{slip}}}^{\text{maxExt}}}$, ${{x_{\text{slip}}}^{\text{maxEtr}}}$, ${{x_{\text{slip}}}^{\text{minExt}}}$, and ${{x_{\text{slip}}}^{\text{minEtr}}}$ must be between or equal to the minimum and maximum ${\mathbf{x}_{\text{slope}}}$ values. ${{y_{\text{slip}}}^{\text{max}}}$ cannot be below the minimum ${\mathbf{y}_{\text{slope}}}$ value. ${{y_{\text{slip}}}^{\text{min}}}$ cannot be above the maximum ${\mathbf{y}_{\text{slope}}}$ value. All $x$ values of ${\mathbf{x}_{\text{cs}}}\text{,}{\mathbf{y}_{\text{cs}}}$ must be between ${{x_{\text{slip}}}^{\text{minEtr}}}$ and ${{x_{\text{slip}}}^{\text{maxExt}}}$. All $y$ values of ${\mathbf{x}_{\text{cs}}}\text{,}{\mathbf{y}_{\text{cs}}}$ must not be below ${{y_{\text{slip}}}^{\text{min}}}$. For any given vertex in ${\mathbf{x}_{\text{cs}}}\text{,}{\mathbf{y}_{\text{cs}}}$ the $y$ value must not exceed the ${\mathbf{y}_{\text{slope}}}$ value corresponding to the same $x$ value. The first and last vertices in ${\mathbf{x}_{\text{cs}}}\text{,}{\mathbf{y}_{\text{cs}}}$ must each be equal to one of the vertices formed by ${\mathbf{x}_{\text{slope}}}$ and ${\mathbf{y}_{\text{slope}}}$. The slope between consecutive vertices must be always increasing as $x$ increases. The internal angle between consecutive vertices in ${\mathbf{x}_{\text{cs}}}\text{,}{\mathbf{y}_{\text{cs}}}$ must not be below 110 degrees.
        
\\ \midrule \\
Source & \cite{li2010}
         
\\ \midrule \\
RefBy & \hyperref[displayGraph]{FR: Display-Graph} and \hyperref[determineCritSlip]{FR: Determine-Critical-Slip-Surface}
        
\\ \bottomrule
\end{tabular}
\end{minipage}

\subsubsection{Data Constraints}
\label{Sec:DataConstraints}
\hyperref[Table:InDataConstraints]{Table:InDataConstraints} shows the data constraints on the input variables. The column for physical constraints gives the physical limitations on the range of values that can be taken by the variable. The uncertainty column provides an estimate of the confidence with which the physical quantities can be measured. This information would be part of the input if one were performing an uncertainty quantification exercise. The constraints are conservative, to give the user of the model the flexibility to experiment with unusual situations. The column of typical values is intended to provide a feel for a common scenario.

\begin{longtable}{l l l l}
\toprule
\textbf{Var} & \textbf{Physical Constraints} & \textbf{Typical Value} & \textbf{Uncert.}
\\
\midrule
\endhead
$c'$ & $c'\gt{}0$ & $10.0\cdot{}10^{3}$ ${\text{Pa}}$ & 10$\%$
\\
${{x_{\text{slip}}}^{\text{maxEtr}}}$ & -- & $20.0$ ${\text{m}}$ & 10$\%$
\\
${{x_{\text{slip}}}^{\text{maxExt}}}$ & -- & $100.0$ ${\text{m}}$ & 10$\%$
\\
${{x_{\text{slip}}}^{\text{minEtr}}}$ & -- & $0.0$ ${\text{m}}$ & 10$\%$
\\
${{x_{\text{slip}}}^{\text{minExt}}}$ & -- & $50.0$ ${\text{m}}$ & 10$\%$
\\
${\mathbf{x}_{\text{slope}}}$ & -- & $0.0$ ${\text{m}}$ & 10$\%$
\\
${\mathbf{x}_{\text{wt}}}$ & -- & $0.0$ ${\text{m}}$ & 10$\%$
\\
${{y_{\text{slip}}}^{\text{max}}}$ & -- & $30.0$ ${\text{m}}$ & 10$\%$
\\
${{y_{\text{slip}}}^{\text{min}}}$ & -- & $0.0$ ${\text{m}}$ & 10$\%$
\\
${\mathbf{y}_{\text{slope}}}$ & -- & $0.0$ ${\text{m}}$ & 10$\%$
\\
${\mathbf{y}_{\text{wt}}}$ & -- & $0.0$ ${\text{m}}$ & 10$\%$
\\
${γ_{\text{dry}}}$ & ${γ_{\text{dry}}}\gt{}0$ & $20.0\cdot{}10^{3}$ $\frac{\text{N}}{\text{m}^{3}}$ & 10$\%$
\\
${γ_{\text{sat}}}$ & ${γ_{\text{sat}}}\gt{}0$ & $20.0\cdot{}10^{3}$ $\frac{\text{N}}{\text{m}^{3}}$ & 10$\%$
\\
${γ_{w}}$ & ${γ_{w}}\gt{}0$ & $9.8\cdot{}10^{3}$ $\frac{\text{N}}{\text{m}^{3}}$ & 10$\%$
\\
$φ'$ & $0\lt{}φ'\lt{}90$ & $25.0$ ${{}^{\circ}}$ & 10$\%$
\\
\bottomrule
\caption{Input Data Constraints}
\label{Table:InDataConstraints}
\end{longtable}
\subsubsection{Properties of a Correct Solution}
\label{Sec:CorSolProps}
\hyperref[Table:OutDataConstraints]{Table:OutDataConstraints} shows the data constraints on the output variables. The column for physical constraints gives the physical limitations on the range of values that can be taken by the variable.

\begin{longtable}{l l}
\toprule
\textbf{Var} & \textbf{Physical Constraints}
\\
\midrule
\endhead
${F_{\text{S}}}$ & ${F_{\text{S}}}\gt{}0$
\\
$\text{(x,y)}$ & --
\\
\bottomrule
\caption{Output Data Constraints}
\label{Table:OutDataConstraints}
\end{longtable}
\section{Requirements}
\label{Sec:Requirements}
This section provides the functional requirements, the tasks and behaviours that the software is expected to complete, and the non-functional requirements, the qualities that the software is expected to exhibit.

\subsection{Functional Requirements}
\label{Sec:FRs}
This section provides the functional requirements, the tasks and behaviours that the software is expected to complete.

\begin{itemize}
\item[Read-and-Store:\phantomsection\label{readAndStore}]{Read the inputs, shown in \hyperref[Table:ReqInputs]{Table:ReqInputs}, and store the data.}
\item[Verify-Input:\phantomsection\label{verifyInput}]{Verify that the input data lie within the physical constraints shown in \hyperref[Sec:DataConstraints]{Section: Data Constraints}.}
\item[Determine-Critical-Slip-Surface:\phantomsection\label{determineCritSlip}]{Determine the critical slip surface for the input slope, corresponding to the minimum factor of safety, by using \hyperref[IM:fctSfty]{IM: fctSfty}, \hyperref[IM:nrmShrFor]{IM: nrmShrFor}, and \hyperref[IM:intsliceFs]{IM: intsliceFs} to calculate the factor of safety for a slip surface and using \hyperref[IM:crtSlpId]{IM: crtSlpId} to find the slip surface that minimizes it.}
\item[Verify-Output:\phantomsection\label{verifyOutput}]{Verify that the minimum factor of safety and critical slip surface satisfy the physical constraints shown in \hyperref[Sec:CorSolProps]{Section: Properties of a Correct Solution}.}
\item[Display-Input:\phantomsection\label{displayInput}]{Display as output the user-supplied inputs listed in \hyperref[Table:inputsToOutputTable]{Table:inputsToOutputTable}.}
\item[Display-Graph:\phantomsection\label{displayGraph}]{Display the critical slip surface of the 2D slope, as determined from \hyperref[IM:crtSlpId]{IM: crtSlpId}, graphically.}
\item[Display-Factor-of-Safety:\phantomsection\label{displayFS}]{Display the value of the factor of safety for the critical slip surface, as determined from \hyperref[IM:fctSfty]{IM: fctSfty}, \hyperref[IM:nrmShrFor]{IM: nrmShrFor}, and \hyperref[IM:intsliceFs]{IM: intsliceFs}.}
\item[Display-Interslice-Normal-Forces:\phantomsection\label{displayNormal}]{Using \hyperref[IM:fctSfty]{IM: fctSfty}, \hyperref[IM:nrmShrFor]{IM: nrmShrFor}, and \hyperref[IM:intsliceFs]{IM: intsliceFs}, calculate and graphically display the interslice normal forces.}
\item[Display-Interslice-Shear-Forces:\phantomsection\label{displayShear}]{Using \hyperref[IM:fctSfty]{IM: fctSfty}, \hyperref[IM:nrmShrFor]{IM: nrmShrFor}, and \hyperref[IM:intsliceFs]{IM: intsliceFs}, calculate and graphically display the interslice shear forces.}
\item[Write-Results-To-File:\phantomsection\label{writeToFile}]{Provide the option of writing the output result data, as given in \hyperref[displayInput]{FR: Display-Input}, \hyperref[displayGraph]{FR: Display-Graph}, \hyperref[displayFS]{FR: Display-Factor-of-Safety}, \hyperref[displayNormal]{FR: Display-Interslice-Normal-Forces}, and \hyperref[displayShear]{FR: Display-Interslice-Shear-Forces}, to a file.}
\end{itemize}
\begin{longtable}{l l l}
\toprule
\textbf{Symbol} & \textbf{Description} & \textbf{Units}
\\
\midrule
\endhead
$\text{(x,y)}$ & Cartesian position coordinates & ${\text{m}}$
\\
$c'$ & Effective cohesion & ${\text{Pa}}$
\\
$const_f$ & Decision on f & --
\\
${{x_{\text{slip}}}^{\text{maxEtr}}}$ & Maximum entry $x$-coordinate & ${\text{m}}$
\\
${{x_{\text{slip}}}^{\text{maxExt}}}$ & Maximum exit $x$-coordinate & ${\text{m}}$
\\
${{x_{\text{slip}}}^{\text{minEtr}}}$ & Minimum entry $x$-coordinate & ${\text{m}}$
\\
${{x_{\text{slip}}}^{\text{minExt}}}$ & Minimum exit $x$-coordinate & ${\text{m}}$
\\
${\mathbf{x}_{\text{slope}}}$ & $x$-coordinates of the slope & ${\text{m}}$
\\
${\mathbf{x}_{\text{wt}}}$ & $x$-coordinates of the water table & ${\text{m}}$
\\
${{y_{\text{slip}}}^{\text{max}}}$ & Maximum $y$-coordinate & ${\text{m}}$
\\
${{y_{\text{slip}}}^{\text{min}}}$ & Minimum $y$-coordinate & ${\text{m}}$
\\
${\mathbf{y}_{\text{slope}}}$ & $y$-coordinates of the slope & ${\text{m}}$
\\
${\mathbf{y}_{\text{wt}}}$ & $y$-coordinates of the water table & ${\text{m}}$
\\
${γ_{\text{dry}}}$ & Soil dry unit weight & $\frac{\text{N}}{\text{m}^{3}}$
\\
${γ_{\text{sat}}}$ & Soil saturated unit weight & $\frac{\text{N}}{\text{m}^{3}}$
\\
${γ_{w}}$ & Unit weight of water & $\frac{\text{N}}{\text{m}^{3}}$
\\
$φ'$ & Effective angle of friction & ${{}^{\circ}}$
\\
\bottomrule
\caption{Required Inputs following \hyperref[readAndStore]{FR: Read-and-Store}}
\label{Table:ReqInputs}
\end{longtable}
\begin{longtable}{l l}
\toprule
\textbf{Symbol} & \textbf{Name}
\\
\midrule
\endhead
$const_f$ & decision on f
\\
${{x_{\text{slip}}}^{\text{maxExt}}}$ & maximum exit $x$-coordinate
\\
${{x_{\text{slip}}}^{\text{maxEtr}}}$ & maximum entry $x$-coordinate
\\
${{x_{\text{slip}}}^{\text{minExt}}}$ & minimum exit $x$-coordinate
\\
${{x_{\text{slip}}}^{\text{minEtr}}}$ & minimum entry $x$-coordinate
\\
${{y_{\text{slip}}}^{\text{max}}}$ & maximum $y$-coordinate
\\
${{y_{\text{slip}}}^{\text{min}}}$ & minimum $y$-coordinate
\\
\bottomrule
\caption{Inputs to be returned as output}
\label{Table:inputsToOutputTable}
\end{longtable}
\subsection{Non-Functional Requirements}
\label{Sec:NFRs}
This section provides the non-functional requirements, the qualities that the software is expected to exhibit.

\begin{itemize}
\item[Correct:\phantomsection\label{correct}]{The outputs of the code have the properties described in \hyperref[Sec:CorSolProps]{Section: Properties of a Correct Solution}.}
\item[Understandable:\phantomsection\label{understandable}]{The code is modularized with complete module guide and module interface specification.}
\item[Reusable:\phantomsection\label{reusable}]{The code is modularized.}
\item[Maintainable:\phantomsection\label{maintainable}]{The traceability between requirements, assumptions, theoretical models, general definitions, data definitions, instance models, likely changes, unlikely changes, and modules is completely recorded in traceability matrices in the SRS and module guide.}
\end{itemize}
\section{Likely Changes}
\label{Sec:LCs}
This section lists the likely changes to be made to the software.

\begin{itemize}
\item[Calculate-Inhomogeneous-Soil-Layers:\phantomsection\label{LC_inhomogeneous}]{\hyperref[assumpSLH]{A: Soil-Layer-Homogeneous} - The system currently assumes the soil mass is homogeneous. In the future, calculations can be added for inconsistent soil properties throughout.}
\item[Calculate-Seismic-Force:\phantomsection\label{LC_seismic}]{\hyperref[assumpSF]{A: Seismic-Force} - The system currently assumes no seismic force. In the future, calculations can be added for the presence of seismic force.}
\item[Calculate-External-Force:\phantomsection\label{LC_external}]{\hyperref[assumpSL]{A: Surface-Load} - The system currently assumes no external forces. In the future, calculations can be added for an imposed surface load on the slope.}
\end{itemize}
\section{Unlikely Changes}
\label{Sec:UCs}
This section lists the unlikely changes to be made to the software.

\begin{itemize}
\item[Normal-And-Shear-Linear-Only:\phantomsection\label{UC_normshearlinear}]{Changes related to \hyperref[assumpINSFL]{A: Interslice-Norm-Shear-Forces-Linear} are not possible due to the dependency of the calculations on the linear relationship between interslice normal forces and interslice shear forces.}
\item[2D-Analysis-Only:\phantomsection\label{UC_2donly}]{\hyperref[assumpENSL]{A: Effective-Norm-Stress-Large} allows for 2D analysis with these models only because stress along the $z$-direction is zero. These models do not take into account stress in the $z$-direction, and therefore cannot be used without manipulation to attempt three-dimensional analysis.}
\end{itemize}
\section{Traceability Matrices and Graphs}
\label{Sec:TraceMatrices}
The purpose of the traceability matrices is to provide easy references on what has to be additionally modified if a certain component is changed. Every time a component is changed, the items in the column of that component that are marked with an ``X'' should be modified as well. \hyperref[Table:TraceMatAvsA]{Table:TraceMatAvsA} shows the dependencies of assumptions on the assumptions. \hyperref[Table:TraceMatAvsAll]{Table:TraceMatAvsAll} shows the dependencies of data definitions, theoretical models, general definitions, instance models, requirements, likely changes, and unlikely changes on the assumptions. \hyperref[Table:TraceMatRefvsRef]{Table:TraceMatRefvsRef} shows the dependencies of data definitions, theoretical models, general definitions, and instance models with each other. \hyperref[Table:TraceMatAllvsR]{Table:TraceMatAllvsR} shows the dependencies of requirements, goal statements on the data definitions, theoretical models, general definitions, and instance models.

\begin{longtable}{l l l l l l l l l l l l l l l l l}
\toprule
\textbf{} & \textbf{\hyperref[assumpSSC]{A: Slip-Surface-Concave}} & \textbf{\hyperref[assumpFOS]{A: Factor-of-Safety}} & \textbf{\hyperref[assumpSLH]{A: Soil-Layer-Homogeneous}} & \textbf{\hyperref[assumpSP]{A: Soil-Properties}} & \textbf{\hyperref[assumpSLI]{A: Soil-Layers-Isotropic}} & \textbf{\hyperref[assumpINSFL]{A: Interslice-Norm-Shear-Forces-Linear}} & \textbf{\hyperref[assumpPSC]{A: Plane-Strain-Conditions}} & \textbf{\hyperref[assumpENSL]{A: Effective-Norm-Stress-Large}} & \textbf{\hyperref[assumpSBSBISL]{A: Surface-Base-Slice-between-Interslice-Straight-Lines}} & \textbf{\hyperref[assumpES]{A: Edge-Slices}} & \textbf{\hyperref[assumpSF]{A: Seismic-Force}} & \textbf{\hyperref[assumpSL]{A: Surface-Load}} & \textbf{\hyperref[assumpWIBE]{A: Water-Intersects-Base-Edge}} & \textbf{\hyperref[assumpWISE]{A: Water-Intersects-Surface-Edge}} & \textbf{\hyperref[assumpNESSS]{A: Negligible-Effect-Surface-Slope-Seismic}} & \textbf{\hyperref[assumpHFSM]{A: Hydrostatic-Force-Slice-Midpoint}}
\\
\midrule
\endhead
\hyperref[assumpSSC]{A: Slip-Surface-Concave} &  &  &  &  &  &  &  &  &  &  &  &  &  &  &  & 
\\
\hyperref[assumpFOS]{A: Factor-of-Safety} &  &  &  &  &  &  &  &  &  &  &  &  &  &  &  & 
\\
\hyperref[assumpSLH]{A: Soil-Layer-Homogeneous} &  &  &  &  &  &  &  &  &  &  &  &  &  &  &  & 
\\
\hyperref[assumpSP]{A: Soil-Properties} &  &  &  &  &  &  &  &  &  &  &  &  &  &  &  & 
\\
\hyperref[assumpSLI]{A: Soil-Layers-Isotropic} &  &  &  &  &  &  &  &  &  &  &  &  &  &  &  & 
\\
\hyperref[assumpINSFL]{A: Interslice-Norm-Shear-Forces-Linear} &  &  &  &  &  &  &  &  &  &  &  &  &  &  &  & 
\\
\hyperref[assumpPSC]{A: Plane-Strain-Conditions} &  &  &  &  &  &  &  &  &  &  &  &  &  &  &  & 
\\
\hyperref[assumpENSL]{A: Effective-Norm-Stress-Large} &  &  &  &  &  &  &  &  &  &  &  &  &  &  &  & 
\\
\hyperref[assumpSBSBISL]{A: Surface-Base-Slice-between-Interslice-Straight-Lines} &  &  &  &  &  &  &  &  &  &  &  &  &  &  &  & 
\\
\hyperref[assumpES]{A: Edge-Slices} &  &  &  &  &  &  &  &  &  &  &  &  &  &  &  & 
\\
\hyperref[assumpSF]{A: Seismic-Force} &  &  &  &  &  &  &  &  &  &  &  &  &  &  &  & 
\\
\hyperref[assumpSL]{A: Surface-Load} &  &  &  &  &  &  &  &  &  &  &  &  &  &  &  & 
\\
\hyperref[assumpWIBE]{A: Water-Intersects-Base-Edge} &  &  &  &  &  &  &  &  &  &  &  &  &  &  &  & 
\\
\hyperref[assumpWISE]{A: Water-Intersects-Surface-Edge} &  &  &  &  &  &  &  &  &  &  &  &  &  &  &  & 
\\
\hyperref[assumpNESSS]{A: Negligible-Effect-Surface-Slope-Seismic} &  &  &  &  &  &  &  &  &  &  &  &  &  &  &  & 
\\
\hyperref[assumpHFSM]{A: Hydrostatic-Force-Slice-Midpoint} &  &  &  &  &  &  &  &  &  &  &  &  &  &  &  & 
\\
\bottomrule
\caption{Traceability Matrix Showing the Connections Between Assumptions dependence of each other.}
\label{Table:TraceMatAvsA}
\end{longtable}
\begin{longtable}{l l l l l l l l l l l l l l l l l}
\toprule
\textbf{} & \textbf{\hyperref[assumpSSC]{A: Slip-Surface-Concave}} & \textbf{\hyperref[assumpFOS]{A: Factor-of-Safety}} & \textbf{\hyperref[assumpSLH]{A: Soil-Layer-Homogeneous}} & \textbf{\hyperref[assumpSP]{A: Soil-Properties}} & \textbf{\hyperref[assumpSLI]{A: Soil-Layers-Isotropic}} & \textbf{\hyperref[assumpINSFL]{A: Interslice-Norm-Shear-Forces-Linear}} & \textbf{\hyperref[assumpPSC]{A: Plane-Strain-Conditions}} & \textbf{\hyperref[assumpENSL]{A: Effective-Norm-Stress-Large}} & \textbf{\hyperref[assumpSBSBISL]{A: Surface-Base-Slice-between-Interslice-Straight-Lines}} & \textbf{\hyperref[assumpES]{A: Edge-Slices}} & \textbf{\hyperref[assumpSF]{A: Seismic-Force}} & \textbf{\hyperref[assumpSL]{A: Surface-Load}} & \textbf{\hyperref[assumpWIBE]{A: Water-Intersects-Base-Edge}} & \textbf{\hyperref[assumpWISE]{A: Water-Intersects-Surface-Edge}} & \textbf{\hyperref[assumpNESSS]{A: Negligible-Effect-Surface-Slope-Seismic}} & \textbf{\hyperref[assumpHFSM]{A: Hydrostatic-Force-Slice-Midpoint}}
\\
\midrule
\endhead
\hyperref[DD:intersliceWtrF]{DD: intersliceWtrF} &  &  &  &  &  &  &  &  &  &  &  &  &  &  &  & 
\\
\hyperref[DD:angleA]{DD: angleA} &  &  &  &  &  &  &  &  & X &  &  &  &  &  &  & 
\\
\hyperref[DD:angleB]{DD: angleB} &  &  &  &  &  &  &  &  & X &  &  &  &  &  &  & 
\\
\hyperref[DD:lengthB]{DD: lengthB} &  &  &  &  &  &  &  &  &  &  &  &  &  &  &  & 
\\
\hyperref[DD:lengthLb]{DD: lengthLb} &  &  &  &  &  &  &  &  &  &  &  &  &  &  &  & 
\\
\hyperref[DD:lengthLs]{DD: lengthLs} &  &  &  &  &  &  &  &  &  &  &  &  &  &  &  & 
\\
\hyperref[DD:slcHeight]{DD: slcHeight} &  &  &  &  &  &  &  &  & X &  &  &  &  &  &  & 
\\
\hyperref[DD:normStress]{DD: normStress} &  &  &  &  &  &  &  &  &  &  &  &  &  &  &  & 
\\
\hyperref[DD:tangStress]{DD: tangStress} &  &  &  &  &  &  &  &  &  &  &  &  &  &  &  & 
\\
\hyperref[DD:torque]{DD: torque} &  &  &  &  &  &  &  &  &  &  &  &  &  &  &  & 
\\
\hyperref[DD:ratioVariation]{DD: ratioVariation} &  &  &  &  &  &  &  &  &  &  &  &  &  &  &  & 
\\
\hyperref[DD:convertFunc1]{DD: convertFunc1} &  &  &  &  &  &  &  &  &  &  &  &  &  &  &  & 
\\
\hyperref[DD:convertFunc2]{DD: convertFunc2} &  &  &  &  &  &  &  &  &  &  &  &  &  &  &  & 
\\
\hyperref[DD:nrmForceSumDD]{DD: nrmForceSumDD} &  &  &  &  &  &  &  &  &  &  &  &  &  &  &  & 
\\
\hyperref[DD:watForceSumDD]{DD: watForceSumDD} &  &  &  &  &  &  &  &  &  &  &  &  &  &  &  & 
\\
\hyperref[DD:sliceHghtRightDD]{DD: sliceHghtRightDD} &  &  &  &  &  &  &  &  &  &  &  &  &  &  &  & 
\\
\hyperref[DD:sliceHghtLeftDD]{DD: sliceHghtLeftDD} &  &  &  &  &  &  &  &  &  &  &  &  &  &  &  & 
\\
\hyperref[TM:factOfSafety]{TM: factOfSafety} &  &  &  &  &  &  &  &  &  &  &  &  &  &  &  & 
\\
\hyperref[TM:equilibrium]{TM: equilibrium} &  &  &  &  &  &  &  & X &  &  &  &  &  &  &  & 
\\
\hyperref[TM:mcShrStrgth]{TM: mcShrStrgth} &  &  &  &  &  &  &  &  & X &  &  &  &  &  &  & 
\\
\hyperref[TM:effStress]{TM: effStress} &  &  &  &  &  &  &  &  &  &  &  &  &  &  &  & 
\\
\hyperref[TM:NewtonSecLawMot]{TM: NewtonSecLawMot} &  &  &  &  &  &  &  &  &  &  &  &  &  &  &  & 
\\
\hyperref[GD:normForcEq]{GD: normForcEq} &  &  &  &  &  &  &  &  &  &  &  &  &  &  &  & 
\\
\hyperref[GD:bsShrFEq]{GD: bsShrFEq} &  &  &  &  &  &  &  &  &  &  &  &  &  &  &  & 
\\
\hyperref[GD:resShr]{GD: resShr} &  &  & X & X & X &  & X &  &  &  &  &  &  &  &  & 
\\
\hyperref[GD:mobShr]{GD: mobShr} &  & X &  &  &  &  &  &  &  &  &  &  &  &  &  & 
\\
\hyperref[GD:effNormF]{GD: effNormF} &  &  &  &  &  &  & X &  &  &  &  &  &  &  &  & 
\\
\hyperref[GD:resShearWO]{GD: resShearWO} &  &  &  &  &  &  &  &  &  &  &  &  &  &  &  & 
\\
\hyperref[GD:mobShearWO]{GD: mobShearWO} &  &  &  &  &  &  &  &  &  &  &  &  &  &  &  & 
\\
\hyperref[GD:normShrR]{GD: normShrR} &  &  &  &  &  & X &  &  &  &  &  &  &  &  &  & 
\\
\hyperref[GD:momentEql]{GD: momentEql} &  &  &  &  &  &  &  &  &  &  &  &  &  &  & X & X
\\
\hyperref[GD:weight]{GD: weight} &  &  &  &  &  &  &  &  &  &  &  &  &  &  &  & 
\\
\hyperref[GD:sliceWght]{GD: sliceWght} &  &  & X &  &  &  & X &  & X &  &  &  & X & X &  & 
\\
\hyperref[GD:hsPressure]{GD: hsPressure} &  &  &  &  &  &  &  &  &  &  &  &  &  &  &  & 
\\
\hyperref[GD:baseWtrF]{GD: baseWtrF} &  &  &  &  &  &  & X &  & X &  &  &  & X &  &  & X
\\
\hyperref[GD:srfWtrF]{GD: srfWtrF} &  &  &  &  &  &  & X &  & X &  &  &  &  & X &  & X
\\
\hyperref[IM:fctSfty]{IM: fctSfty} &  &  &  &  &  & X &  &  &  & X & X & X &  &  &  & 
\\
\hyperref[IM:nrmShrFor]{IM: nrmShrFor} &  &  &  &  &  & X &  &  &  & X & X & X &  &  &  & 
\\
\hyperref[IM:nrmShrForNum]{IM: nrmShrForNum} &  &  &  &  &  &  &  &  &  &  &  &  &  &  &  & 
\\
\hyperref[IM:nrmShrForDen]{IM: nrmShrForDen} &  &  &  &  &  &  &  &  &  &  &  &  &  &  &  & 
\\
\hyperref[IM:intsliceFs]{IM: intsliceFs} &  &  &  &  &  &  &  &  &  & X &  &  &  &  &  & 
\\
\hyperref[IM:crtSlpId]{IM: crtSlpId} & X &  &  &  &  &  &  &  &  &  &  &  &  &  &  & 
\\
\hyperref[readAndStore]{FR: Read-and-Store} &  &  &  &  &  &  &  &  &  &  &  &  &  &  &  & 
\\
\hyperref[verifyInput]{FR: Verify-Input} &  &  &  &  &  &  &  &  &  &  &  &  &  &  &  & 
\\
\hyperref[determineCritSlip]{FR: Determine-Critical-Slip-Surface} &  &  &  &  &  &  &  &  &  &  &  &  &  &  &  & 
\\
\hyperref[verifyOutput]{FR: Verify-Output} &  &  &  &  &  &  &  &  &  &  &  &  &  &  &  & 
\\
\hyperref[displayInput]{FR: Display-Input} &  &  &  &  &  &  &  &  &  &  &  &  &  &  &  & 
\\
\hyperref[displayGraph]{FR: Display-Graph} &  &  &  &  &  &  &  &  &  &  &  &  &  &  &  & 
\\
\hyperref[displayFS]{FR: Display-Factor-of-Safety} &  &  &  &  &  &  &  &  &  &  &  &  &  &  &  & 
\\
\hyperref[displayNormal]{FR: Display-Interslice-Normal-Forces} &  &  &  &  &  &  &  &  &  &  &  &  &  &  &  & 
\\
\hyperref[displayShear]{FR: Display-Interslice-Shear-Forces} &  &  &  &  &  &  &  &  &  &  &  &  &  &  &  & 
\\
\hyperref[writeToFile]{FR: Write-Results-To-File} &  &  &  &  &  &  &  &  &  &  &  &  &  &  &  & 
\\
\hyperref[correct]{NFR: Correct} &  &  &  &  &  &  &  &  &  &  &  &  &  &  &  & 
\\
\hyperref[understandable]{NFR: Understandable} &  &  &  &  &  &  &  &  &  &  &  &  &  &  &  & 
\\
\hyperref[reusable]{NFR: Reusable} &  &  &  &  &  &  &  &  &  &  &  &  &  &  &  & 
\\
\hyperref[maintainable]{NFR: Maintainable} &  &  &  &  &  &  &  &  &  &  &  &  &  &  &  & 
\\
\hyperref[LC_inhomogeneous]{LC: Calculate-Inhomogeneous-Soil-Layers} &  &  & X &  &  &  &  &  &  &  &  &  &  &  &  & 
\\
\hyperref[LC_seismic]{LC: Calculate-Seismic-Force} &  &  &  &  &  &  &  &  &  &  & X &  &  &  &  & 
\\
\hyperref[LC_external]{LC: Calculate-External-Force} &  &  &  &  &  &  &  &  &  &  &  & X &  &  &  & 
\\
\hyperref[UC_normshearlinear]{UC: Normal-And-Shear-Linear-Only} &  &  &  &  &  & X &  &  &  &  &  &  &  &  &  & 
\\
\hyperref[UC_2donly]{UC: 2D-Analysis-Only} &  &  &  &  &  &  &  & X &  &  &  &  &  &  &  & 
\\
\bottomrule
\caption{Traceability Matrix Showing the Connections Between Assumptions and Other Items}
\label{Table:TraceMatAvsAll}
\end{longtable}
\begin{longtable}{l l l l l l l l l l l l l l l l l l l l l l l l l l l l l l l l l l l l l l l l l l l}
\toprule
\textbf{} & \textbf{\hyperref[DD:intersliceWtrF]{DD: intersliceWtrF}} & \textbf{\hyperref[DD:angleA]{DD: angleA}} & \textbf{\hyperref[DD:angleB]{DD: angleB}} & \textbf{\hyperref[DD:lengthB]{DD: lengthB}} & \textbf{\hyperref[DD:lengthLb]{DD: lengthLb}} & \textbf{\hyperref[DD:lengthLs]{DD: lengthLs}} & \textbf{\hyperref[DD:slcHeight]{DD: slcHeight}} & \textbf{\hyperref[DD:normStress]{DD: normStress}} & \textbf{\hyperref[DD:tangStress]{DD: tangStress}} & \textbf{\hyperref[DD:torque]{DD: torque}} & \textbf{\hyperref[DD:ratioVariation]{DD: ratioVariation}} & \textbf{\hyperref[DD:convertFunc1]{DD: convertFunc1}} & \textbf{\hyperref[DD:convertFunc2]{DD: convertFunc2}} & \textbf{\hyperref[DD:nrmForceSumDD]{DD: nrmForceSumDD}} & \textbf{\hyperref[DD:watForceSumDD]{DD: watForceSumDD}} & \textbf{\hyperref[DD:sliceHghtRightDD]{DD: sliceHghtRightDD}} & \textbf{\hyperref[DD:sliceHghtLeftDD]{DD: sliceHghtLeftDD}} & \textbf{\hyperref[TM:factOfSafety]{TM: factOfSafety}} & \textbf{\hyperref[TM:equilibrium]{TM: equilibrium}} & \textbf{\hyperref[TM:mcShrStrgth]{TM: mcShrStrgth}} & \textbf{\hyperref[TM:effStress]{TM: effStress}} & \textbf{\hyperref[TM:NewtonSecLawMot]{TM: NewtonSecLawMot}} & \textbf{\hyperref[GD:normForcEq]{GD: normForcEq}} & \textbf{\hyperref[GD:bsShrFEq]{GD: bsShrFEq}} & \textbf{\hyperref[GD:resShr]{GD: resShr}} & \textbf{\hyperref[GD:mobShr]{GD: mobShr}} & \textbf{\hyperref[GD:effNormF]{GD: effNormF}} & \textbf{\hyperref[GD:resShearWO]{GD: resShearWO}} & \textbf{\hyperref[GD:mobShearWO]{GD: mobShearWO}} & \textbf{\hyperref[GD:normShrR]{GD: normShrR}} & \textbf{\hyperref[GD:momentEql]{GD: momentEql}} & \textbf{\hyperref[GD:weight]{GD: weight}} & \textbf{\hyperref[GD:sliceWght]{GD: sliceWght}} & \textbf{\hyperref[GD:hsPressure]{GD: hsPressure}} & \textbf{\hyperref[GD:baseWtrF]{GD: baseWtrF}} & \textbf{\hyperref[GD:srfWtrF]{GD: srfWtrF}} & \textbf{\hyperref[IM:fctSfty]{IM: fctSfty}} & \textbf{\hyperref[IM:nrmShrFor]{IM: nrmShrFor}} & \textbf{\hyperref[IM:nrmShrForNum]{IM: nrmShrForNum}} & \textbf{\hyperref[IM:nrmShrForDen]{IM: nrmShrForDen}} & \textbf{\hyperref[IM:intsliceFs]{IM: intsliceFs}} & \textbf{\hyperref[IM:crtSlpId]{IM: crtSlpId}}
\\
\midrule
\endhead
\hyperref[DD:intersliceWtrF]{DD: intersliceWtrF} &  &  &  &  &  &  &  &  &  &  &  &  &  &  &  &  &  &  &  &  &  &  &  &  &  &  &  &  &  &  &  &  &  &  &  &  &  &  &  &  &  & 
\\
\hyperref[DD:angleA]{DD: angleA} &  &  &  &  &  &  &  &  &  &  &  &  &  &  &  &  &  &  &  &  &  &  &  &  &  &  &  &  &  &  &  &  &  &  &  &  &  &  &  &  &  & 
\\
\hyperref[DD:angleB]{DD: angleB} &  &  &  &  &  &  &  &  &  &  &  &  &  &  &  &  &  &  &  &  &  &  &  &  &  &  &  &  &  &  &  &  &  &  &  &  &  &  &  &  &  & 
\\
\hyperref[DD:lengthB]{DD: lengthB} &  &  &  &  &  &  &  &  &  &  &  &  &  &  &  &  &  &  &  &  &  &  &  &  &  &  &  &  &  &  &  &  &  &  &  &  &  &  &  &  &  & 
\\
\hyperref[DD:lengthLb]{DD: lengthLb} &  & X &  & X &  &  &  &  &  &  &  &  &  &  &  &  &  &  &  &  &  &  &  &  &  &  &  &  &  &  &  &  &  &  &  &  &  &  &  &  &  & 
\\
\hyperref[DD:lengthLs]{DD: lengthLs} &  &  & X & X &  &  &  &  &  &  &  &  &  &  &  &  &  &  &  &  &  &  &  &  &  &  &  &  &  &  &  &  &  &  &  &  &  &  &  &  &  & 
\\
\hyperref[DD:slcHeight]{DD: slcHeight} &  &  &  &  &  &  &  &  &  &  &  &  &  &  &  & X & X &  &  &  &  &  &  &  &  &  &  &  &  &  &  &  &  &  &  &  &  &  &  &  &  & 
\\
\hyperref[DD:normStress]{DD: normStress} &  &  &  &  &  &  &  &  &  &  &  &  &  &  &  &  &  &  &  &  &  &  &  &  &  &  &  &  &  &  &  &  &  &  &  &  &  &  &  &  &  & 
\\
\hyperref[DD:tangStress]{DD: tangStress} &  &  &  &  &  &  &  &  &  &  &  &  &  &  &  &  &  &  &  &  &  &  &  &  &  &  &  &  &  &  &  &  &  &  &  &  &  &  &  &  &  & 
\\
\hyperref[DD:torque]{DD: torque} &  &  &  &  &  &  &  &  &  &  &  &  &  &  &  &  &  &  &  &  &  &  &  &  &  &  &  &  &  &  &  &  &  &  &  &  &  &  &  &  &  & 
\\
\hyperref[DD:ratioVariation]{DD: ratioVariation} &  &  &  &  &  &  &  &  &  &  &  &  &  &  &  &  &  &  &  &  &  &  &  &  &  &  &  &  &  &  &  &  &  &  &  &  &  &  &  &  &  & 
\\
\hyperref[DD:convertFunc1]{DD: convertFunc1} &  & X &  &  &  &  &  &  &  &  & X &  &  &  &  &  &  &  &  &  &  &  &  &  &  &  &  &  &  &  &  &  &  &  &  &  &  &  &  &  &  & 
\\
\hyperref[DD:convertFunc2]{DD: convertFunc2} &  & X &  &  &  &  &  &  &  &  & X & X &  &  &  &  &  &  &  &  &  &  &  &  &  &  &  &  &  &  &  &  &  &  &  &  &  &  &  &  &  & 
\\
\hyperref[DD:nrmForceSumDD]{DD: nrmForceSumDD} &  &  &  &  &  &  &  &  &  &  &  &  &  &  &  &  &  &  &  &  &  &  &  &  &  &  &  &  &  &  &  &  &  &  &  &  &  &  &  &  &  & 
\\
\hyperref[DD:watForceSumDD]{DD: watForceSumDD} &  &  &  &  &  &  &  &  &  &  &  &  &  &  &  &  &  &  &  &  &  &  &  &  &  &  &  &  &  &  &  &  &  &  &  &  &  &  &  &  &  & 
\\
\hyperref[DD:sliceHghtRightDD]{DD: sliceHghtRightDD} &  &  &  &  &  &  &  &  &  &  &  &  &  &  &  &  &  &  &  &  &  &  &  &  &  &  &  &  &  &  &  &  &  &  &  &  &  &  &  &  &  & 
\\
\hyperref[DD:sliceHghtLeftDD]{DD: sliceHghtLeftDD} &  &  &  &  &  &  &  &  &  &  &  &  &  &  &  &  &  &  &  &  &  &  &  &  &  &  &  &  &  &  &  &  &  &  &  &  &  &  &  &  &  & 
\\
\hyperref[TM:factOfSafety]{TM: factOfSafety} &  &  &  &  &  &  &  &  &  &  &  &  &  &  &  &  &  &  &  &  &  &  &  &  &  &  &  &  &  &  &  &  &  &  &  &  &  &  &  &  &  & 
\\
\hyperref[TM:equilibrium]{TM: equilibrium} &  &  &  &  &  &  &  &  &  &  &  &  &  &  &  &  &  &  &  &  &  &  &  &  &  &  &  &  &  &  &  &  &  &  &  &  &  &  &  &  &  & 
\\
\hyperref[TM:mcShrStrgth]{TM: mcShrStrgth} &  &  &  &  &  &  &  &  &  &  &  &  &  &  &  &  &  &  &  &  &  &  &  &  &  &  &  &  &  &  &  &  &  &  &  &  &  &  &  &  &  & 
\\
\hyperref[TM:effStress]{TM: effStress} &  &  &  &  &  &  &  & X &  &  &  &  &  &  &  &  &  &  &  &  &  &  &  &  &  &  &  &  &  &  &  &  &  &  &  &  &  &  &  &  &  & 
\\
\hyperref[TM:NewtonSecLawMot]{TM: NewtonSecLawMot} &  &  &  &  &  &  &  &  &  &  &  &  &  &  &  &  &  &  &  &  &  &  &  &  &  &  &  &  &  &  &  &  &  &  &  &  &  &  &  &  &  & 
\\
\hyperref[GD:normForcEq]{GD: normForcEq} &  & X & X &  &  &  &  &  &  &  &  &  &  &  &  &  &  &  & X &  &  &  &  &  &  &  &  &  &  &  &  &  & X &  &  & X &  &  &  &  &  & 
\\
\hyperref[GD:bsShrFEq]{GD: bsShrFEq} &  & X & X &  &  &  &  &  &  &  &  &  &  &  &  &  &  &  & X &  &  &  &  &  &  &  &  &  &  &  &  &  & X &  &  & X &  &  &  &  &  & 
\\
\hyperref[GD:resShr]{GD: resShr} &  &  &  &  & X &  &  & X & X &  &  &  &  &  &  &  &  &  &  & X &  &  &  &  &  &  &  &  &  &  &  &  &  &  &  &  &  &  &  &  &  & 
\\
\hyperref[GD:mobShr]{GD: mobShr} &  &  &  &  & X &  &  &  &  &  &  &  &  &  &  &  &  & X &  &  &  &  &  &  & X &  &  &  &  &  &  &  &  &  &  &  &  &  &  &  &  & 
\\
\hyperref[GD:effNormF]{GD: effNormF} &  &  &  &  &  &  &  & X &  &  &  &  &  &  &  &  &  &  &  &  & X &  &  &  &  &  &  &  &  &  &  &  &  &  & X &  &  &  &  &  &  & 
\\
\hyperref[GD:resShearWO]{GD: resShearWO} & X & X & X &  & X &  &  &  &  &  &  &  &  &  &  &  &  &  &  &  &  &  &  &  &  &  &  &  &  &  &  &  & X &  & X & X &  &  &  &  &  & 
\\
\hyperref[GD:mobShearWO]{GD: mobShearWO} & X & X & X &  &  &  &  &  &  &  &  &  &  &  &  &  &  &  &  &  &  &  &  &  &  &  &  &  &  &  &  &  & X &  &  & X &  &  &  &  &  & 
\\
\hyperref[GD:normShrR]{GD: normShrR} &  &  &  &  &  &  &  &  &  &  & X &  &  &  &  &  &  &  &  &  &  &  &  &  &  &  &  &  &  &  &  &  &  &  &  &  &  &  &  &  &  & 
\\
\hyperref[GD:momentEql]{GD: momentEql} &  & X & X & X &  &  & X &  &  & X &  &  &  &  &  &  &  &  & X &  &  &  &  &  &  &  &  &  &  &  &  & X & X &  &  & X &  &  &  &  &  & 
\\
\hyperref[GD:weight]{GD: weight} &  &  &  &  &  &  &  &  &  &  &  &  &  &  &  &  &  &  &  &  &  & X &  &  &  &  &  &  &  &  &  &  &  &  &  &  &  &  &  &  &  & 
\\
\hyperref[GD:sliceWght]{GD: sliceWght} &  &  &  & X &  &  &  &  &  &  &  &  &  &  &  &  &  &  &  &  &  &  &  &  &  &  &  &  &  &  &  & X &  &  &  &  &  &  &  &  &  & 
\\
\hyperref[GD:hsPressure]{GD: hsPressure} &  &  &  &  &  &  &  &  &  &  &  &  &  &  &  &  &  &  &  &  &  &  &  &  &  &  &  &  &  &  &  &  &  &  &  &  &  &  &  &  &  & 
\\
\hyperref[GD:baseWtrF]{GD: baseWtrF} &  &  &  &  & X &  &  &  &  &  &  &  &  &  &  &  &  &  &  &  &  &  &  &  &  &  &  &  &  &  &  &  &  & X & X &  &  &  &  &  &  & 
\\
\hyperref[GD:srfWtrF]{GD: srfWtrF} &  &  &  &  &  & X &  &  &  &  &  &  &  &  &  &  &  &  &  &  &  &  &  &  &  &  &  &  &  &  &  &  &  & X &  & X &  &  &  &  &  & 
\\
\hyperref[IM:fctSfty]{IM: fctSfty} &  &  &  &  &  &  &  &  &  &  &  & X & X &  &  &  &  &  &  &  &  &  & X & X &  & X &  & X & X & X &  &  &  &  &  &  & X & X &  &  & X & 
\\
\hyperref[IM:nrmShrFor]{IM: nrmShrFor} &  &  &  &  &  &  &  &  &  &  &  &  &  &  &  &  &  &  &  &  &  &  &  &  &  &  &  &  &  & X & X &  &  &  &  &  & X & X & X & X & X & 
\\
\hyperref[IM:nrmShrForNum]{IM: nrmShrForNum} & X & X & X & X &  &  & X &  &  &  &  &  &  &  &  &  &  &  &  &  &  &  &  &  &  &  &  &  &  &  &  &  &  &  &  & X &  & X &  &  &  & 
\\
\hyperref[IM:nrmShrForDen]{IM: nrmShrForDen} &  &  &  & X &  &  &  &  &  &  & X &  &  &  &  &  &  &  &  &  &  &  &  &  &  &  &  &  &  &  &  &  &  &  &  &  &  & X &  &  &  & 
\\
\hyperref[IM:intsliceFs]{IM: intsliceFs} &  &  &  &  &  &  &  &  &  &  &  & X & X &  &  &  &  &  &  &  &  &  &  &  &  &  &  & X & X &  &  &  &  &  &  &  & X & X &  &  & X & 
\\
\hyperref[IM:crtSlpId]{IM: crtSlpId} &  &  &  &  &  &  &  &  &  &  &  &  &  &  &  &  &  &  &  &  &  &  &  &  &  &  &  &  &  &  &  &  &  &  &  &  &  &  &  &  &  & 
\\
\bottomrule
\caption{Traceability Matrix Showing the Connections Between Items and Other Sections}
\label{Table:TraceMatRefvsRef}
\end{longtable}
\begin{longtable}{l l l l l l l l l l l l l l l l l l l l l l l l l l l l l l l l l l l l l l l l l l l l l l l l l l l l l l l l l}
\toprule
\textbf{} & \textbf{\hyperref[DD:intersliceWtrF]{DD: intersliceWtrF}} & \textbf{\hyperref[DD:angleA]{DD: angleA}} & \textbf{\hyperref[DD:angleB]{DD: angleB}} & \textbf{\hyperref[DD:lengthB]{DD: lengthB}} & \textbf{\hyperref[DD:lengthLb]{DD: lengthLb}} & \textbf{\hyperref[DD:lengthLs]{DD: lengthLs}} & \textbf{\hyperref[DD:slcHeight]{DD: slcHeight}} & \textbf{\hyperref[DD:normStress]{DD: normStress}} & \textbf{\hyperref[DD:tangStress]{DD: tangStress}} & \textbf{\hyperref[DD:torque]{DD: torque}} & \textbf{\hyperref[DD:ratioVariation]{DD: ratioVariation}} & \textbf{\hyperref[DD:convertFunc1]{DD: convertFunc1}} & \textbf{\hyperref[DD:convertFunc2]{DD: convertFunc2}} & \textbf{\hyperref[DD:nrmForceSumDD]{DD: nrmForceSumDD}} & \textbf{\hyperref[DD:watForceSumDD]{DD: watForceSumDD}} & \textbf{\hyperref[DD:sliceHghtRightDD]{DD: sliceHghtRightDD}} & \textbf{\hyperref[DD:sliceHghtLeftDD]{DD: sliceHghtLeftDD}} & \textbf{\hyperref[TM:factOfSafety]{TM: factOfSafety}} & \textbf{\hyperref[TM:equilibrium]{TM: equilibrium}} & \textbf{\hyperref[TM:mcShrStrgth]{TM: mcShrStrgth}} & \textbf{\hyperref[TM:effStress]{TM: effStress}} & \textbf{\hyperref[TM:NewtonSecLawMot]{TM: NewtonSecLawMot}} & \textbf{\hyperref[GD:normForcEq]{GD: normForcEq}} & \textbf{\hyperref[GD:bsShrFEq]{GD: bsShrFEq}} & \textbf{\hyperref[GD:resShr]{GD: resShr}} & \textbf{\hyperref[GD:mobShr]{GD: mobShr}} & \textbf{\hyperref[GD:effNormF]{GD: effNormF}} & \textbf{\hyperref[GD:resShearWO]{GD: resShearWO}} & \textbf{\hyperref[GD:mobShearWO]{GD: mobShearWO}} & \textbf{\hyperref[GD:normShrR]{GD: normShrR}} & \textbf{\hyperref[GD:momentEql]{GD: momentEql}} & \textbf{\hyperref[GD:weight]{GD: weight}} & \textbf{\hyperref[GD:sliceWght]{GD: sliceWght}} & \textbf{\hyperref[GD:hsPressure]{GD: hsPressure}} & \textbf{\hyperref[GD:baseWtrF]{GD: baseWtrF}} & \textbf{\hyperref[GD:srfWtrF]{GD: srfWtrF}} & \textbf{\hyperref[IM:fctSfty]{IM: fctSfty}} & \textbf{\hyperref[IM:nrmShrFor]{IM: nrmShrFor}} & \textbf{\hyperref[IM:nrmShrForNum]{IM: nrmShrForNum}} & \textbf{\hyperref[IM:nrmShrForDen]{IM: nrmShrForDen}} & \textbf{\hyperref[IM:intsliceFs]{IM: intsliceFs}} & \textbf{\hyperref[IM:crtSlpId]{IM: crtSlpId}} & \textbf{\hyperref[readAndStore]{FR: Read-and-Store}} & \textbf{\hyperref[verifyInput]{FR: Verify-Input}} & \textbf{\hyperref[determineCritSlip]{FR: Determine-Critical-Slip-Surface}} & \textbf{\hyperref[verifyOutput]{FR: Verify-Output}} & \textbf{\hyperref[displayInput]{FR: Display-Input}} & \textbf{\hyperref[displayGraph]{FR: Display-Graph}} & \textbf{\hyperref[displayFS]{FR: Display-Factor-of-Safety}} & \textbf{\hyperref[displayNormal]{FR: Display-Interslice-Normal-Forces}} & \textbf{\hyperref[displayShear]{FR: Display-Interslice-Shear-Forces}} & \textbf{\hyperref[writeToFile]{FR: Write-Results-To-File}} & \textbf{\hyperref[correct]{NFR: Correct}} & \textbf{\hyperref[understandable]{NFR: Understandable}} & \textbf{\hyperref[reusable]{NFR: Reusable}} & \textbf{\hyperref[maintainable]{NFR: Maintainable}}
\\
\midrule
\endhead
\hyperref[identifyCritAndFS]{GS: Identify-Crit-and-FS} &  &  &  &  &  &  &  &  &  &  &  &  &  &  &  &  &  &  &  &  &  &  &  &  &  &  &  &  &  &  &  &  &  &  &  &  &  &  &  &  &  &  &  &  &  &  &  &  &  &  &  &  &  &  &  & 
\\
\hyperref[determineNormalF]{GS: Determine-Normal-Forces} &  &  &  &  &  &  &  &  &  &  &  &  &  &  &  &  &  &  &  &  &  &  &  &  &  &  &  &  &  &  &  &  &  &  &  &  &  &  &  &  &  &  &  &  &  &  &  &  &  &  &  &  &  &  &  & 
\\
\hyperref[determineShearF]{GS: Determine-Shear-Forces} &  &  &  &  &  &  &  &  &  &  &  &  &  &  &  &  &  &  &  &  &  &  &  &  &  &  &  &  &  &  &  &  &  &  &  &  &  &  &  &  &  &  &  &  &  &  &  &  &  &  &  &  &  &  &  & 
\\
\hyperref[readAndStore]{FR: Read-and-Store} &  &  &  &  &  &  &  &  &  &  &  &  &  &  &  &  &  &  &  &  &  &  &  &  &  &  &  &  &  &  &  &  &  &  &  &  &  &  &  &  &  &  &  &  &  &  &  &  &  &  &  &  &  &  &  & 
\\
\hyperref[verifyInput]{FR: Verify-Input} &  &  &  &  &  &  &  &  &  &  &  &  &  &  &  &  &  &  &  &  &  &  &  &  &  &  &  &  &  &  &  &  &  &  &  &  &  &  &  &  &  &  &  &  &  &  &  &  &  &  &  &  &  &  &  & 
\\
\hyperref[determineCritSlip]{FR: Determine-Critical-Slip-Surface} &  &  &  &  &  &  &  &  &  &  &  &  &  &  &  &  &  &  &  &  &  &  &  &  &  &  &  &  &  &  &  &  &  &  &  &  & X & X &  &  & X & X &  &  &  &  &  &  &  &  &  &  &  &  &  & 
\\
\hyperref[verifyOutput]{FR: Verify-Output} &  &  &  &  &  &  &  &  &  &  &  &  &  &  &  &  &  &  &  &  &  &  &  &  &  &  &  &  &  &  &  &  &  &  &  &  &  &  &  &  &  &  &  &  &  &  &  &  &  &  &  &  &  &  &  & 
\\
\hyperref[displayInput]{FR: Display-Input} &  &  &  &  &  &  &  &  &  &  &  &  &  &  &  &  &  &  &  &  &  &  &  &  &  &  &  &  &  &  &  &  &  &  &  &  &  &  &  &  &  &  &  &  &  &  &  &  &  &  &  &  &  &  &  & 
\\
\hyperref[displayGraph]{FR: Display-Graph} &  &  &  &  &  &  &  &  &  &  &  &  &  &  &  &  &  &  &  &  &  &  &  &  &  &  &  &  &  &  &  &  &  &  &  &  &  &  &  &  &  & X &  &  &  &  &  &  &  &  &  &  &  &  &  & 
\\
\hyperref[displayFS]{FR: Display-Factor-of-Safety} &  &  &  &  &  &  &  &  &  &  &  &  &  &  &  &  &  &  &  &  &  &  &  &  &  &  &  &  &  &  &  &  &  &  &  &  & X & X &  &  & X &  &  &  &  &  &  &  &  &  &  &  &  &  &  & 
\\
\hyperref[displayNormal]{FR: Display-Interslice-Normal-Forces} &  &  &  &  &  &  &  &  &  &  &  &  &  &  &  &  &  &  &  &  &  &  &  &  &  &  &  &  &  &  &  &  &  &  &  &  & X & X &  &  & X &  &  &  &  &  &  &  &  &  &  &  &  &  &  & 
\\
\hyperref[displayShear]{FR: Display-Interslice-Shear-Forces} &  &  &  &  &  &  &  &  &  &  &  &  &  &  &  &  &  &  &  &  &  &  &  &  &  &  &  &  &  &  &  &  &  &  &  &  & X & X &  &  & X &  &  &  &  &  &  &  &  &  &  &  &  &  &  & 
\\
\hyperref[writeToFile]{FR: Write-Results-To-File} &  &  &  &  &  &  &  &  &  &  &  &  &  &  &  &  &  &  &  &  &  &  &  &  &  &  &  &  &  &  &  &  &  &  &  &  &  &  &  &  &  &  &  &  &  &  & X & X & X & X & X &  &  &  &  & 
\\
\hyperref[correct]{NFR: Correct} &  &  &  &  &  &  &  &  &  &  &  &  &  &  &  &  &  &  &  &  &  &  &  &  &  &  &  &  &  &  &  &  &  &  &  &  &  &  &  &  &  &  &  &  &  &  &  &  &  &  &  &  &  &  &  & 
\\
\hyperref[understandable]{NFR: Understandable} &  &  &  &  &  &  &  &  &  &  &  &  &  &  &  &  &  &  &  &  &  &  &  &  &  &  &  &  &  &  &  &  &  &  &  &  &  &  &  &  &  &  &  &  &  &  &  &  &  &  &  &  &  &  &  & 
\\
\hyperref[reusable]{NFR: Reusable} &  &  &  &  &  &  &  &  &  &  &  &  &  &  &  &  &  &  &  &  &  &  &  &  &  &  &  &  &  &  &  &  &  &  &  &  &  &  &  &  &  &  &  &  &  &  &  &  &  &  &  &  &  &  &  & 
\\
\hyperref[maintainable]{NFR: Maintainable} &  &  &  &  &  &  &  &  &  &  &  &  &  &  &  &  &  &  &  &  &  &  &  &  &  &  &  &  &  &  &  &  &  &  &  &  &  &  &  &  &  &  &  &  &  &  &  &  &  &  &  &  &  &  &  & 
\\
\bottomrule
\caption{Traceability Matrix Showing the Connections Between Requirements, Goal Statements and Other Items}
\label{Table:TraceMatAllvsR}
\end{longtable}
\section{Values of Auxiliary Constants}
\label{Sec:AuxConstants}
There are no auxiliary constants.

\section{References}
\label{Sec:References}
\begin{filecontents*}{bibfile.bib}
@article{fredlund1977,
author={Fredlund, D. G. and Krahn, J.},
title={Comparison of slope stability methods of analysis},
journal={Canadian Geotechnical Journal},
year={1977},
month=apr,
pages={429--439},
volume={14},
number={3}}
@book{huston2008,
author={Huston, Ronald and Josephs, Harold},
title={Practical stress analysis in engineering design},
publisher={CRC Press},
year={2008},
edition={3}}
@inproceedings{karchewski2012,
author={Karchewski, Brandon and Guo, Peijun and Stolle, Dieter},
title={Influence of inherent anisotropy of soil strength on limit equilibrium slope stability analysis},
booktitle={Proceedings of the 65th annual Canadian GeoTechnical Conference},
year={2012},
address={Winnipeg, MB, Canada},
organization={Canadian Geotechnical Society}}
@mastersthesis{koothoor2013,
author={Koothoor, Nirmitha},
title={A document drive approach to certifying scientific computing software},
school={McMaster University},
year={2013},
address={Hamilton, ON, Canada}}
@article{morgenstern1965,
author={Morgenstern, N. R. and Price, P. E.},
title={The analysis of the stability of general slip surfaces},
journal={Géotechnique},
year={1965},
month=jan,
pages={79--93},
number={15}}
@article{parnasClements1986,
author={Parnas, David L. and Clements, P. C.},
title={A rational design process: How and why to fake it},
journal={IEEE Transactions on Software Engineering},
year={1986},
month=feb,
volume={12},
number={2},
pages={251--257},
address={Washington, USA}}
@article{chen2005,
author={Qian, Q. H. and Zhu, D. Y. and Lee, C. F. and Chen, G. R.},
title={A concise algorithm for computing the factor of safety using the morgenstern price method},
journal={Canadian Geotechnical Journal},
year={2005},
month=feb,
volume={42},
number={1},
pages={272--278}}
@inproceedings{smithLai2005,
author={Smith, W. Spencer and Lai, Lei},
title={A new requirements template for scientific computing},
booktitle={Proceedings of the First International Workshop on Situational Requirements Engineering Processes - Methods, Techniques and Tools to Support Situation-Specific Requirements Engineering Processes, SREP'05},
year={2005},
editor={Agerfalk, PJ and Kraiem, N. and Ralyte, J.},
address={Paris, France},
pages={107--121},
note={In conjunction with 13th IEEE International Requirements Engineering Conference,}}
@article{li2010,
author={Yu-Chao, Li and Yun-Min, Chen and Zhan, Tony L. T. and Sao-Sheng, Ling and Cleall, Peter John},
title={An efficient approach for locating the critical slip surface in slope stability analyses using a real-coded genetic algorithm},
journal={Canadian Geotechnical Journal},
year={2010},
month=jun,
pages={806--820},
volume={47},
number={7}}
\end{filecontents*}
\nocite{*}
\bibstyle{ieeetr}
\printbibliography[heading=none]
\end{document}
